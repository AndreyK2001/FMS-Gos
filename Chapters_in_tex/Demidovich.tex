\section{Тензор деформаций Грина. Формулы выражения компонент тензора Грина через компоненты вектора перемещения. Вычисление с помощью тензора Грина: относительного изменения длины элементарного волокна; изменения угла между двумя элементарными волокнами; относительного изменения элементарного объема. Тензор малых деформаций Коши и тензор малых поворотов.}
Деформацию тела можно изучать пользуясь двумя основными подходами МДТТ – подходом Лагранжа и подходом Эйлера. При Лагранжевом подходе для описания деформации используется \textbf{тензор деформации Грина.} Знание его компонентов позволяет вычислять длины отрезков и углы между отрезками после деформации. (При Эйлеровом подходе используется тензор деформаций Альманси. Для описания малых деформаций применяется тензор малых деформаций Коши.)

\begin{figure}[h!]
  \centering
  \includegraphics[width=0.6\textwidth]{images/10.1.jpg}
\end{figure}


Рассмотрим два состояния и не будем записывать зависимость от времени. Начальный радиус-вектор выражен через криволинейные координаты:
$$
\vec{R}=\vec{R}(\vec{\xi}) \Leftrightarrow X_i=X_i\left(\xi^1, \xi^2, \xi^3\right)
$$
При Лагранжевском подходе исходим из закона движения:

$$
\vec{r}=\vec{r}(\vec{\xi}) \Leftrightarrow x_i=x_i\left(\xi^1, \xi^2, \xi^3\right)
$$
Элементарный вектор до и после деформации:
$$
\begin{gathered}
d \vec{R}=\frac{\partial \vec{R}}{\partial \xi^i} d \xi^i=\vec{E}_i d \xi^i, d \vec{r}=\frac{\partial \vec{r}}{\partial \xi^i} d \xi^i=\vec{e}_i d \xi^i \\
|d \vec{R}|^2=d \vec{R} \cdot d \vec{R}=\left(\vec{E}_i d \xi^i\right) \cdot\left(\vec{E}_j d \xi^j\right)=\left(\vec{E}_i \cdot \vec{E}_j\right) d \xi^i d \xi^j=G_{i j} d \xi^i d \xi^j \\
|d \vec{r}|^2=d \vec{r} \cdot d \vec{r}=\left(\vec{e}_i d \xi^i\right) \cdot\left(\vec{e}_j d \xi^j\right)=\left(\vec{e}_i \cdot \vec{e}_j\right) d \xi^i d \xi^j=g_{i j} d \xi^i d \xi^j \\
|d \vec{r}|^2-|d \vec{R}|^2=g_{i j} d \xi^i d \xi^j-G_{i j} d \xi^i d \xi^j=2 \Gamma_{i j} d \xi^i d \xi^j \\
\Gamma_{i j}=\frac{g_{i j}-G_{i j}}{2}
\end{gathered}
$$

$
\displaystyle
\text { Материальный градиент: } \vec{\nabla}_{\bar{R}} \equiv \vec{E}^\alpha \partial / \partial \xi^\alpha$Выразим тензор Грина через вектор перемещения:


$
\displaystyle
\vec{r}=\vec{R}+\vec{u} \Rightarrow \vec{\nabla}_{\vec{R}} \overrightarrow{\mathbf{r}}=\vec{\nabla}_{\vec{R}} \overrightarrow{\mathbf{R}}+\vec{\nabla}_{\vec{R}} \overrightarrow{\mathbf{u}} \Leftrightarrow \vec{E}^\alpha \frac{\partial \overrightarrow{\mathbf{r}}}{\partial \xi^\alpha}=\underset{\sim}{\mathbf{E}}+\vec{\nabla}_{\vec{R}} \overrightarrow{\mathbf{u}} \Leftrightarrow \overrightarrow{\mathbf{E}}^\alpha \overrightarrow{\mathbf{e}}_\alpha=\underset{\sim}{\mathbf{E}}+\vec{\nabla}_{\vec{R}} \overrightarrow{\mathbf{u}}, \quad\left(\overrightarrow{\mathbf{E}}^\beta \overrightarrow{\mathbf{e}}_\beta\right)^{\mathbf{T}}=\overrightarrow{\mathbf{e}}_\beta \overrightarrow{\mathbf{E}}^\beta=\underset{\sim}{\mathbf{E}}+\overrightarrow{\mathbf{u}} \vec{\nabla}_{\bar{R}}
$

$\underset{\sim}{\mathbf{g}_{\bar{R}}}=g_{\alpha \beta} \overrightarrow{\mathbf{E}}^\alpha \overrightarrow{\mathbf{E}}^\beta=\underline{\overrightarrow{\mathbf{E}}^\alpha \overrightarrow{\mathbf{e}}_\alpha \cdot \overrightarrow{\mathbf{e}}_\beta \overrightarrow{\mathbf{E}}^\beta}=\left(\underset{\sim}{\mathbf{E}}+\vec{\nabla}_{\vec{R}} \overrightarrow{\mathbf{u}}\right) \cdot\left(\underset{\sim}{\mathbf{E}}+\overrightarrow{\mathbf{u}} \vec{\nabla}_{\bar{R}}\right)=\underset{\sim}{\mathbf{E}}+\vec{\nabla}_{\bar{R}} \overrightarrow{\mathbf{u}}+\overrightarrow{\mathbf{u}} \vec{\nabla}_{\vec{R}}+\vec{\nabla}_{\vec{R}} \overrightarrow{\mathbf{u}} \cdot \overrightarrow{\mathbf{u}} \vec{\nabla}_{\vec{R}}$

$\displaystyle \underset{\sim}{\Gamma}=\frac{1}{2}\left(\mathbf{g}_{\vec{R}}-\underset{\sim}{\mathbf{E}}\right) \Rightarrow \underset{\sim}{\Gamma}=\frac{1}{2}\left(\vec{\nabla}_{\vec{R}} \overrightarrow{\mathbf{u}}+\overrightarrow{\mathbf{u}} \vec{\nabla}_{\vec{R}}+\vec{\nabla}_{\vec{R}} \overrightarrow{\mathbf{u}} \cdot \overrightarrow{\mathbf{u}} \vec{\nabla}_{\vec{R}}\right) \Leftrightarrow \Gamma_{i j}=\frac{1}{2}\left(u_{j, i}+u_{i, j}+u_{q, i} u_{, j}^q\right)$

Обозначим единичные направляющие вектора до и после деформирования как $\vec{\tau}, \hat\vec{\tau}$

\begin{figure}[h!]
  \centering
  \includegraphics[width=0.2\textwidth]{images/10.2.jpg}
\end{figure}


$$
\begin{aligned}
& \vec{\tau}=\left(\tau^1, \tau^2, \tau^3\right)_{\vec{R}}=\frac{d \vec{R}}{|d \vec{R}|}=\frac{d \xi^1}{|d \vec{R}|} \vec{E}_1+\frac{d \xi^2}{|d \vec{R}|} \vec{E}_2+\frac{d \xi^3}{|d \vec{R}|} \vec{E}_3 \Rightarrow \tau^i=\frac{d \xi^i}{|d \vec{R}|},|\vec{\tau}|=\sqrt{\tau^i \tau^j G_{i j}}=1 \\
& \hat{\vec{\tau}}=\left(\hat{\tau}_{\vec{r}}^1, \hat{\tau}_{\vec{r}}^2, \hat{\tau}_{\vec{r}}^3\right)_{\vec{r}}=\frac{d \vec{r}}{|d \vec{r}|}=\frac{d \xi^1}{|d \vec{r}|} \vec{e}_1+\frac{d \xi^2}{|d \vec{r}|} \vec{e}_2+\frac{d \xi^3}{|d \vec{r}|} \vec{e}_3 \Rightarrow \hat{\tau}_{\vec{r}}^i=\frac{d \xi^i}{|d \vec{r}|},|\hat{\vec{\tau}}|=\sqrt{\hat{\tau}_{\vec{r}}^i \hat{\tau}_{\vec{r}}^j g_{i j}}=1 \\
&|d \vec{r}|^2-|d \vec{R}|^2=2 \Gamma_{i j} d \xi^i d \xi^j \Leftrightarrow \frac{|d \vec{r}|}{|d \vec{R}|}=\sqrt{1+2 \Gamma_{i j} \frac{d \xi^i}{|d \vec{R}|} \frac{d \xi^j}{|d \vec{R}|}}=\sqrt{1+2 \Gamma_{i j} \tau^i \tau^j} \Leftrightarrow\\
&|d \vec{r}|=\sqrt{1+2 \cdot \Gamma_{i j} \tau^i \tau^j|d \vec{R}|}
\end{aligned}
$$
 Относительное изменение длины элементарного волокна в заданном направлении $\vec{\tau}$:

$$
\lambda(\vec{\tau})=\frac{|d \vec{r}|-|d \vec{R}|}{|d \vec{R}|}=\frac{|d \vec{r}|}{|d \vec{R}|}-1=\sqrt{1+2 \cdot \Gamma_{i j} \tau^i \tau^j}-1 \Rightarrow \begin{aligned}
& \lambda(\vec{\tau})=\sqrt{1+2 \cdot \Gamma_{i j} \tau^i \tau^j}-1 
\end{aligned}
$$
\textbf{Изменения угла между двумя элементарными волокнами} 


Выразим компоненты единичных векторов через лагранжевы компоненты $ u^i$  вектора перемещения т.P  

\begin{figure}[h!]
  \centering
  \includegraphics[width=0.4\textwidth]{images/10.3.jpg} 
\end{figure}



$ d \vec{r}=d \vec{R}+d \vec{u}=\frac{\partial \vec{R}}{\partial \xi^m} d \xi^m+\frac{\partial \vec{u}}{\partial \xi^i} d \xi^i=\vec{E}_m d \xi^m+u_{, i}^m \vec{E}_m d \xi^i=\left(\delta_i^m d \xi^i+u_{, i}^m d \xi^i\right) \vec{E}_m=\left(\delta_i^m+u_{, i}^m\right) d \xi^i \vec{E}_m$


$
\begin{gathered}
\frac{d \xi^i}{|d \vec{R}|}=\tau^i, \quad|d \vec{r}|=\sqrt{1+2 \cdot \Gamma_{i j} \tau^i \tau^j}|d \vec{R}|, \quad d \vec{r}=\left(\delta_i^m+u_i^m\right) d \xi^i \cdot \vec{E}_m \\
\vec{\tau}=\tau^i \overrightarrow{E_i}, \quad \hat{\vec{\tau}}=\frac{d \vec{r}}{|d \vec{r}|}=\frac{\left(\delta_i^m+u_i^m\right) d \xi^i \cdot \vec{E}_m}{\sqrt{1+2 \Gamma_{i j} \tau^i \tau^j}|d \vec{R}|}=\frac{\left(\delta_i^m+u_i^m\right) \tau^i}{1+\lambda(\vec{\tau})} \vec{E}_m \equiv \hat{\tau}^m \vec{E}_m
\end{gathered}
$
$
\begin{aligned}
& \vec{\mu}=\mu^j \vec{E}_j, \quad \hat{\vec{\mu}}=\frac{\left(\delta_j^m+u_{, j}^m\right) \mu^j}{1+\lambda(\vec{\mu})} \vec{E}_m=\frac{\left(\delta_j^m+u_{, j}^m\right) \mu^j}{1+\lambda(\vec{\mu})} G_{m k} \vec{E}^k=\frac{\left(\delta_j^m G_{m k}+u_{, j}^m G_{m k}\right) \mu^j}{1+\lambda(\vec{\mu})} \vec{E}^k=\\
& =\frac{\left(G_{j k}+u_{k, j}\right) \mu^j}{1+\lambda(\vec{\mu})} \vec{E}^k \equiv \hat{\mu}_k \vec{E}^k \\
& \cos \hat{\varphi}=\hat{\tau}^k \hat{\mu}_k=\frac{\left(\delta_i^k+u_{, i}^k\right) \tau^i}{1+\lambda(\vec{\tau})} \frac{\left(G_{j k}+u_{k, j}\right) \mu^j}{1+\lambda(\vec{\mu})}=\frac{\left(G_{j k} \delta_i^k+\delta_i^k u_{k, j}+u_{, i}^k G_{j k}+u_{i, i}^k u_{k, j}\right) \tau^i \mu^j}{(1+\lambda(\vec{\tau}))(1+\lambda(\vec{\mu}))}=\\
& =\frac{\left(G_{j i}+u_{i, j}+u_{j, i}+u_{i, i}^k u_{k, j}\right) \tau^i \mu^j}{(1+\lambda(\vec{\tau}))(1+\lambda(\vec{\mu}))}
\end{aligned}
$
$\begin{aligned} \cos \hat{\varphi} & =\frac{\left(G_{j i}+u_{i, j}+u_{j, i}+u_{j i}^k u_{k, j}\right) \tau^i \mu^j}{(1+\lambda(\vec{\tau}))(1+\lambda(\vec{\mu}))}=\frac{\left(G_{i j}+2 \cdot \Gamma_{i j}\right) \tau^i \mu^j}{(1+\lambda(\vec{\tau}))(1+\lambda(\vec{\mu}))} =\frac{G_{i j} \tau^i \mu^j+2 \cdot \Gamma_{i j} \tau^i \mu^j}{(1+\lambda(\vec{\tau}))(1+\lambda(\vec{\mu}))} \\ \cos \hat{\varphi} & =\frac{\cos \varphi+2 \cdot \Gamma_{i j} \tau^i \mu^j}{(1+\lambda(\vec{\tau}))(1+\lambda(\vec{\mu}))}.\end{aligned}$


\textbf{Относительного изменения элементарного объема (дилатация)}


В точке $\left(\xi^1, \xi^2, \xi^3\right) \in V_0$ рассмотрим тройку элементарных векторов $d \vec{R}_{(\alpha)}=d \xi_{(\alpha)}^i \vec{E}_i, \alpha=1,2,3$ Объем косого параллелепипеда, построенного на этих векторах равен: 

\begin{figure}[h!]
  \centering
  \includegraphics[width=0.3\textwidth]{images/10.4.jpg} 
\end{figure}

$$
\begin{aligned}
& d V_0=d \vec{R}_{(1)} \cdot\left(d \vec{R}_{(2)} \times d \vec{R}_{(3)}\right)=\left(d \xi_{(1)}^i \vec{E}_i\right) \cdot\left(\left(d \xi_{(2)}^j \vec{E}_j\right) \times\left(d \xi_{(3)}^k \vec{E}_k\right)\right)= \\
& =\vec{E}_i \cdot\left(\vec{E}_j \times \vec{E}_k\right) d \xi_{(1)}^i d \xi_{(2)}^j d \xi_{(3)}^k=\Xi_{i j k} d \xi_{(1)}^i d \xi_{(2)}^j d \xi_{(3)}^k=\sqrt{G} \cdot e_{i j k} d \xi_{(1)}^i d \xi_{(2)}^j d \xi_{(3)}^k
\end{aligned}
$$
После деформации косой параллелепипед становится построенным на векторах $d \vec{r}_{(\alpha)}=d \xi_{(\alpha)}^i \vec{e}_i$, в которые переходят векторы $d \vec{R}_{(\alpha)}$. Частицы, расположенные до деформации в объеме $d V_0$, после деформации займут объем $d V$, равный:
$$
\begin{aligned}
& d V=d \vec{r}_{(1)} \cdot\left(d \vec{r}_{(2)} \times d \vec{r}_{(3)}\right)=\left(d \xi_{(1)}^i \vec{e}_i\right) \cdot\left(\left(d \xi_{(2)}^j \vec{e}_j\right) \times\left(d \xi_{(3)}^k \vec{e}_k\right)\right)= \\
& =\vec{e}_i \cdot\left(\vec{e}_j \times \vec{e}_k\right) d \xi_{(1)}^i d \xi_{(2)}^j d \xi_{(3)}^k=\varepsilon_{i j k} d \xi_{(1)}^i d \xi_{(2)}^j d \xi_{(3)}^k=\sqrt{g} \cdot e_{i j k} d \xi_{(1)}^i d \xi_{(2)}^j d \xi_{(3)}^k\\
&
 \theta \equiv \frac{d V-d V_0}{d V_0}=\frac{d V}{d V_0}-1=\sqrt{\frac{g}{G}}-1
\end{aligned}
$$
$\displaystyle
\sqrt{g}=\vec{e}_1 \cdot\left(\vec{e}_2 \times \vec{e}_3\right)=\left[\left(\delta_1^i+u_{, 1}^i\right) \vec{E}_i\right] \cdot\left[\left(\left(\delta_2^j+u_{, 2}^j\right) \vec{E}_j\right) \times\left(\left(\delta_3^k+u_{, 3}^k\right) \vec{E}_k\right)\right]=\left(\delta_1^i+u_{, 1}^i\right)\left(\delta_2^j+u_{, 2}^j\right)\left(\delta_3^k+u_{, 3}^k\right) \vec{E}_i \cdot\left(\vec{E}_j \times \vec{E}_k\right) =\left(\delta_1^i+u_{, 1}^i\right)\left(\delta_2^j+u_{, 2}^j\right)\left(\delta_3^k+u_{, 3}^k\right) \Xi_{i j k}=\left(\delta_1^i+u_{, 1}^i\right)\left(\delta_2^j+u_{, 2}^j\right)\left(\delta_3^k+u_{, 3}^k\right) e_{i j k} \sqrt{G}=\mathrm{det}\left(\delta_p^q+u_{, p}^q\right) \sqrt{G}=\mathrm{det}\left[\underset{\sim}{\mathbf{E}}+\vec{\nabla}_{\vec{R}} \overrightarrow{\mathbf{u}}\right] \sqrt{G}=\mathrm{det} \underset{\sim}{\mathbf{B}} \cdot \sqrt{G}, \quad \underset{\sim}{\mathbf{B}} \equiv \underset{\sim}{\mathbf{E}}+\vec{\nabla}_{\vec{R}} \overrightarrow{\mathbf{u}}
$
$$
\left[B_i^{\cdot j}\right]=\left(\begin{array}{ccc}
1+u_{, 1}^1 & u_{, 1}^2 & u_{, 1}^3 \\
u_{, 2}^1 & 1+u_{, 2}^2 & u_{, 2}^3 \\
u_{, 3}^1 & u_{, 3}^2 & 1+u_{, 3}^3
\end{array}\right)
$$
$
\displaystyle
\mathrm{det}^2 \underset{\sim}{\mathbf{B}}=\mathrm{det} \underset{\sim}{\mathbf{B}} \cdot \mathrm{det} \underset{\sim}{\mathbf{B}}=\mathrm{det} \underset{\sim}{\mathbf{B}} \cdot \mathrm{det} \underset{\sim}{\mathbf{B}}=\mathrm{det}(\underset{\sim}{\mathbf{B}} \cdot \underset{\sim}{\mathbf{B}})=\mathrm{det}\left(\underset{\sim}{\mathbf{E}}+\vec{\nabla}_{\vec{R}} \overrightarrow{\mathbf{u}}\right) \cdot\left(\underset{\sim}{\mathbf{E}}+\vec{\nabla}_{\vec{R}} \overrightarrow{\mathbf{u}}\right)^{\mathrm{T}}=\mathrm{det}\left(\underset{\sim}{\mathbf{E}}+\vec{\nabla}_{\vec{R}} \overrightarrow{\mathbf{u}}\right) \cdot\left(\underset{\sim}{\mathbf{E}}+\overrightarrow{\mathbf{u}} \vec{\nabla}_{\vec{R}}\right)=\mathrm{det}\left(\underset{\sim}{\mathbf{E}}+\vec{\nabla}_{\vec{R}} \overrightarrow{\mathbf{u}}+\overrightarrow{\mathbf{u}} \vec{\nabla}_{\bar{R}}+\vec{\nabla}_{\bar{R}} \overrightarrow{\mathbf{u}} \cdot \overrightarrow{\mathbf{u}}_{\bar{R}}\right)=\mathrm{det}\left(\underset{\sim}{\mathbf{E}}+\vec{\nabla}_{\vec{R}} \overrightarrow{\mathbf{u}}+\overrightarrow{\mathbf{u}} \vec{\nabla}_{\vec{R}}+\vec{\nabla}_{\vec{R}} \overrightarrow{\mathbf{u}} \cdot \overrightarrow{\mathbf{u}}_{\bar{R}}\right)=\mathrm{det}\left(\underset{\sim}{\mathbf{E}}+2 \cdot \Gamma_{\sim}\right) g / G=\mathrm{det}^2 \underset{\sim}{\mathbf{B}}=\mathrm{det}(\underset{\sim}{\mathbf{E}}+2 \cdot \underset{\sim}{\Gamma})=\mathrm{det}\left[\delta_j^i+2 \Gamma_j^i\right]=e_{i j k}\left(\delta_1^i+2 \Gamma_1^i\right)\left(\delta_2^j+2 \Gamma_2^j\right)\left(\delta_3^k+2 \Gamma_3^k\right)=e_{i j k} \delta_1^i \delta_2^j \delta_3^k +8 e_{i j k} \Gamma_1^i \Gamma_2^j \Gamma_3^k+2 e_{i j k}\left(\delta_1^i \delta_2^j \Gamma_3^k+\delta_1^i \delta_3^k \Gamma_2^j+\delta_2^j \delta_3^k \Gamma_1^i\right)+4 e_{i j k}\left(\delta_1^i \Gamma_2^j \Gamma_3^k+\delta_2^j \Gamma_1^i \Gamma_3^k+\delta_3^k \Gamma_1^i \Gamma_2^j\right)=e_{123}+8 \mathrm{det}\left\|\Gamma_j^i\right\| +2\left(e_{12 k} \Gamma_3^k+e_{1 j 3} \Gamma_2^j+e_{i 23} \Gamma_1^i\right)+4\left(e_{1 j k} \Gamma_2^j \Gamma_3^k+e_{i 2 k} \Gamma_1^i \Gamma_3^k+e_{i j 3} \Gamma_1^i \Gamma_2^j\right)=1+8 \Pi_{\Gamma}+2\left(\Gamma_3^3+\Gamma_2^2+\Gamma_1^1\right) +2\left(2\left(\Gamma_2^2 \Gamma_3^3-\Gamma_2^3 \Gamma_3^2\right)+2\left(\Gamma_1^1 \Gamma_3^3-\Gamma_1^3 \Gamma_3^1\right)+2\left(\Gamma_1^1 \Gamma_2^2-\Gamma_2^2 \Gamma_1^1\right)\right)=1+2 \mathrm{I}_{\Gamma}+4 \Pi_{\Gamma}+8 \Pi_{\Gamma} \text {. } $
$$
\theta \equiv \frac{d V-d V_0}{d V_0}=\sqrt{\frac{g}{G}}-1=\sqrt{1+2 \mathrm{I}_{\Gamma}+4 \mathrm{I}_{\Gamma}+8 \mathrm{III}_{\Gamma}}-1
$$


Для описания малых деформаций
применяется тензор малых деформаций Коши: Деформации и вращения считаются малыми если все компоненты тензора дисторсии существенно меньше единицы по абсолютной величине $\left|d_{i j}\right|=\left|u_{j, i}\right| \ll 1$. В этом случае в $\Gamma_{i j}=\frac{1}{2}\left(u_{j, i}+u_{i, j}+u_{q, i} u_{, j}^q\right)$ можно пренебречь квадратичными слагаемыми по сравнению с линейными составляющими и тогда $\Gamma_{i j} \approx \varepsilon_{i j}=\frac{1}{2}\left(u_{j, i}+u_{i, j}\right)$. 


Получаем: \textbf{тензор малых деформаций Коши} $\varepsilon_{i j}=1 / 2\left(u_{j, i}+u_{i, j}\right)$ 
\textbf{, тензор малых поворотов}: $\omega_{i j}=1 / 2\left(u_{j, i}-u_{i, j}\right)$


\section{Формулы Чезаро. Уравнения совместности Сен-Венана.}
\textbf{Формулы Чезаро:}


Во всех точках тела известно поле деформаций $\varepsilon_{i j}\left(x_i\right)$. В точке $O\left(x_i^o\right)$ известны перемещения $u_i^O$ и вращения $\omega_{i j}^O$. Требуется найти перемещения в точке $P\left(x_i^P\right)$. 

\begin{figure}[h!]
  \centering
  \includegraphics[width=0.3\textwidth]{images/11.1.jpg} 
\end{figure}


Перемещения в точке бесконечно близкой к точке О можно найти по формуле $u_i=u_i^O+d u_i=u_i^O+u_{i, j} d x_j=u_i^O+\left(\varepsilon_{i j}+\omega_{j i}\right) d x_j$. Соединим точки $O$ и $P$ кривой, лежащей целиком в теле. Тогда перемещения в произвольной точке $P$ будут равны
$$
u_i^p=u_i^O+\int_{O P} d u_i=u_i^O+\int_{x_j^O}^{x_j^P} u_{i, j} d x_j=u_i^O+\int_O^P\left(\varepsilon_{i j}+\omega_{j i}\right) d x_j=u_i^O+\int_O^P \varepsilon_{i j} d x_j+\int_O^P \omega_{j i} d x_j
$$
Сначала докажем тождество: $\omega_{j i, k} \equiv \varepsilon_{i k, j}-\varepsilon_{k j, i}$. 


Используем: $\varepsilon_{i j} \equiv \frac{1}{2}\left(u_{i, j}+u_{j, i}\right), \omega_{j i} \equiv \frac{1}{2}\left(u_{i, j}-u_{j, i}\right)$.
$$
\begin{aligned}
\omega_{j i, k} & =\frac{1}{2}\left(u_{i, j k}-u_{j, i k}\right)=\frac{1}{2}\left(u_{i, j k}+u_{k, j i}-u_{k, j i}-u_{j, i k}\right)=\frac{1}{2}\left(u_{i, k j}+u_{k, i j}\right)- \\
& -\frac{1}{2}\left(u_{k, j i}+u_{j, k i}\right)=\frac{1}{2}\left(u_{i, k}+u_{k, i}\right)_{, j}-\frac{1}{2}\left(u_{k, j}+u_{j, k}\right)_{, i}=\varepsilon_{i k, j}-\varepsilon_{k j, i}
\end{aligned}
$$
Проинтегрируем по частям, используя  $\omega_{j i}^P-\omega_{j i}^o=\int_0^P \omega_{j i, k} d x_k$
$
\displaystyle
\int_O^P \omega_{j i} d x_j=\left.\omega_{j i} x_j\right|_O ^P-\int_O^P x_j \omega_{j i, k} d x_k=\omega_{j i}^P x_j^P-\omega_{j i}^O x_j^O-\int_O^P x_j \omega_{j i, k} d x_k=\omega_{j i}^O\left(x_j^P-x_j^O\right)+\left(\omega_{j i}^P-\omega_{j i}^O\right) x_j^P
-\int_O^P x_j \omega_{j i, k} d x_k=\omega_{j i}^O\left(x_j^P-x_j^O\right)+x_j^P \int_O^P \omega_{j i, k} d x_k-\int_O^P x_j \omega_{j i, k} d x_k=\omega_{j i}^O\left(x_j^P-x_j^O\right)+\int_O^P\left(x_j^P-x_j\right) \omega_{j i, k} d x_k 
=\omega_{j i}^O\left(x_j^P-x_j^O\right)+\int_O^P\left(x_j^P-x_j\right)\left(\varepsilon_{i k, j}-\varepsilon_{k j, i}\right) d x_k \stackrel{j \leftrightarrow k}{=} \omega_{j i}^O\left(x_j^P-x_j^O\right)+\int_O^P\left(x_k^P-x_k\right)\left(\varepsilon_{i j, k}-\varepsilon_{j k, i}\right) d x_j.$


$ \displaystyle
u_i^P=u_i^O+\omega_{j i}^O\left(x_j^P-x_j^O\right)+\int_O^P\left[\varepsilon_{i j}+\left(x_k^P-x_k\right)\left(\varepsilon_{i j, k}-\varepsilon_{j k, i}\right)\right] d x_j.
$


\textbf{Соотношения (уравнения) совместности Сен-Венана:}


 Обозначим $\Psi_{i j}=\varepsilon_{i j}+\left(x_k^P-x_k\right)\left(\varepsilon_{i j, k}-\varepsilon_{j k, i}\right)$. Для того, чтобы $\vec{u}$ было векторным полем значение интеграла в формулах Чезаро не должно зависеть от выбора пути интегрирования, а зависит только от выбора точек $O$ и P, т.е: $\oint \Psi_{i j} d x_j=0 \Leftrightarrow \Psi_{i j, l}=\Psi_{i l,j}$
$$
\begin{aligned}
& \left(\varepsilon_{i j}+\left(x_k^P-x_k\right)\left(\varepsilon_{i j, k}-\varepsilon_{j k, i}\right)\right)_{, l}=\left(\varepsilon_{i l}+\left(x_k^P-x_k\right)\left(\varepsilon_{i l, k}-\varepsilon_{l k, i}\right)\right)_{, j} \\
& \varepsilon_{i j, l}-x_{k, l}\left(\varepsilon_{i j, k}-\varepsilon_{j k, i}\right)+\left(x_k^P-x_k\right)\left(\varepsilon_{i j, k l}-\varepsilon_{j k, i l}\right)=\varepsilon_{i l, j}-x_{k, j}\left(\varepsilon_{i l, k}-\varepsilon_{l k, i}\right)+ \\&+\left(x_k^P-x_k\right)\left(\varepsilon_{i l, k j}-\varepsilon_{l k, i j}\right) \\
& \left(x_k^P-x_k\right)\left(\varepsilon_{i j, k l}-\varepsilon_{j k, i l}-\varepsilon_{i l, k j}+\varepsilon_{l k, i j}\right)=\varepsilon_{i l, j}-\delta_{k j}\left(\varepsilon_{i l, k}-\varepsilon_{l k, i}\right)-\varepsilon_{i j, l}+\delta_{k l}\left(\varepsilon_{i j, k}-\varepsilon_{j k, i}\right) \\
& \left(x_k^P-x_k\right)\left(\varepsilon_{i j, k l}-\varepsilon_{j k, i l}-\varepsilon_{i l, k j}+\varepsilon_{l k, i j}\right)=\varepsilon_{i l, j}-\varepsilon_{i l, j}+\varepsilon_{l j, i}-\varepsilon_{i j, l}+\varepsilon_{i j, l}-\varepsilon_{j l, i}=0 \\
& \text { Получаем } 81 \text { формулу: } \quad \varepsilon_{i j, k l}-\varepsilon_{j k, i l}-\varepsilon_{i l, k j}+\varepsilon_{l k, i j}=0, i, j, k, l=1,2,3.
\end{aligned}
$$


Обозначим $\chi_{i j k l} \equiv \varepsilon_{i j, k l}-\varepsilon_{j k, i l}-\varepsilon_{i l, k j}+\varepsilon_{l k, i j}$ и определим его независимые компоненты. 

Переставляем индексы исходя из симметрии:


а) индексы $1 \leftrightarrow 3: \quad \chi_{k j i l}=\varepsilon_{k j, i l}-\varepsilon_{j i, k l}-\varepsilon_{k l, i j}+\varepsilon_{l i, k j}=-\chi_{i j k l}$. (Независимых 27)


б) индексы $2 \leftrightarrow 4: \quad \chi_{i l k j}=\varepsilon_{i l, k j}-\varepsilon_{l k, i j}-\varepsilon_{i j, k l}+\varepsilon_{j k, i l}=-\chi_{i j k l}$. (Независимых 9)
в) индексы $1 \leftrightarrow 2$ и $3 \leftrightarrow 4: \chi_{j i l k}=\varepsilon_{j i, l k}-\varepsilon_{i l, j k}-\varepsilon_{j k, l i}+\varepsilon_{k l, j i}=\chi_{i j k l}$. (Независимых 6)
Независимые компоненты: $\chi_{\underline{1} \underline{12} 2}, \chi_{\underline{1} 123}, \chi_{\underline{1} 1 \underline{3} 3}, \chi_{\underline{1} 2 \underline{2} 3}, \chi_{\underline{2} 2 \underline{3} 3}, \chi_{\underline{1} 2 \underline{3} 3}$.
$$
\begin{aligned}
& \chi_{1122}=\varepsilon_{11,22}-\varepsilon_{12,12}-\varepsilon_{12,21}+\varepsilon_{22,11}=0 ; \quad \chi_{1123}=\varepsilon_{11,23}-\varepsilon_{12,13}-\varepsilon_{13,21}+\varepsilon_{32,11}=0 \\
& \chi_{1133}=\varepsilon_{11,33}-\varepsilon_{13,13}-\varepsilon_{13,31}+\varepsilon_{33,11}=0 ; \quad \chi_{1223}=\varepsilon_{12,23}-\varepsilon_{22,13}-\varepsilon_{13,22}+\varepsilon_{32,12}=0 \\
& \chi_{2233}=\varepsilon_{22,33}-\varepsilon_{23,23}-\varepsilon_{23,32}+\varepsilon_{33,22}=0 ; \quad \chi_{1233}=\varepsilon_{12,33}-\varepsilon_{23,13}-\varepsilon_{13,23}+\varepsilon_{33,12}=0 \\
& \left\{\begin{array}{l}
\varepsilon_{11,22}+\varepsilon_{22,11}=2 \varepsilon_{12,12} \\
\varepsilon_{11,33}+\varepsilon_{33,11}=2 \varepsilon_{13,13} \\
\varepsilon_{22,33}+\varepsilon_{33,22}=2 \varepsilon_{23,23}
\end{array} ;\left\{\begin{array}{l}
\left(\varepsilon_{12,3}-\varepsilon_{23,1}+\varepsilon_{31,2}\right)_{, 1}=\varepsilon_{11,23} \\
\left(\varepsilon_{12,3}+\varepsilon_{23,1}-\varepsilon_{31,2}\right)_{, 2}=\varepsilon_{22,13} \\
\left(-\varepsilon_{12,3}+\varepsilon_{23,1}+\varepsilon_{31,2}\right)_{, 3}=\varepsilon_{33,12}
\end{array}\right.\right. 
\end{aligned}
$$
Соотношения
(уравнения)
Сен-Венана:
$$
\left\{\begin{array}{l}
\varepsilon_{11,22}+\varepsilon_{22,11}=2 \varepsilon_{12,12} \\
\varepsilon_{11,33}+\varepsilon_{33,11}=2 \varepsilon_{13,13} \\
\varepsilon_{22,33}+\varepsilon_{33,22}=2 \varepsilon_{23,23}
\end{array} ;\left\{\begin{array}{l}
\left(\varepsilon_{12,3}-\varepsilon_{23,1}+\varepsilon_{31,2}\right)_{, 1}=\varepsilon_{11,23} \\
\left(\varepsilon_{12,3}+\varepsilon_{23,1}-\varepsilon_{31,2}\right)_{, 2}=\varepsilon_{22,13} \\
\left(-\varepsilon_{12,3}+\varepsilon_{23,1}+\varepsilon_{31,2}\right)_{, 3}=\varepsilon_{33,12}
\end{array}\right.\right.
$$


\section{Обобщенный изотермический линейный закон Гука. Потенциал линейно-упругого тела.  Три общих типа симметрии упругих констант. Три специальных типа симметрии: ортотропный, трансверсально-изотропный и изотропный материалы. Материальные константы Ламе. Технические константы. Формулы, которые связывают константы Ламе и технические константы.}

\begin{figure}[h!]
  \centering
  \includegraphics[width=0.4\textwidth]{images/12.1.jpg} 
\end{figure}


В упругом теле приложение нагрузки приводит к мгновенному появлению деформаций в теле, которые не изменяются с течением времени и полностью исчезают после снятия нагрузки. Связь между напряжениями и деформациями может быть нелинейной. Однако многие материалы при небольших нагрузках показывают линейную зависимость напряжений от деформаций. Такие материалы называются \textbf{линейно-упругими материалами.}  При постоянной температуре линейные связи $\underset{\sim}{\sigma} \sim \underset{\sim}{\mathcal{E}}$ и $\underset{\sim}{\mathcal{E}} \sim \underset{\sim}{\sigma}$ осуществляются с помощью тензоров четвертого ранга: ( $\tenq{\mathbf{C}}$ - тензор упругих модулей; $ \tenq{\mathbf{J}}$ - тензор упругих податливостей). 
$$
\begin{aligned}
& \underset{\sim}{\sigma}=\tenq{\mathbf{C}}: \underset{\sim}{\mathcal{E}} \equiv {\tenq{\mathbf{C}}} \cdot \cdot \underset{\sim}{\mathcal{E}} \Leftrightarrow \underset{\sim}{\mathcal{E}}=\tenq{\mathbf{J}}: \underset{\sim}{\sigma} \equiv \tenq{\mathbf{J}} \cdot\cdot\underset{\sim}{\sigma}\\
& \sigma^{i j}(\vec{x})=C^{i j k l}(\vec{x}) \varepsilon_{k l}(\vec{x}) \Leftrightarrow \varepsilon_{k l}(\vec{x})=J_{k l i j}(\vec{x}) \sigma^{i j}(\vec{x}) \\
& \text{В декартовой системе координат:} \\
& C_{i j p q}=C_{i j q p}, \varepsilon_{p q}=\varepsilon_{q p} \Rightarrow \sigma_{i j}=C_{i j 11} \varepsilon_{11}+C_{i j 22} \varepsilon_{22}+C_{i j 33} \varepsilon_{33}+C_{i j 23}\left(2 \varepsilon_{23}\right)+\\
&+C_{i j 13}\left(2 \varepsilon_{13}\right)+C_{i j 12}\left(2 \varepsilon_{12}\right), i, j=1,2,3 \text{(зав. от 6 компонент } \varepsilon )\\
& (\varepsilon_{11}^*=\varepsilon_{11}, \varepsilon_{22}^*=\varepsilon_{22}, \varepsilon_{33}^*=\varepsilon_{33},  \varepsilon_{12}^*=2 \varepsilon_{12}, \varepsilon_{23}^*=2 \varepsilon_{23}, \varepsilon_{13}^*=2 \varepsilon_{13}) \\
\end{aligned}
$$

Изотермические потенциалы линейно-упругого материала:
$$
W\left(\varepsilon_{k l}^*\right): \sigma^{i j}=\frac{\partial W}{\partial \varepsilon_{i j}^*}, \quad \mathrm{~W}\left(\sigma^{i j}\right): \varepsilon_{i j}^*=\frac{\partial \mathrm{W}}{\partial \sigma^{i j}} \\
$$
 Три общих типа симметрии упругих констант:

 
 1.$\quad
 \begin{aligned} & \frac{\partial^2 W}{\partial \varepsilon_{k l}^* \partial \varepsilon_{i j}^*}=\frac{\partial}{\partial \varepsilon_{k l}^*}\left(\frac{\partial W}{\partial \varepsilon_{i j}^*}\right)=\frac{\partial \sigma^{i j}}{\partial \varepsilon_{k l}^*}=\frac{\partial C^{i j p q} \varepsilon_{p q}}{\partial \varepsilon_{k l}^*}=C^{i j p q} \delta_p^k \delta_q^l=C^{i j k l} \\ & \frac{\partial^2 W}{\partial \varepsilon_{k l}^* \partial \varepsilon_{i j}^*}=\frac{\partial}{\partial \varepsilon_{i j}^*}\left(\frac{\partial W}{\partial \varepsilon_{k l}^*}\right)=\frac{\partial \sigma^{k l}}{\partial \varepsilon_{i j}^*}=\frac{\partial C^{k l p q} \varepsilon_{p q}}{\partial \varepsilon_{i j}^*}=C^{k l p q} \delta_p^i \delta_q^j=C^{k l i j}\end{aligned} \Rightarrow$

 
 $ C^{i j k l}=C^{k l i j} \quad J_{i j k l}=J_{k l i j} \quad W=\frac{1}{2} C^{i j k l} \varepsilon_{i j} \varepsilon_{k l} \quad \mathrm{~W}=\frac{1}{2} J_{i j k l} \sigma^{i j} \sigma^{k l}=W $

2. Симметрирование: $\mathrm{Sym}_{34} \underset{\approx}{\mathbf{C}} \Leftrightarrow C^{i j(k l)}=\frac{1}{2}\left(C^{i j k l}+C^{i j l k}\right)$. 

Альтернирование: $\mathrm{Alt}_{34} \underset{\approx}{\mathbf{C}} \Leftrightarrow C^{i j[k]}=\frac{1}{2}\left(C^{i j k l}-C^{i j l k}\right)$

( свёртка антисимметрично и симметричного)$C^{i j[k l]} \varepsilon_{k l}=0$


 $\Rightarrow C^{i j k l} \varepsilon_{k l}=\left(C^{i j(k l)}+C^{i j[k]]}\right) \varepsilon_{k l}=C^{i j(k l)} \varepsilon_{k l} \Rightarrow C^{i j k l}=C^{i j(k l)}=C^{i j l k}$

 
$
 3. \quad\sigma^{i j}=\sigma^{j i} \Rightarrow C^{i j k l}=C^{j i k l} \quad C^{i j k l}=C^{j i k l}=C^{i j l k}=C^{k l i j} \quad J_{i j k l}=J_{j i k l}=J_{i j l k}=J_{k l i j}
$
Установленные типы симметрии оставляют 21 независимых компонент.
$$
\left(\begin{array}{l}
\sigma_1 \\
\sigma_2 \\
\sigma_3 \\
\sigma_4 \\
\sigma_5 \\
\sigma_6
\end{array}\right)=\left(\begin{array}{llllll}
C_{11} & C_{12} & C_{13} & C_{14} & C_{15} & C_{16} \\
C_{12} & C_{22} & C_{23} & C_{24} & C_{25} & C_{26} \\
C_{13} & C_{23} & C_{33} & C_{34} & C_{35} & C_{36} \\
C_{14} & C_{24} & C_{34} & C_{44} & C_{45} & C_{46} \\
C_{15} & C_{25} & C_{35} & C_{45} & C_{55} & C_{56} \\
C_{16} & C_{26} & C_{36} & C_{46} & C_{56} & C_{66}
\end{array}\right) \cdot\left(\begin{array}{c}
\varepsilon_1 \\
\varepsilon_2 \\
\varepsilon_3 \\
2 \varepsilon_4 \\
2 \varepsilon_5 \\
2 \varepsilon_6
\end{array}\right)
$$
(Правило соответствия индексов $(i, i) \leftrightarrow i, \quad(i, j) \leftrightarrow 9-(i+j)$)


\textbf{Ортотропный материал} --- это материал, у которого есть три перпендикулярные оси симметрии упругих свойств. Эти оси называются главными осями ортотропии. Ортотропный материал характеризуется девятью независимыми упругими константами:
$$
\left(\begin{array}{l}
\sigma_1 \\
\sigma_2 \\
\sigma_3 \\
\sigma_4 \\
\sigma_5 \\
\sigma_6
\end{array}\right)=\left(\begin{array}{cccccc}
C_{11} & C_{12} & C_{13} & 0 & 0 & 0 \\
C_{12} & C_{22} & C_{23} & 0 & 0 & 0 \\
C_{13} & C_{23} & C_{33} & 0 & 0 & 0 \\
0 & 0 & 0 & C_{44} & 0 & 0 \\
0 & 0 & 0 & 0 & C_{55} & 0 \\
0 & 0 & 0 & 0 & 0 & C_{66}
\end{array}\right) \cdot\left(\begin{array}{c}
\varepsilon_1 \\
\varepsilon_2 \\
\varepsilon_3 \\
2 \varepsilon_4 \\
2 \varepsilon_5 \\
2 \varepsilon_6
\end{array}\right)
$$


\textbf{Трансверсально-изотропный} –-- это ортотропный материал, у которого одна из плоскостей
ортотропии является плоскостью упругой изотропии (т.е. любое преобразование координат в этой
плоскости не меняет значений упругих констант).Трансверсально-изотропный упругий материал характеризуется пятью независимыми константами. 
$$
\left(\begin{array}{cccccc}
C_{11} & C_{12} & C_{13} & 0 & 0 & 0 \\
C_{12} & C_{11} & C_{13} & 0 & 0 & 0 \\
C_{13} & C_{13} & C_{33} & 0 & 0 & 0 \\
0 & 0 & 0 & C_{44} & 0 & 0 \\
0 & 0 & 0 & 0 & C_{44} & 0 \\
0 & 0 & 0 & 0 & 0 & \frac{C_{11}-C_{12}}{2}
\end{array}\right)
$$

\textbf{Изотропный упругий материал} обладает одинаковыми свойствами по всем направлениям и характеризуется двумя независимыми константами. Эти константы называются константы (коэффициенты) Ламе : $C_{11}-C_{12}=2 \mu, \quad C_{12}=\lambda$
$$
\left(\begin{array}{c}
\sigma_1 \\
\sigma_2 \\
\sigma_3 \\
\sigma_4 \\
\sigma_5 \\
\sigma_6
\end{array}\right)=\left(\begin{array}{cccccc}
\lambda+2 \mu & \lambda & \lambda & 0 & 0 & 0 \\
\lambda & \lambda+2 \mu & \lambda & 0 & 0 & 0 \\
\lambda & \lambda & \lambda+2 \mu & 0 & 0 & 0 \\
0 & 0 & 0 & \mu & 0 & 0 \\
0 & 0 & 0 & 0 & \mu & 0 \\
0 & 0 & 0 & 0 & 0 & \mu
\end{array}\right) \cdot\left(\begin{array}{c}
\varepsilon_1 \\
\varepsilon_2 \\
\varepsilon_3 \\
2 \varepsilon_4 \\
2 \varepsilon_5 \\
2 \varepsilon_6
\end{array}\right)
$$
Квадратичная форма $W=\frac{1}{2} C_{i j k l} \varepsilon_{i j} \varepsilon_{k l}$ является положительно-определенной( критерий Сельвестра) $ \Rightarrow$ ограничения на константы Ламе:
$\left\{ \begin{array}{c}
     \mu>0\\
      3\lambda+2\mu>0\\
    \lambda+2\mu>0 \\
        \end{array}\right.$

        
Константа $\lambda$ не имеет явного физического смысла. 

Если в теле реализована деформация сдвига $\varepsilon_{12}=\gamma / 2$, то $\sigma_{12}=2 \mu \varepsilon_{12}=\mu \gamma  \Rightarrow $. Модуль сдвига : $G \equiv \mu$


(Закон Гука при чистом сдвиге: $\tau=G \gamma$. Далее, в формулах прямого закона Гука произведем свертку:
$$
\sigma_{i j}=\lambda \theta \delta_{i j}+2 \mu \varepsilon_{i j} \Rightarrow \sigma_{i i}=\lambda \theta \delta_{i i}+2 \mu \varepsilon_{i i} \Rightarrow 3 \sigma=3 \lambda \theta+2 \mu \theta \Rightarrow \sigma=\left(\lambda+\frac{2}{3} \mu\right) \theta
$$
Модуль объёмного сжатия : $K=\lambda+\frac{2}{3} \mu>0$ это коэффициент пропорциональности в выражении среднего напряжения через дилатацию.)

\textbf{Технические константы: модуль Юнга и коэффициент Пуассона}

\begin{figure}[h!]
  \centering
  \includegraphics[width=0.3\textwidth]{images/12.2.jpg}
\end{figure}



Экспериментальный закон Гука: при одноосном растяжении призматического тела относ ное удлинение пропорционально приложенной силе на единицу площади $P / S=E \cdot \Delta l / l$, $E$ --- модуль Юнга, $\sigma_{11} \approx P / S \equiv p$ --- напряжение, $\varepsilon_{11} \approx \Delta l / l$ --- деформация $\Rightarrow \sigma_{11}=E \varepsilon_{11}$. 

Экспериментальный закон Пуассона: продольное растяжении приводит к равномерному поперечному сжатию $\Delta d / d=-v \cdot \Delta l / l \Rightarrow \varepsilon_{22}=\varepsilon_{33}=-v \varepsilon_{11}.$ Коэффициент Пуассона - $v$. 
 

Рассмотрим обратный закон Гука и выразим его уравнения через технические константы.
$$
\begin{aligned}
& \begin{array}{c}
\sigma_{i j}=p \delta_{1 i} \delta_{1 j} \\
\sigma=p / 3
\end{array} \Rightarrow \varepsilon_{i j}=-\frac{3 \lambda}{2 \mu(3 \lambda+2 \mu)} \sigma \delta_{i j}+\frac{1}{2 \mu} \sigma_{i j}=\left(-\frac{\lambda}{2 \mu(3 \lambda+2 \mu)} \delta_{i j}+\frac{1}{2 \mu} \delta_{1 i} \delta_{1 j}\right) p \\
& \varepsilon_{11}=\left(-\frac{\lambda}{2 \mu(3 \lambda+2 \mu)}+\frac{1}{2 \mu}\right) p=\frac{\lambda+\mu}{\mu(3 \lambda+2 \mu)} p \Rightarrow E=\frac{\mu(3 \lambda+2 \mu)}{\lambda+\mu} \\
& \begin{array}{l}
\varepsilon_{22} =
\varepsilon_{33}
=\end{array} \frac{-\lambda}{2 \mu(3 \lambda+2 \mu)} p=\frac{-\lambda}{2 \mu(3 \lambda+2 \mu)}\left(E \varepsilon_{11}\right)=\frac{-\lambda}{2(\lambda+\mu)} \varepsilon_{11} \Rightarrow v=\frac{\lambda}{2(\lambda+\mu)} 
\end{aligned}
$$
Выразим коэффициенты Ламе через технические константы.
$$
\begin{aligned}
& \begin{array}{c}
1-2 v=\mu /(\lambda+\mu) \\
1+v=(3 \lambda+2 \mu) / 2(\lambda+\mu)
\end{array} \Rightarrow \frac{E v}{(1-2 v)(1+v)}= \\& =\frac{\mu(3 \lambda+2 \mu)}{\lambda+\mu} \cdot \frac{\lambda}{2(\lambda+\mu)} \cdot \frac{\lambda+\mu}{\mu} \cdot \frac{2(\lambda+\mu)}{3 \lambda+2 \mu}=\lambda \\
& \frac{E}{2(1+v)}=\frac{\mu(3 \lambda+2 \mu)}{\lambda+\mu} \cdot \frac{\lambda+\mu}{3 \lambda+2 \mu}=\mu, \quad \frac{E}{3(1-2 v)}=\frac{\mu(3 \lambda+2 \mu)}{\lambda+\mu} \cdot \frac{\lambda+\mu}{3 \mu}=\lambda+\frac{2}{3} \mu=K \\
& E=\frac{\mu(3 \lambda+2 \mu)}{\lambda+\mu}, \quad v=\frac{\lambda}{2(\lambda+\mu)}, \quad \lambda=\frac{E v}{(1-2 v)(1+v)}, \quad G=\mu=\frac{E}{2(1+v)}, \\ & \quad K=\frac{E}{3(1-2 v)}
\end{aligned}
$$


$\Rightarrow\text{(из критерия Сельвестра)} -1<v<0.5 \text{ и } E>0 $

\vfill


\section{Общая постановка статической краевой задачи теории упругости при малых деформациях и малых поворотах: неизменность геометрической области, занятой деформируемым телом; система пятнадцати уравнений во внутренних точках области; объяснить необходимость условий Сен-Венана; три типа краевых условий. Принцип суперпозиции линейных задач. Теорема единственности решения статической задачи. Постановка задачи теории упругости в перемещениях. Уравнение Ламе.}


В случае малых деформаций и малых вращений элементарные отрезки до и после деформирования различаются мало и по длине и по направлению: $d \vec{r} \approx d \vec{R}$. Следовательно, отсчетная конфигурация тела и его актуальная конфигурация практически совпадают, и в этом случае пропадает разница между Лагранжевыми и Эйлеровыми координатами, т.е. $x_i \approx X_i$. Пропадает разница $\partial / \partial x_i \approx \partial / \partial X_i$ между пространственными и материальными градиентами и между тензорами Грина, Альманси и Коши: $A_{i j} \approx \Gamma_{i j} \approx \varepsilon_{i j}=1 / 2\left(u_{j, i}+u_{i, j}\right)$. При этом вектор перемещений в многих случаях теряет физический смысл и его рассматривают лишь как векторное поле, определенное в трехмерном евклидовом пространстве.

\begin{figure}[h!]
  \centering
  \includegraphics[width=0.5\textwidth]{images/13.1.jpg}
\end{figure}



Упругое равновесие изотропного тела описывается системой 15-ти уравнений на 15-ть функций:


$\begin{aligned} & \left.\sigma_{i j} n_j\right|_{\vec{y} \in \Sigma_p}=P_i^0(\vec{y})  \text{ -- условия Неймана} \\ &
 \sigma_{i j, j}=-\rho F_i \text{ -- уравнения равновесия} \\ &
\sigma_{i j}=\lambda \varepsilon_{k k} \delta_{i j}+2 \mu \varepsilon_{i j} \text{ --  закон Гука} \\ & 
\left.u_i\right|_{\vec{y} \in \Sigma_u}=U_i^0(\vec{y}) \text{ -- условия Дирихле} \\ &
 \varepsilon_{i j}=\Delta_{i j k l} u_{l, k} \text{ -- уравнения Коши}  \\ &
 e_{i k l} e_{j p q} \varepsilon_{k p, l q}=0 \text{ -- уравнения Сен-Венана}\\ &
 i, j=1,2,3 
 \end{aligned}$


\textbf{Необходимость условий Сен-Венана}:
Пусть известны шесть компонент $\varepsilon_{ij}$ тензора деформаций. Требуется по этим компонентам найти три $u_{i}$
непрерывные компоненты вектора перемещений. Формулы компонент тензора Коши можно считать
системой из шести дифференциальных уравнений в частных производных относительно трех компонент
вектора перемещений. Уравнений больше чем неизвестных, поэтому решение может не существовать
(система переопределена). Требуются дополнительные условия на компоненты тензора Коши, которые
гарантируют существование вектора перемещения. Таким условием является в частности формула Чезаро:


$ \displaystyle
u_i^P=u_i^O+\omega_{j i}^O\left(x_j^P-x_j^O\right)+\int_O^P\left[\varepsilon_{i j}+\left(x_k^P-x_k\right)\left(\varepsilon_{i j, k}-\varepsilon_{j k, i}\right)\right] d x_j.
$


 Обозначим $\Psi_{i j}=\varepsilon_{i j}+\left(x_k^P-x_k\right)\left(\varepsilon_{i j, k}-\varepsilon_{j k, i}\right)$. Для того, чтобы $\vec{u}$ было векторным полем значение интеграла в формулах Чезаро не должно зависеть от выбора пути интегрирования, а зависит только от выбора точек начальной и конечной точек (O и Р), т.е: $\oint \Psi_{i j} d x_j=0 \Leftrightarrow \Psi_{i j, l}=\Psi_{i l,j}$
$$
\begin{aligned}
& \left(\varepsilon_{i j}+\left(x_k^P-x_k\right)\left(\varepsilon_{i j, k}-\varepsilon_{j k, i}\right)\right)_{, l}=\left(\varepsilon_{i l}+\left(x_k^P-x_k\right)\left(\varepsilon_{i l, k}-\varepsilon_{l k, i}\right)\right)_{, j} \\
& \varepsilon_{i j, l}-x_{k, l}\left(\varepsilon_{i j, k}-\varepsilon_{j k, i}\right)+\left(x_k^P-x_k\right)\left(\varepsilon_{i j, k l}-\varepsilon_{j k, i l}\right)=\varepsilon_{i l, j}-x_{k, j}\left(\varepsilon_{i l, k}-\varepsilon_{l k, i}\right)+ \\ & +\left(x_k^P-x_k\right)\left(\varepsilon_{i l, k j}-\varepsilon_{l k, i j}\right) \\
& \left(x_k^P-x_k\right)\left(\varepsilon_{i j, k l}-\varepsilon_{j k, i l}-\varepsilon_{i l, k j}+\varepsilon_{l k, i j}\right)=\varepsilon_{i l, j}-\delta_{k j}\left(\varepsilon_{i l, k}-\varepsilon_{l k, i}\right)-\varepsilon_{i j, l}+\delta_{k l}\left(\varepsilon_{i j, k}-\varepsilon_{j k, i}\right) \\
& \left(x_k^P-x_k\right)\left(\varepsilon_{i j, k l}-\varepsilon_{j k, i l}-\varepsilon_{i l, k j}+\varepsilon_{l k, i j}\right)=\varepsilon_{i l, j}-\varepsilon_{i l, j}+\varepsilon_{l j, i}-\varepsilon_{i j, l}+\varepsilon_{i j, l}-\varepsilon_{j l, i}=0 \\
& \text { Получаем } 81 \text { формулу: } \quad \varepsilon_{i j, k l}-\varepsilon_{j k, i l}-\varepsilon_{i l, k j}+\varepsilon_{l k, i j}=0, i, j, k, l=1,2,3.
\end{aligned}
$$


Обозначим $\chi_{i j k l} \equiv \varepsilon_{i j, k l}-\varepsilon_{j k, i l}-\varepsilon_{i l, k j}+\varepsilon_{l k, i j}$ и определим его независимые компоненты. Переставляя индексы исходя из симметрии, получаем 6 езависимых компонент: $\chi_{\underline{1} \underline{12} 2}, \chi_{\underline{1} 123}, \chi_{\underline{1} 1 \underline{3} 3}, \chi_{\underline{1} 2 \underline{2} 3}, \chi_{\underline{2} 2 \underline{3} 3}, \chi_{\underline{1} 2 \underline{3} 3}$.
Соотношения
(уравнения)
Сен-Венана:
$$
\left\{\begin{array}{l}
\varepsilon_{11,22}+\varepsilon_{22,11}=2 \varepsilon_{12,12} \\
\varepsilon_{11,33}+\varepsilon_{33,11}=2 \varepsilon_{13,13} \\
\varepsilon_{22,33}+\varepsilon_{33,22}=2 \varepsilon_{23,23}
\end{array} ;\left\{\begin{array}{l}
\left(\varepsilon_{12,3}-\varepsilon_{23,1}+\varepsilon_{31,2}\right)_{, 1}=\varepsilon_{11,23} \\
\left(\varepsilon_{12,3}+\varepsilon_{23,1}-\varepsilon_{31,2}\right)_{, 2}=\varepsilon_{22,13} \\
\left(-\varepsilon_{12,3}+\varepsilon_{23,1}+\varepsilon_{31,2}\right)_{, 3}=\varepsilon_{33,12}
\end{array}\right.\right.
$$

\textbf{3 типа краевых условий:}


\underline{Первая краевая задача (задача Дирихле, задача в перемещениях)}

\begin{figure}[h!]
  \centering
  \includegraphics[width=0.3\textwidth]{images/13.2.jpg}
\end{figure}



На границе тела заданы перемещения. Для однородного материала:
$$\displaystyle
\begin{aligned} & \left\{\begin{array}{l}(\lambda+\mu) \mathrm{grad} \mathrm{div} \overrightarrow{\mathrm{u}}+\mu \Delta \overrightarrow{\mathrm{u}}+\rho \vec{F}=\overrightarrow{0} \\ \left.u_i(\vec{y})\right|_{\vec{y} \in \partial V}=U_i^0(\vec{y}), i=1,2,3, \vec{x} \in V\end{array}\right. \Rightarrow \left\{\begin{array}{l}(\lambda+\mu) u_{j, j i}(\vec{x})+\mu u_{i, j j}(\vec{x})=-\rho F_i(\vec{x}) \\ \left.u_i(\vec{y})\right|_{\vec{y} \in \partial V}=U_i^0(\vec{y}), \quad i=1,2,3, \quad \vec{x} \in V\end{array}\right.\end{aligned}$$

 
После того, как найден вектор перемещений, по формулам Коши находим тензор деформаций:
$$
\begin{aligned}
\underset{\sim}{\varepsilon} & =\tenq{\Delta} \cdot\cdot(\vec{\nabla} \otimes \overrightarrow{\mathbf{u}}) \Rightarrow
\varepsilon_{i j} & =\frac{1}{2}\left(u_{j, i}+u_{i, j}\right)
\end{aligned}
$$


Определяем тензор напряжений с помощью закона Гука:
$$
\begin{aligned}
& \underset{\sim}{\sigma}=(\lambda \underset{\sim}{g} \otimes \underset{\sim}{g}+2 \mu \tenq{\Delta}) \cdot \underset{\sim}{\varepsilon} \Rightarrow
& \sigma_{i j}=\lambda \varepsilon_{k k} \delta_{i j}+2 \mu \varepsilon_{i j}
\end{aligned}
$$


\underline{Вторая задача (задача Нейнмана, задача в напряжениях)}

\begin{figure}[h!]
  \centering
  \includegraphics[width=0.3\textwidth]{images/13.3.jpg}
\end{figure}




На границе тела заданы напряжения, 6 функций. Для однородного материала:
$$
\begin{aligned}
& \frac{1+v}{3} \Delta \sigma_{33}+\sigma_{, 33}=-\frac{2(1+v)}{3} \rho F_{3,3}-\frac{v(1+v)}{3(1-v)} \rho F_{k, k} \\
& \frac{1+v}{3} \Delta \sigma_{23}+\sigma_{23}=-\frac{1+v}{3} \rho\left(F_{2,3}+F_{3,2}\right) \\
& \frac{1+v}{3} \Delta \sigma_{13}+\sigma_{, 13}=-\frac{1+v}{3} \rho\left(F_{1,3}+F_{3,1}\right) \\
& \frac{1+v}{3} \Delta \sigma_{22}+\sigma_{, 22}=-\frac{2(1+v)}{3} \rho F_{2,2}-\frac{v(1+v)}{3(1-v)} \rho F_{k, k} \\
& \frac{1+v}{3} \Delta \sigma_{11}+\sigma_{, 11}=-\frac{2(1+v)}{3} \rho F_{1,1}-\frac{v(1+v)}{3(1-v)} \rho F_{k, k} \\
& \frac{1+v}{3} \Delta \sigma_{12}+\sigma_{, 12}=-\frac{1+v}{3} \rho\left(F_{1,2}+F_{2,1}\right) \\
& \left.\sigma_{i j} n_j\right|_{\bar{y} \in \partial V}=P_i^0(\vec{y}), \quad \sigma_{i j, j}=-\rho F_i, \quad \vec{x} \in V 
\end{aligned}
$$
Из обратного закона Гука вычисляем тензор деформаций:
$$
\begin{aligned}
& \underset{\sim}{\varepsilon}=\left(-\frac{v}{E} \underset{\sim}{\mathbf{g}} \otimes \underset{\sim}{\mathbf{g}}+\frac{1+v}{E} \tenq{\Delta}\right) \cdot\cdot \underset{\sim}{\sigma}  \Rightarrow
& \varepsilon_{i j}=-\frac{3 v}{E} \sigma \delta_{i j}+\frac{1+v}{E} \sigma_{i j} 
\end{aligned}
$$
Находим вектор перемещения с помощью формул Чезаро:
$$
u_i^P=u_i^O+\omega_{j i}^O\left(x_j^P-x_j^O\right)+\int_0^P\left[\varepsilon_{i j}+\left(x_k^P-x_k\right)\left(\varepsilon_{i j, k}-\varepsilon_{j k, i}\right)\right] d x_j.
$$
\underline{Смешанная краевая задача}

Частично задаются напряжения и премещения (см. полуобратный метод Сен-Венана), комбинация 1 и 2 задач.


Из линейности этих уравнений описывающих граничные услови я по напряжениям, деформациям и перемещениям следует \textbf{принцип суперпозиции линейных задач} --- возможность наложения  различных решений системы уравнений при условии, что их сумма удовлетворяет граничным условиям. 
(Принцип суперпозиции позволяет выделить четыре основные задачи: о растяжении, чистом изгибе, о кручении и об изгибе силой.)


\textbf{Теорема о единственности решения краевой статической задачи}


Докажем, что однородная статическая задача в отсутствии массовых сил имеет единственное решение.


$$
\begin{aligned}
& \begin{array}{ccc}
\sigma_{i j, j}=0, \quad \sigma_{i j}=\lambda \varepsilon_{k k} \delta_{i j}+2 \mu \varepsilon_{i j} & \left.\sigma_{i j} n_j\right|_{\vec{y} \in \Sigma_p}=0 \\
\varepsilon_{i j}=\Delta_{i j k l} u_{l, k}, \quad e_{i k l} e_{j p q} \varepsilon_{k p, l q}=0 & \left.u_i\right|_{\vec{y} \in \Sigma_u}=0
\end{array} 
\end{aligned}$$

Преобразуем уравнения равновесия, свернув его с вектором перемещения. Результат проинтегрируем по всему телу: 
$ 
\displaystyle
0=\sigma_{i j, j} u_i=\left(\sigma_{i j} u_i\right)_{, j}-\sigma_{i j} u_{i, j}=\left(\sigma_{i j} u_i\right)_{, j}-\sigma_{i j} \varepsilon_{i j} \Rightarrow 0=\int_V\left(\left(\sigma_{i j} u_i\right)_{, j}-\sigma_{i j} \varepsilon_{i j}\right) d \vec{x}=\int_V\left(\sigma_{i j} u_i\right)_{, j} d \vec{x}-\int_V \sigma_{i j} \varepsilon_{i j} d \vec{x}=\int_{\partial V} \sigma_{i j} u_i n_j d \Sigma-\int_V \sigma_{i j} \varepsilon_{i j} d \vec{x}=\int_{\Sigma_p} \sigma_{i j} u_i n_j d \Sigma+ \int_{\Sigma_u} \sigma_{i j} u_i n_j d \Sigma- \int_V \sigma_{i j} \varepsilon_{i j} d \vec{x}=\int_{\Sigma_p} P_i^0 u_i d \Sigma+\int_{\Sigma_u} \sigma_{i j} n_j u_i d \Sigma-\int_V \sigma_{i j} \varepsilon_{i j} d \vec{x}=-\int_V \sigma_{i j} \varepsilon_{i j} d \vec{x} \Rightarrow  \int_V \sigma_{i j} \varepsilon_{i j} d \vec{x}=\int_V\left(K \theta^2+G \Gamma^2\right) d \vec{x}=0 \Rightarrow\left\{\begin{array}{l}
\theta=0 \Rightarrow \varepsilon_{i j}=e_{i j} \\
\Gamma=0 \Rightarrow e_{i j} \equiv 0
\end{array} \Rightarrow \varepsilon_{i j} \equiv 0\right.
$

Из формул Чезаро получаем $u_i=u_i^0+\omega_{j i}^0\left(x_j-x_j^0\right)$. 


Из закона Гука получаем $\sigma_{i j}=C_{i j k l} \varepsilon_{k l} \equiv 0$.
Вывод: решением однородной статической краевой задачи упругости является движение абсолютно твердого тела (перемещение и вращение). 


Теперь рассмотрим неоднородную задачу и предположим, что существует два различных решения: $\left(\underset{\sim}{\boldsymbol{\varepsilon}^{\prime}}, \underset{\sim}{\boldsymbol{\sigma}^{\prime}}, \vec{u}^{\prime}\right)$ и $\left(\underset{\sim}{\boldsymbol{\varepsilon^{\prime\prime}}}, \underset{\sim}{\boldsymbol{\sigma}^{\prime \prime}}, \vec{u}^{\prime \prime}\right)$. Рассмотрим разность этих двух задач:
$$
\left\{\begin{array}{l}
\sigma_{i j, j}^{\prime}=-\rho F_i \\
\sigma_{i j}^{\prime}=\lambda \theta^{\prime} \delta_{i j}+2 \mu \varepsilon_{i j}^{\prime} \\
e_{i k l} e_{j p q} \varepsilon_{k p, l q}^{\prime}=0 \\
\varepsilon_{i j}^{\prime}=\Delta_{i j k l} u_{l, k}^{\prime} \\
\left.\sigma_{i j}^{\prime} n_j\right|_{\vec{y} \in \Sigma_p}=P_i^0 \\
\left.u_i^{\prime}\right|_{\bar{y} \in \Sigma_u}=U_i^0
\end{array} \quad-\left\{\begin{array}{l}
\sigma_{i j, j}^{\prime \prime}=-\rho F_i \\
\sigma_{i j}^{\prime \prime}=\lambda \theta^{\prime \prime} \delta_{i j}+2 \mu \varepsilon_{i j}^{\prime \prime} \\
e_{i k l} e_{j p q} \varepsilon_{k p, l q}^{\prime \prime}=0 \\
\varepsilon_{i j}^{\prime \prime}=\Delta_{i j k l} u_{l, k}^{\prime \prime} \\
\left.\sigma_{i j}^{\prime \prime} n_j\right|_{\bar{y} \in \Sigma_p}=P_i^0 \\
\left.u_i^{\prime \prime}\right|_{\bar{y} \in \Sigma_u}=U_i^0
\end{array}=\right.\right.
\left\{\begin{array}{l}
\left(\sigma_{i j}^{\prime}-\sigma_{i j}^{\prime \prime}\right)_{, j}=0 \\
\left(\sigma_{i j}^{\prime}-\sigma_{i j}^{\prime \prime}\right)=\lambda\left(\theta^{\prime}-\theta^{\prime \prime}\right) \delta_{i j}+\\
+2 \mu\left(\varepsilon_{i j}^{\prime}-\varepsilon_{i j}^{\prime \prime}\right)\\
e_{i k l} e_{j p q}\left(\varepsilon_{k p}^{\prime}-\varepsilon_{k p}^{\prime \prime}\right)_{, l q}=0 \\
\left(\varepsilon_{i j}^{\prime}-\varepsilon_{i j}^{\prime \prime}\right)=\Delta_{i j k l}\left(u_l^{\prime}-u_i^{\prime}\right. \\
\left.\left(\sigma_{i j}^{\prime}-\sigma_{i j}^{\prime \prime}\right) n_j\right|_{\bar{y} \in \Sigma_p}=0 \\
\left.\left(u_i^{\prime}-u_i^{\prime \prime}\right)\right|_{\bar{y} \in \Sigma_u}=0
\end{array}\right.
$$
Два решения отличаются на движение абсолютно твёрдого тела: $\left(\underset{\sim}\varepsilon^\prime-{\underset{\sim}{\boldsymbol{\varepsilon}}}^{\prime \prime}, \underset{\sim}{\boldsymbol{\sigma}}-{\underset{\sim}{\boldsymbol{\sigma}}}^{\prime \prime}, \vec{u}^{\prime}-\vec{u}^{\prime \prime}\right)$


Для исследования свойств системы уравнений равновесия и получения аналитических решений можно исключить из системы уравнений компоненты напряжения и деформации с помощью закона Гука и кинематических соотношений, тогда получаются \textbf{уравнения в перемещениях} (см. первую краевую задачу). Деформации и напряжения определяются по ним. Можно не рассматривать перемещения, а только условия совместности, закон Гука и уравнения равновесия, тогда можно получить\textbf{уравнения в напряжениях} (см. вторую краевую задачу).


\textbf{Уравнение Ламе}


Подставим уравнения Коши в закон Гука (для изотропной однородной среды) и полученный результат подставим в уравнения движения:


$ \displaystyle
\underset{\sim}{\boldsymbol{\sigma}}=(\lambda \underset{\sim}{\mathbf{g}} \otimes \underset{\sim}{\mathbf{g}}+2 \mu \tenq{\mathbf{\Delta}}) \cdot\cdot \underset{\sim}{\boldsymbol{\varepsilon}}=\lambda \underset{\sim}{\mathbf{g}} \otimes \underset{\sim}{\mathbf{g}} \cdot\cdot \underset{\sim}{\boldsymbol{\varepsilon}}+2 \mu \underset{\sim}{\mathbf{D^S}}=\lambda \underset{\sim}{\mathbf{g}} \theta+\mu \underset{\sim}{\mathrm{D^T}}+\mu \underset{\sim}{\mathbf{D}} \Rightarrow \vec{\nabla} \cdot \underset{\sim}{\sigma}=\lambda \vec{\nabla} \otimes \theta+\mu \vec{\nabla} \cdot(\overrightarrow{\mathrm{u}} \otimes \vec{\nabla})
+\mu \vec{\nabla} \cdot(\vec{\nabla} \otimes \overrightarrow{\mathrm{u}})=\lambda \vec{\nabla} \otimes \theta+\mu \theta \otimes \vec{\nabla}+\mu \Delta \overrightarrow{\mathrm{u}}=(\lambda+\mu) \vec{\nabla} \theta+\mu \Delta \overrightarrow{\mathrm{u}}=(\lambda+\mu) \mathrm{grad} \mathrm{div} \overrightarrow{\mathrm{u}}+\mu \Delta \overrightarrow{\mathrm{u}} $
$$
(\lambda+\mu) \mathrm{grad} \mathrm{div} \overrightarrow{\mathrm{u}}+\mu \Delta \overrightarrow{\mathrm{u}}+\rho \vec{F}=\rho \frac{\partial^2 \overrightarrow{\mathrm{u}}}{\partial t^2}
$$
Уравнения Ламе в координатной форме (порядок индексов несущественен):


В декартовых координатах:
$(\lambda+\mu) \frac{\partial^2 u_\alpha}{\partial x_\alpha \partial x_i}+\mu \frac{\partial^2 u_i}{\partial x_\alpha \partial x_\alpha}+\rho F_i=\rho \frac{\partial^2 u_i}{\partial t^2}, i=1,2,3$.


Разделение на потенциальные:
$$(\lambda+2 \mu) \mathrm{grad} \mathrm{div} \overrightarrow{\mathrm{u}}-\mathrm{rot} \mathrm{rot} \overrightarrow{\mathrm{u}}+\rho \vec{F}=\rho \partial^2 \overrightarrow{\mathrm{u}} / \partial t^2$$
и вихревые слагаемые:
$$
(\lambda+2 \mu) u_{\alpha \beta}^\alpha g^{i \beta}-\mu(\mathrm{rot} \mathrm{rot} \vec{u})^i+\rho F^i=\rho \partial^2 u^i / \partial t^2, i=1,2,3 \text {. }
$$
уравнение равновесия Ламе: $(\lambda+\mu) \mathrm{grad} \mathrm{div} \overrightarrow{\mathrm{u}}+\mu \Delta \overrightarrow{\mathrm{u}}+\rho \vec{F}=\overrightarrow{0}$



\section{Общая постановка статической краевой задачи теории упругости при малых деформациях и малых поворотах: неизменность геометрической области, занятой деформируемым телом; система пятнадцати уравнений во внутренних точках области; объяснить необходимость условий Сен-Венана; три типа краевых условий. Постановка задачи теории упругости в напряжениях. Уравнения Бельтрами-Митчелла.}


В случае малых деформаций и малых вращений элементарные отрезки до и после деформирования различаются мало и по длине и по направлению: $d \vec{r} \approx d \vec{R}$. Следовательно, отсчетная конфигурация тела и его актуальная конфигурация практически совпадают, и в этом случае пропадает разница между Лагранжевыми и Эйлеровыми координатами, т.е. $x_i \approx X_i$. Пропадает разница $\partial / \partial x_i \approx \partial / \partial X_i$ между пространственными и материальными градиентами и между тензорами Грина, Альманси и Коши: $A_{i j} \approx \Gamma_{i j} \approx \varepsilon_{i j}=1 / 2\left(u_{j, i}+u_{i, j}\right)$. При этом вектор перемещений в многих случаях теряет физический смысл и его рассматривают лишь как векторное поле, определенное в трехмерном евклидовом пространстве.

\begin{figure}[h!]
  \centering
  \includegraphics[width=0.3\textwidth]{images/13.1.jpg}
\end{figure}



Упругое равновесие изотропного тела описывается системой 15-ти уравнений на 15-ть функций:


$\begin{aligned} & \left.\sigma_{i j} n_j\right|_{\vec{y} \in \Sigma_p}=P_i^0(\vec{y})  \text{ -- условия Неймана} \\ &
 \sigma_{i j, j}=-\rho F_i \text{ -- уравнения равновесия} \\ &
\sigma_{i j}=\lambda \varepsilon_{k k} \delta_{i j}+2 \mu \varepsilon_{i j} \text{ --  закон Гука} \\ & 
\left.u_i\right|_{\vec{y} \in \Sigma_u}=U_i^0(\vec{y}) \text{ -- условия Дирихле} \\ &
 \varepsilon_{i j}=\Delta_{i j k l} u_{l, k} \text{ -- уравнения Коши}  \\ &
 e_{i k l} e_{j p q} \varepsilon_{k p, l q}=0 \text{ -- уравнения Сен-Венана}\\ &
 i, j=1,2,3 
 \end{aligned}$


\textbf{Необходимость условий Сен-Венана}:
Пусть известны шесть компонент $\varepsilon_{ij}$ тензора деформаций. Требуется по этим компонентам найти три $u_{i}$
непрерывные компоненты вектора перемещений. Формулы компонент тензора Коши можно считать
системой из шести дифференциальных уравнений в частных производных относительно трех компонент
вектора перемещений. Уравнений больше чем неизвестных, поэтому решение может не существовать
(система переопределена). Требуются дополнительные условия на компоненты тензора Коши, которые
гарантируют существование вектора перемещения. Таким условием является в частности формула Чезаро:


$ \displaystyle
u_i^P=u_i^O+\omega_{j i}^O\left(x_j^P-x_j^O\right)+\int_O^P\left[\varepsilon_{i j}+\left(x_k^P-x_k\right)\left(\varepsilon_{i j, k}-\varepsilon_{j k, i}\right)\right] d x_j.
$


 Обозначим $\Psi_{i j}=\varepsilon_{i j}+\left(x_k^P-x_k\right)\left(\varepsilon_{i j, k}-\varepsilon_{j k, i}\right)$. Для того, чтобы $\vec{u}$ было векторным полем значение интеграла в формулах Чезаро не должно зависеть от выбора пути интегрирования, а зависит только от выбора точек начальной и конечной точек (O и Р), т.е: $\oint \Psi_{i j} d x_j=0 \Leftrightarrow \Psi_{i j, l}=\Psi_{i l,j}$
$$
\begin{aligned}
& \left(\varepsilon_{i j}+\left(x_k^P-x_k\right)\left(\varepsilon_{i j, k}-\varepsilon_{j k, i}\right)\right)_{, l}=\left(\varepsilon_{i l}+\left(x_k^P-x_k\right)\left(\varepsilon_{i l, k}-\varepsilon_{l k, i}\right)\right)_{, j} \\
& \varepsilon_{i j, l}-x_{k, l}\left(\varepsilon_{i j, k}-\varepsilon_{j k, i}\right)+\left(x_k^P-x_k\right)\left(\varepsilon_{i j, k l}-\varepsilon_{j k, i l}\right)=\varepsilon_{i l, j}-x_{k, j}\left(\varepsilon_{i l, k}-\varepsilon_{l k, i}\right)+\\ &+\left(x_k^P-x_k\right)\left(\varepsilon_{i l, k j}-\varepsilon_{l k, i j}\right) \\
& \left(x_k^P-x_k\right)\left(\varepsilon_{i j, k l}-\varepsilon_{j k, i l}-\varepsilon_{i l, k j}+\varepsilon_{l k, i j}\right)=\varepsilon_{i l, j}-\delta_{k j}\left(\varepsilon_{i l, k}-\varepsilon_{l k, i}\right)-\varepsilon_{i j, l}+\delta_{k l}\left(\varepsilon_{i j, k}-\varepsilon_{j k, i}\right) \\
& \left(x_k^P-x_k\right)\left(\varepsilon_{i j, k l}-\varepsilon_{j k, i l}-\varepsilon_{i l, k j}+\varepsilon_{l k, i j}\right)=\varepsilon_{i l, j}-\varepsilon_{i l, j}+\varepsilon_{l j, i}-\varepsilon_{i j, l}+\varepsilon_{i j, l}-\varepsilon_{j l, i}=0 \\
& \text { Получаем } 81 \text { формулу: } \quad \varepsilon_{i j, k l}-\varepsilon_{j k, i l}-\varepsilon_{i l, k j}+\varepsilon_{l k, i j}=0, i, j, k, l=1,2,3.
\end{aligned}
$$


Обозначим $\chi_{i j k l} \equiv \varepsilon_{i j, k l}-\varepsilon_{j k, i l}-\varepsilon_{i l, k j}+\varepsilon_{l k, i j}$ и определим его независимые компоненты. Переставляя индексы исходя из симметрии, получаем 6 езависимых компонент: $\chi_{\underline{1} \underline{12} 2}, \chi_{\underline{1} 123}, \chi_{\underline{1} 1 \underline{3} 3}, \chi_{\underline{1} 2 \underline{2} 3}, \chi_{\underline{2} 2 \underline{3} 3}, \chi_{\underline{1} 2 \underline{3} 3}$.
Соотношения
(уравнения)
Сен-Венана:
$$
\left\{\begin{array}{l}
\varepsilon_{11,22}+\varepsilon_{22,11}=2 \varepsilon_{12,12} \\
\varepsilon_{11,33}+\varepsilon_{33,11}=2 \varepsilon_{13,13} \\
\varepsilon_{22,33}+\varepsilon_{33,22}=2 \varepsilon_{23,23}
\end{array} ;\left\{\begin{array}{l}
\left(\varepsilon_{12,3}-\varepsilon_{23,1}+\varepsilon_{31,2}\right)_{, 1}=\varepsilon_{11,23} \\
\left(\varepsilon_{12,3}+\varepsilon_{23,1}-\varepsilon_{31,2}\right)_{, 2}=\varepsilon_{22,13} \\
\left(-\varepsilon_{12,3}+\varepsilon_{23,1}+\varepsilon_{31,2}\right)_{, 3}=\varepsilon_{33,12}
\end{array}\right.\right.
$$

\textbf{3 типа краевых условий:}


\underline{Первая краевая задача (задача Дирихле, задача в перемещениях)}
\begin{figure}[h!]
  \centering
  \includegraphics[width=0.3\textwidth]{images/13.2.jpg}
\end{figure}

На границе тела заданы перемещения. Для однородного материала:
$$\displaystyle
\begin{aligned} & \left\{\begin{array}{l}(\lambda+\mu) \mathrm{grad} \mathrm{div} \overrightarrow{\mathrm{u}}+\mu \Delta \overrightarrow{\mathrm{u}}+\rho \vec{F}=\overrightarrow{0} \\ \left.u_i(\vec{y})\right|_{\vec{y} \in \partial V}=U_i^0(\vec{y}), i=1,2,3, \vec{x} \in V\end{array}\right. \Rightarrow \left\{\begin{array}{l}(\lambda+\mu) u_{j, j i}(\vec{x})+\mu u_{i, j j}(\vec{x})=-\rho F_i(\vec{x}) \\ \left.u_i(\vec{y})\right|_{\vec{y} \in \partial V}=U_i^0(\vec{y}), \quad i=1,2,3, \quad \vec{x} \in V\end{array}\right.\end{aligned}$$

 
После того, как найден вектор перемещений, по формулам Коши находим тензор деформаций:
$$
\begin{aligned}
\underset{\sim}{\varepsilon} & =\tenq{\Delta} \cdot\cdot(\vec{\nabla} \otimes \overrightarrow{\mathbf{u}}) \Rightarrow
\varepsilon_{i j} & =\frac{1}{2}\left(u_{j, i}+u_{i, j}\right)
\end{aligned}
$$


Определяем тензор напряжений с помощью закона Гука:
$$
\begin{aligned}
& \underset{\sim}{\sigma}=(\lambda \underset{\sim}{g} \otimes \underset{\sim}{g}+2 \mu \tenq{\Delta}) \cdot \underset{\sim}{\varepsilon} \Rightarrow
& \sigma_{i j}=\lambda \varepsilon_{k k} \delta_{i j}+2 \mu \varepsilon_{i j}
\end{aligned}
$$


\underline{Вторая задача (задача Нейнмана, задача в напряжениях)}

\begin{figure}[h!]
  \centering
  \includegraphics[width=0.3\textwidth]{images/13.3.jpg}
\end{figure}



На границе тела заданы напряжения, 6 функций. Для однородного материала:
$$
\begin{aligned}
& \frac{1+v}{3} \Delta \sigma_{33}+\sigma_{, 33}=-\frac{2(1+v)}{3} \rho F_{3,3}-\frac{v(1+v)}{3(1-v)} \rho F_{k, k} \\
& \frac{1+v}{3} \Delta \sigma_{23}+\sigma_{23}=-\frac{1+v}{3} \rho\left(F_{2,3}+F_{3,2}\right) \\
& \frac{1+v}{3} \Delta \sigma_{13}+\sigma_{, 13}=-\frac{1+v}{3} \rho\left(F_{1,3}+F_{3,1}\right) \\
& \frac{1+v}{3} \Delta \sigma_{22}+\sigma_{, 22}=-\frac{2(1+v)}{3} \rho F_{2,2}-\frac{v(1+v)}{3(1-v)} \rho F_{k, k} \\
& \frac{1+v}{3} \Delta \sigma_{11}+\sigma_{, 11}=-\frac{2(1+v)}{3} \rho F_{1,1}-\frac{v(1+v)}{3(1-v)} \rho F_{k, k} \\
& \frac{1+v}{3} \Delta \sigma_{12}+\sigma_{, 12}=-\frac{1+v}{3} \rho\left(F_{1,2}+F_{2,1}\right) \\
& \left.\sigma_{i j} n_j\right|_{\bar{y} \in \partial V}=P_i^0(\vec{y}), \quad \sigma_{i j, j}=-\rho F_i, \quad \vec{x} \in V 
\end{aligned}
$$
Из обратного закона Гука вычисляем тензор деформаций:
$$
\begin{aligned}
& \underset{\sim}{\varepsilon}=\left(-\frac{v}{E} \underset{\sim}{\mathbf{g}} \otimes \underset{\sim}{\mathbf{g}}+\frac{1+v}{E} \tenq{\Delta}\right) \cdot\cdot \underset{\sim}{\sigma}  \Rightarrow
& \varepsilon_{i j}=-\frac{3 v}{E} \sigma \delta_{i j}+\frac{1+v}{E} \sigma_{i j} 
\end{aligned}
$$
Находим вектор перемещения с помощью формул Чезаро:
$$
u_i^P=u_i^O+\omega_{j i}^O\left(x_j^P-x_j^O\right)+\int_0^P\left[\varepsilon_{i j}+\left(x_k^P-x_k\right)\left(\varepsilon_{i j, k}-\varepsilon_{j k, i}\right)\right] d x_j.
$$
\underline{Смешанная краевая задача}

Частично задаются напряжения и премещения (см. полуобратный метод Сен-Венана), комбинация 1 и 2 задач.


Для исследования свойств системы уравнений равновесия и получения аналитических решений можно исключить из системы уравнений компоненты напряжения и деформации с помощью закона Гука и кинематических соотношений, тогда получаются \textbf{уравнения в перемещениях} (см. первую краевую задачу). Деформации и напряжения определяются по ним. Можно не рассматривать перемещения, а только условия совместности, закон Гука и уравнения равновесия, тогда можно получить\textbf{уравнения в напряжениях} (см. вторую краевую задачу).


\textbf{Уравнения Бельтрама--Митчела}


Уравнения Бельтрами-Митчелла применяют для решения статических задач однородного, изотропного линейно-упругого тела. Получим их подстановкой закона Гука в уравнения совместности и равновесия.
$$
\begin{aligned}
&\left\{\begin{array}{l}
\varepsilon_{11,22}+\varepsilon_{22,11}=2 \varepsilon_{12,12} \\
\varepsilon_{11,33}+\varepsilon_{33,11}=2 \varepsilon_{13,13} \\
\varepsilon_{22,33}+\varepsilon_{33,22}=2 \varepsilon_{23,23}
\end{array} ;\left\{\begin{array}{l}
\left(\varepsilon_{12,3}+\varepsilon_{23,1}+\varepsilon_{31,2}\right)_{, 1}=\varepsilon_{11,23} \\\left(\varepsilon_{12,3}+\varepsilon_{23,1}-\varepsilon_{31,2}\right)_{, 2}=\varepsilon_{22,13} \\
\left(-\varepsilon_{12,3}+\varepsilon_{23,1}+\varepsilon_{31,2}\right)_{, 3}=\varepsilon_{33,12}
\end{array}\right.\right. \\
& \sigma_{i j, j}+\rho F_i=0,\quad i=1,2,3.\quad \varepsilon_{i j}=-\frac{3 v}{E} \sigma \delta_{i j}+\frac{1+v}{E} \sigma_{i j} 
\end{aligned}
$$
Подставим обратный закон Гука в уравнения Сен-Венана. Получим щесть уравнений Сен-Венана, выраженные через компоненты тензора напряжений:
$$
\left\{\begin{array}{l}
\sigma_{11,22}+\sigma_{22,11}-\frac{3 v}{1+v}\left(\sigma_{, 11}+\sigma_{, 22}\right)=2 \sigma_{12,12} \\
\sigma_{11,33}+\sigma_{33,11}-\frac{3 v}{1+v}\left(\sigma_{, 11}+\sigma_{, 33}\right)=2 \sigma_{13,13} \\
\sigma_{22,33}+\sigma_{33,22}-\frac{3 v}{1+v}\left(\sigma_{, 22}+\sigma_{, 33}\right)=2 \sigma_{23,23}
\end{array} ; \begin{cases}\sigma_{11,23}-\frac{3 v}{1+v} \sigma_{, 23}=\left(\sigma_{12,3}-\sigma_{23,1}+\sigma_{31,2}\right)_{, 1} & \\
\sigma_{22,13}-\frac{3 v}{1+v} \sigma_{, 13}=\left(\sigma_{12,3}+\sigma_{23,1}-\sigma_{31,2}\right)_{, 2} &  \\
\sigma_{33,12}-\frac{3 v}{1+v} \sigma_{, 12}=\left(-\sigma_{12,3}+\sigma_{23,1}+\sigma_{31,2}\right)_{, 3} & \end{cases}\right.
$$


Продифференцируем первое уравнение равновесия по $\boldsymbol{x}_1$, второе по $\boldsymbol{x}_2$, третье уравнение по $\boldsymbol{x}_3$. Теперь сложим первых два уравнения и вычтем третье. Повторим операцию ещё 2 раза: сложим первое и третье уравнения и вычтем второе, а затем сложим второе и третье уравнения и вычтем первое:
$$
\begin{aligned}
& \sigma_{11,1}+\sigma_{12,2}+\sigma_{13,3}+\rho F_1=0 \quad \quad \sigma_{11,11}+\sigma_{12,21}+\sigma_{13,31}+\rho F_{1,1}=0 \quad \\
& \sigma_{21,1}+\sigma_{22,2}+\sigma_{23,3}+\rho F_2=0 \Rightarrow \sigma_{21,12}+\sigma_{22,22}+\sigma_{23,32}+\rho F_{2,2}=0 \Rightarrow \\
& \sigma_{31,1}+\sigma_{32,2}+\sigma_{33,3}+\rho F_3=0 \quad \quad \sigma_{31,13}+\sigma_{32,23}+\sigma_{33,33}+\rho F_{3,3}=0 \quad 
\end{aligned}
$$
$$
\begin{aligned}
&\underline{\sigma_{11,11}}+\underline{\underline{\sigma_{22,22}}}-\sigma_{33,33}=\rho\left(-F_{1,1}-F_{2,2}+F_{3,3}\right)-2 \sigma_{12,12} \\
&\sigma_{11,11}-\sigma_{22,22}+\sigma_{33,33}=\rho\left(-F_{1,1}+F_{2,2}-F_{3,3}\right)-2 \sigma_{13,13} \\
& -\sigma_{11,11}+\sigma_{22,22}+\sigma_{33,33}=\rho\left(F_{1,1}-F_{2,2}-F_{3,3}\right)-2 \sigma_{23,23} \\
&
\end{aligned}
$$
Сложим первое равенство и первое уравнение Сен-Венана 


$\underline{\sigma_{11,22}}+\underline{\sigma_{22,11}}-(3 v / 1+v)\left(\sigma_{, 11}+\sigma_{, 22}\right)=2 \sigma_{12,12}$ :
$$
\begin{aligned}
& \left(\sigma_{11}+\sigma_{22}\right)_{, 11}+\left(\sigma_{11}+\sigma_{22}\right)_{, 22}-\sigma_{33,33}-\frac{3 v}{1+v}\left(\sigma_{, 11}+\sigma_{, 22}\right)=\rho\left(-F_{1,1}-F_{2,2}+F_{3,3}\right) \\
& \left(3 \sigma-\underline{\sigma_{33}}\right)_{, 11}+\left(3 \sigma-\underline{\sigma_{33}}\right)_{, 22}-\underline{\sigma_{33,33}}-\frac{3 v}{1+v}\left(\sigma_{, 11}+\sigma_{, 22}\right)=\rho\left(-F_{1,1}-F_{2,2}+F_{3,3}\right) \\
& 3\left(\sigma_{, 11}+\sigma_{, 22}\right)-\left(\sigma_{33,11}+\sigma_{33,22}+\sigma_{33,33}\right)-\frac{3 v}{1+v}\left(\sigma_{, 11}+\sigma_{, 22}\right)=\rho\left(-F_{1,1}-F_{2,2}+F_{3,3}\right) \\
& \frac{3}{1+v}\left(\Delta \sigma-\sigma_{, 33}\right)-\Delta \sigma_{33}=\rho\left(F_{3,3}-F_{2,2}-F_{1,1}\right) \Rightarrow \\
&\Delta \sigma-\frac{1+v}{3} \Delta \sigma_{33}-\sigma_{, 33}=\frac{1+v}{3} \rho\left(F_{3,3}-F_{2,2}-F_{1,1}\right)
\end{aligned}
$$
Аналогично ещё два раза: сложим второе измененное уравнение равновесия и второе уравнение СенВенана, а затем сложим третье измененное уравнение равновесия и третье уравнение Сен-Венана:
$\displaystyle
\begin{aligned}
\left\{\begin{array}{l}
\Delta \sigma-\frac{1+v}{3} \Delta \sigma_{33}-\sigma_{,33}=\frac{1+v}{3} \rho\left(F_{3,3}-F_{2,2}-F_{1,1}\right) \\
\Delta \sigma-\frac{1+v}{3} \Delta \sigma_{22}-\sigma_{, 22}=\frac{1+v}{3} \rho\left(-F_{3,3}+F_{2,2}-F_{1,1}\right) \\
\Delta \sigma-\frac{1+v}{3} \Delta \sigma_{11}-\sigma_{,11}=\frac{1+v}{3} \rho\left(-F_{3,3}-F_{2,2}+F_{1,1}\right)
\end{array}\right.
\end{aligned}
$
$\displaystyle
3\Delta \sigma-(1+v) \Delta \sigma-\Delta \sigma=-\frac{1+v}{3} \rho F_{k, k} \Rightarrow
(1-v) \Delta \sigma=-\frac{1+v}{3} \rho F_{k, k} \Rightarrow
\Delta \sigma=-\frac{1+v}{3(1-v)} \rho F_{k, k}
$

Далее, заменим в первом уравнении $\Delta \sigma$ с помощью уравнения Пуассона и умножим это уравнение на -1:
$$
\begin{aligned}
& \frac{1+v}{3} \Delta \sigma_{33}+\sigma_{,33}=-\frac{1+v}{3} \rho\left(F_{3,3}-F_{2,2}-F_{1,1}\right)-\frac{1+v}{3(1-v)} \rho F_{k, k}=\\
&-\frac{2(1+v)}{3} \rho F_{3,3}+\frac{1+v}{3} \rho\left(F_{3,3}+F_{2,2}+F_{1,1}\right)-\frac{1+v}{3(1-v)} \rho F_{k, k}= \\
& =-\frac{2(1+v)}{3} \rho F_{3,3}+\left(\frac{1+v}{3}-\frac{1+v}{3(1-v)}\right) \rho F_{k, k}=-\frac{2(1+v)}{3} \rho F_{3,3}-\frac{v(1+v)}{3(1-v)} \rho F_{k, k}
\end{aligned} $$
Аналогично для второго и третьего уравнения Сен-Венана. Получим первые три уравнения Бельтрами-Митчелла:
$$
\begin{aligned}
& \frac{1+v}{3} \Delta \sigma_{33}+\sigma_{,33}=-\frac{2(1+v)}{3} \rho F_{3,3}-\frac{v(1+v)}{3(1-v)} \rho F_{k, k} \\
& \frac{1+v}{3} \Delta \sigma_{22}+\sigma_{, 22}=-\frac{2(1+v)}{3} \rho F_{2,2}-\frac{v(1+v)}{3(1-v)} \rho F_{k, k}\\
& \frac{1+v}{3} \Delta \sigma_{11}+\sigma_{, 11}=-\frac{2(1+v)}{3} \rho F_{1,1}-\frac{v(1+v)}{3(1-v)} \rho F_{k, k}
\end{aligned}
$$

Продифференцируем второе уравнение равновесия по $x_3$, третье уравнениие равновесия по $x_2$ и сложим их с четвертым уравнением Сен-Венана:
$$
\begin{aligned}
\left\{\begin{array}{l}
 \sigma_{2 1, 13}+\underline{\sigma_{22,23}}+\underline{\underline{\sigma_{23,33}}}=-\rho F_{2,3} \\
\sigma_{31,12}+\underline{\underline{\sigma_{32,22}}}+\underline{\sigma_{33,32}}=-\rho F_{3,2} \\
\sigma_{11,23}-\frac{3 v}{1+v} \sigma_{, 23}=\sigma_{12,31}-\sigma_{23,11}+\sigma_{31,12}
\end{array}\right.
\end{aligned}
$$
$$
\begin{aligned}
& \left(\sigma_{11}+\sigma_{22}+\sigma_{33}\right)_{, 23}+\left(\sigma_{23,11}+\sigma_{32,22}+\sigma_{23,33}\right)-\frac{3 v}{1+v} \sigma_{, 23}=-\rho\left(F_{2,3}+F_{3,2}\right) \\
& \Delta \sigma_{23}+\left(3-\frac{3 v}{1+v}\right) \sigma_{23}=-\rho\left(F_{2,3}+F_{3,2}\right) \quad \\
&\frac{1+v}{3} \Delta \sigma_{23}+\sigma_{23}=-\frac{1+v}{3} \rho\left(F_{2,3}+F_{3,2}\right)
\end{aligned}
$$
Аналогично: 1-ое уравнение равновесия дифференцируем по $\boldsymbol{x}_{\mathbf{3}}$, 3-ее уравнение - по $\boldsymbol{x}_{\mathbf{1}}$ и сложим их с 5-м уравнением Сен-Венана. Далее: 1ое уравнение равновесия дифференцируем по $\boldsymbol{x}_2,\quad 2$-ое уравнение - по $\boldsymbol{x}_{\mathbf{1}}$ и сложим их с 6-м уравнением Сен-Венана.


Получаем ещё три уравнения Бельтрами-Митчелла:

$$
\begin{aligned}
& \frac{1+v}{3} \Delta \sigma_{23}+\sigma_{, 23}=-\frac{1+v}{3} \rho\left(F_{2,3}+F_{3,2}\right) \\
& \frac{1+v}{3} \Delta \sigma_{13}+\sigma_{, 13}=-\frac{1+v}{3} \rho\left(F_{1,3}+F_{3,1}\right) \\
& \frac{1+v}{3} \Delta \sigma_{12}+\sigma_{, 12}=-\frac{1+v}{3} \rho\left(F_{1,2}+F_{2,1}\right) \\
& \frac{1+v}{3} \Delta \sigma_{33}+\sigma_{33}=-\frac{2(1+v)}{3} \rho F_{3,3}-\frac{v(1+v)}{3(1-v)} \rho F_{k, k} \\
& \frac{1+v}{3} \Delta \sigma_{22}+\sigma_{, 22}=-\frac{2(1+v)}{3} \rho F_{2,2}-\frac{v(1+v)}{3(1-v)} \rho F_{k, k} \\
& \frac{1+v}{3} \Delta \sigma_{11}+\sigma_{11}=-\frac{2(1+v)}{3} \rho F_{1,1}-\frac{v(1+v)}{3(1-v)} \rho F_{k, k}
\end{aligned}
$$


 Система уравнений Бельтрами-Митчелла пригодна только для случая линейно-упругого изотропного однородного тела при изотермическом или адиабатическом прочессе статического деформирования, тогда как шесть уравнений совместности Сен-Венана пригодны для любого тела при любых нагрузках.


\section{Ослабление граничных условий: принцип Сен-Венана. Определение простейших задачах теории упругости. Формула Чезаро для простейших задач. Полуобратный метод Сен-Венана: решение задачи о растяжении призматического бруса под действием собственного веса.}

\begin{figure}[h!]
  \centering
  \includegraphics[width=0.4\textwidth]{images/14.1.jpg}
\end{figure}

 

Точный учёт граничных условий (усилий и перемещений) делает краевую
задачу очень сложной. Принцип Сен-Венана гласит, что \textbf{уравновешенная
система внешних сил, приложенная к упругому телу, когда все точки
приложения сил этой системы лежат внутри некоторой сферы, производит
пренебрежимо малые деформации в точках тела, на расстояниях от
сферы, достаточно больших по сравнению с её радиусом.} Таким образом,
при решении задач заданная система сил, приложенная к небольшой части
упругого тела, заменяется другой, удобной для упрощения задачи,
статически эквивалентной системой сил (т.е. имеющей те же главный
вектор и главный момент), приложенной к той же части тела.


\textbf{Простейшие задачи теории упругости} – это краевые задачи равновесия (также называются задачи статики или статические задачи) однородных изотропных тел, в которых в любой точке тела
компоненты напряжений (а значит и деформаций) постоянны или линейно зависят от координат.
Уравнения совместности выполнены тождественно, потому что вторые частные производные от констант
и линейных функций равны нулю. Уравнения Бельтрами-Митчелла выполняются тождественно (что не гарантирует выполнения уравнений равновесия), если
массовые силы являются константами, т. е. не зависят от координат точек тела. В ином случае эти
уравнения являются условиями, которым должны удовлетворять массовые силы, чтобы в теле было
реализовано простейшее напряженно-деформированное состояние.


\textbf{Полуобратный метод Сен-Венана} – метод решения задач, при котором частично задаются напряжения и
перемещения, а затем при помощи уравнений теории упругости определяются условия, которым должны
удовлетворять неизвестные компоненты напряжений и перемещений. На последнем этапе решения
можно воспользоваться формулами Чезаро:
$$
u_i^A=u_i^O+\omega_{j i}^O\left(x_j^A-x_j^O\right)+\int_O^A\left[\varepsilon_{i j}+\left(x_k^A-x_k\right)\left(\varepsilon_{i j, k}-\varepsilon_{j k, i}\right)\right] d x_j.
$$
Упростим формулы Чезаро для случая, когда все компоненты $\varepsilon_{i j, k}$ являются константами: 
Примем за $O$ начало координат и исключим движение абсолютно твердого тела: $x_i^O=u_i^O=\omega_{i j}^O=0$.
Криволинейный интеграл не зависит от пути интегрирования. Предполагая тело выпуклым, соединим точки $O$ и $A$ прямолинейным отрезком и параметризуем переменной $t \in[0 ; 1]$.

\begin{figure}[h!]
  \centering
  \includegraphics[width=0.2\textwidth]{images/14.2.jpg}  
\end{figure}

Тогда 
$$
\begin{aligned}
& \begin{array}{l}
x_k=x_k^A t \\
x_j=x_j^A t
\end{array}: \int_0^A\left(x_k^A-x_k\right)\left(\varepsilon_{i j, k}-\varepsilon_{j k, i}\right) d x_j=\left(\varepsilon_{i j, k}-\varepsilon_{j k, i}\right) \int_0^1\left(x_k^A-x_k^A t\right) x_j^A d t= \\
& =\left(\varepsilon_{i j, k}-\varepsilon_{j k, i}\right) x_k^A x_j^A \int_0^1(1-t) d t=\frac{\left(\varepsilon_{i j, k}-\varepsilon_{j k, i}\right)}{2} x_k^A x_j^A \\
& u_i=x_j \int_0^1 \varepsilon_{i j}^1\left(x_1 t, x_2 t, x_3 t\right) d t+\frac{x_k x_j}{2}\left(\varepsilon_{i j, k}-\varepsilon_{j k, i}\right)
\end{aligned}
$$

Если все компоненты постоянны:$\displaystyle u_i=x_j \int_0^1 \varepsilon_{i j} d t =  x_j \varepsilon_{i j} \int_0^1 d t=\varepsilon_{i j} x_j$


$\displaystyle
u_i=\varepsilon_{i j} x_j, \vec{u}=\underset{\sim}{\varepsilon} \cdot \vec{r}=\vec{r} \cdot \underset{\sim}{\varepsilon} \Rightarrow
\vec{u}=\vec{u}^O-{\underset{\sim}{\omega}}^O \cdot \vec{r}+\underset{\sim}{\varepsilon} \cdot \vec{r}=\vec{u}^O+\left(\underset{\sim}{\varepsilon}-{\underset{\sim}{\omega}}^O\right) \cdot \vec{r}
$


\textbf{Задача о растяжении призматическиого бруса под действием собственного веса.}

\begin{figure}[h!]
  \centering
  \includegraphics[width=0.15\textwidth]{images/14.3.jpg}    
\end{figure}


Призматический брус закреплен вертикально в поле силы тяжести. Начало координат выбрали в центре масс закрепленного торца. Третья координатная ось вдоль оси бруса. $\vec{F}=g \vec{\kappa}_3$. Предположим, что: $\sigma_{i j}=\left(a x_3+b\right) \delta_{i 3} \delta_{j 3}$. Уравнения Бельтрами-Митчелла выполнены. Также выполнены первых два уравнения равновесия. Подставим в третье: $\sigma_{3 j, j}=-\rho g \Rightarrow a=-\rho g \Rightarrow \sigma_{i j}=\left(-\rho g x_3+b\right) \delta_{i 3} \delta_{j 3}$. Проверим свободу от нагрузки на боковой поверхности бруса, где $\vec{n}=\left(n_1, n_2, 0\right):\left.P_i^{(\vec{n})}\right|_{\Sigma_{\text {bos }}}=\sigma_{i j} n_j=\left(a x_3+b\right) \delta_{i 3} \delta_{j 3} n_j=0$. На нижнем торце бруса имеем $\vec{n}=(0,0,1):\left.P_i^{(\bar{n})}\right|_{\Sigma_{\max }}(a l+b) \delta_{i 3} \delta_{j 3} n_j=(a l+b) \delta_{i 3}=0$.
Получаем: $b=-a l \Rightarrow \sigma_{i j}=\rho g\left(l-x_3\right) \delta_{i 3} \delta_{j 3}$. Определим усилия на верхнем торце $\vec{n}=(0,0,-1)$ : $\left.P_i^{(\bar{n})}\right|_{\Sigma_{\mathrm{segs}}}=-\rho g l \delta_{i 3} \Rightarrow P S=\rho g l S=M g$. Такое распределение соответствует идеальному закреплению, когда существуют только нормальные равномерно распределенные напряжения.
$$
\left[\sigma_{i j}\right]=\rho g\left(l-x_3\right)\left(\begin{array}{lll}
0 & 0 & 0 \\
0 & 0 & 0 \\
0 & 0 & 1
\end{array}\right)
$$
Принцип Сен-Венана позволяет применять найденные формулы вдали от зоны закрепления и при неидеальном закреплении. Далее, применяем обобщенный закон Гука и находим компоненты тензора деформаций Коши:
$ \sigma_{i j}= \rho g\left(l-x_3\right) \delta_{i 3} \delta_{j 3} ;\quad \sigma=\frac{\sigma_{k k}}{3}=\frac{\rho g}{3}\left(l-x_3\right) \Rightarrow \varepsilon_{i j}=-\frac{3 v}{E} \sigma \delta_{i j}+\frac{1+v}{E} \sigma_{i j}=\left(-\frac{v}{E} \delta_{i j}+\frac{1+v}{E} \delta_{i 3} \delta_{j 3}\right) \rho g\left(l-x_3\right) \varepsilon_{11}= \varepsilon_{22}=-\frac{v}{E} \rho g\left(l-x_3\right) ; \varepsilon_{33}=\frac{\rho g}{E}\left(l-x_3\right) ; \varepsilon_{12}=\varepsilon_{13}=\varepsilon_{23}=0.$


Компоненты деформаций зависят только от $x_3$, а матрица - диагональная.
$ \displaystyle
u_1=x_j \int_0^1 \varepsilon_{1 j}\left(x_3 t\right) d t+\frac{x_k x_j}{2}\left(\varepsilon_{1 j, k}-\varepsilon_{j k, 1}\right)=x_1 \int_0^1 \varepsilon_{11}\left(x_3 t\right) d t+\frac{x_3 x_1}{2} \varepsilon_{11,3}=\frac{v x_1}{E x_3} \rho g \int_0^1\left(l-x_3 t\right) d\left(l-x_3 t\right)+\frac{x_3 x_1}{2} \frac{v}{E} \rho g=\left.\frac{v x_1}{E} \frac{\rho g}{2 x_3}\left(l-x_3 t\right)^2\right|_0 ^1+\frac{x_3 x_1}{2} \frac{v}{E} \rho g=\frac{v \rho g}{2 E} x_1\left(\frac{\left(l-x_3\right)^2-l^2}{x_3}+x_3\right)=-\frac{v \rho g}{E} x_1\left(l-x_3\right) u_2= x_j \int_0^1 \varepsilon_{2 j} d t+\frac{x_k x_j}{2}\left(\varepsilon_{2 j, k}-\varepsilon_{j k, 2}\right)=x_2 \int_0^1 \varepsilon_{22}\left(x_3 t\right) d t+\frac{x_3 x_2}{2} \varepsilon_{22,3}=\frac{v x_2}{E x_3} \rho g \int_0^1\left(l-x_3 t\right) d\left(l-x_3 t\right)+\frac{x_3 x_2}{2} \frac{v}{E} \rho g=\left.\frac{v x_2}{E x_3} \rho g \frac{\left(l-x_3 t\right)^2}{2}\right|_0 ^1+\frac{x_3 x_2}{2} \frac{v}{E} \rho g=\frac{v \rho g}{2 E} x_2\left(\frac{\left(l-x_3\right)^2-l^2}{x_3}+x_3\right)=-\frac{v \rho g}{E} x_2\left(l-x_3\right)$

$ \displaystyle
u_3=x_j \int_0^1 \varepsilon_{3 j} d t+\frac{x_k x_j}{2}\left(\varepsilon_{3 j, k}-\varepsilon_{j k, 3}\right)=x_3 \int_0^1 \varepsilon_{33} d t+\frac{x_k x_j}{2} \varepsilon_{3 j, k}-\frac{x_k x_j}{2} \varepsilon_{j k, 3}=x_3 \int_0^1 \varepsilon_{33}\left(x_3 t\right) d t+\frac{x_3 x_3}{2} \varepsilon_{33,3}
-\frac{x_3 x_3}{2} \varepsilon_{33,3}-\frac{x_2 x_2}{2} \varepsilon_{22,3}-\frac{x_1 x_1}{2} \varepsilon_{11,3}=-\frac{\rho g}{E} \int_0^1\left(l-x_3 t\right) d\left(-x_3 t\right)-\frac{v \rho g}{2 E}\left[\left(x_2\right)^2+\left(x_1\right)^2\right]
=-\left.\frac{\rho g}{E} \frac{\left(l-x_3 t\right)^2}{2}\right|_0 ^1-\frac{v \rho g}{2 E}\left[\left(x_2\right)^2+\left(x_1\right)^2\right]=-\frac{\rho g}{2 E}\left(\left(x_3\right)^2+v\left[\left(x_2\right)^2+\left(x_1\right)^2\right]-2 l x_3\right)$


$$ \text{Получаем:}\quad u_1=-\frac{v}{E} \rho g\left(l-x_3\right) x_1 ; \quad u_2=-\frac{v}{E} \rho g\left(l-x_3\right) x_2;  $$
$$ u_3=-\frac{\rho g}{2 E}\left(\left(x_3\right)^2+v\left(\left(x_1\right)^2+\left(x_2\right)^2\right)-2 l x_3\right)$$


\vfill

\section{Ослабление граничных условий: принцип Сен-Венана. Определение простейших задачах теории упругости. Формула Чезаро для простейших задач. Полуобратный метод Сен-Венана: решение задачи о кручении призматического бруса круглого сечения.}

\begin{figure}[h!]
  \centering
  \includegraphics[width=0.4\textwidth]{images/14.1.jpg}   
\end{figure}


Точный учёт граничных условий (усилий и перемещений) делает краевую
задачу очень сложной. Принцип Сен-Венана гласит, что \textbf{уравновешенная
система внешних сил, приложенная к упругому телу, когда все точки
приложения сил этой системы лежат внутри некоторой сферы, производит
пренебрежимо малые деформации в точках тела, на расстояниях от
сферы, достаточно больших по сравнению с её радиусом.} Таким образом,
при решении задач заданная система сил, приложенная к небольшой части
упругого тела, заменяется другой, удобной для упрощения задачи,
статически эквивалентной системой сил (т.е. имеющей те же главный
вектор и главный момент), приложенной к той же части тела.


\textbf{Простейшие задачи теории упругости} – это краевые задачи равновесия (также называются задачи статики или статические задачи) однородных изотропных тел, в которых в любой точке тела
компоненты напряжений (а значит и деформаций) постоянны или линейно зависят от координат.
Уравнения совместности выполнены тождественно, потому что вторые частные производные от констант
и линейных функций равны нулю. Уравнения Бельтрами-Митчелла выполняются тождественно (что не гарантирует выполнения уравнений равновесия), если
массовые силы являются константами, т. е. не зависят от координат точек тела. В ином случае эти
уравнения являются условиями, которым должны удовлетворять массовые силы, чтобы в теле было
реализовано простейшее напряженно-деформированное состояние.


\textbf{Полуобратный метод Сен-Венана} – метод решения задач, при котором частично задаются напряжения и
перемещения, а затем при помощи уравнений теории упругости определяются условия, которым должны
удовлетворять неизвестные компоненты напряжений и перемещений. На последнем этапе решения
можно воспользоваться формулами Чезаро:
$$
u_i^A=u_i^O+\omega_{j i}^O\left(x_j^A-x_j^O\right)+\int_O^A\left[\varepsilon_{i j}+\left(x_k^A-x_k\right)\left(\varepsilon_{i j, k}-\varepsilon_{j k, i}\right)\right] d x_j.
$$
Упростим формулы Чезаро для случая, когда все компоненты $\varepsilon_{i j, k}$ являются константами: 
Примем за $O$ начало координат и исключим движение абсолютно твердого тела: $x_i^O=u_i^O=\omega_{i j}^O=0$.
Криволинейный интеграл не зависит от пути интегрирования. Предполагая тело выпуклым, соединим точки $O$ и $A$ прямолинейным отрезком и параметризуем переменной $t \in[0 ; 1]$.

\begin{figure}[h!]
  \centering
  \includegraphics[width=0.2\textwidth]{images/14.2.jpg}   
\end{figure}

Тогда  
$$
\begin{aligned}
& \begin{array}{l}
x_k=x_k^A t \\
x_j=x_j^A t
\end{array}: \int_0^A\left(x_k^A-x_k\right)\left(\varepsilon_{i j, k}-\varepsilon_{j k, i}\right) d x_j=\left(\varepsilon_{i j, k}-\varepsilon_{j k, i}\right) \int_0^1\left(x_k^A-x_k^A t\right) x_j^A d t= \\
& =\left(\varepsilon_{i j, k}-\varepsilon_{j k, i}\right) x_k^A x_j^A \int_0^1(1-t) d t=\frac{\left(\varepsilon_{i j, k}-\varepsilon_{j k, i}\right)}{2} x_k^A x_j^A \\
& u_i=x_j \int_0^1 \varepsilon_{i j}^1\left(x_1 t, x_2 t, x_3 t\right) d t+\frac{x_k x_j}{2}\left(\varepsilon_{i j, k}-\varepsilon_{j k, i}\right)
\end{aligned}
$$
Если все компоненты постоянны:$\displaystyle u_i=x_j \int_0^1 \varepsilon_{i j} d t =  x_j \varepsilon_{i j} \int_0^1 d t=\varepsilon_{i j} x_j$


$\displaystyle
u_i=\varepsilon_{i j} x_j, \vec{u}=\underset{\sim}{\varepsilon} \cdot \vec{r}=\vec{r} \cdot \underset{\sim}{\varepsilon} \Rightarrow
\vec{u}=\vec{u}^O-{\underset{\sim}{\omega}}^O \cdot \vec{r}+\underset{\sim}{\varepsilon} \cdot \vec{r}=\vec{u}^O+\left(\underset{\sim}{\varepsilon}-{\underset{\sim}{\omega}}^O\right) \cdot \vec{r}
$


\textbf{Задача о кручении призматического бруса круглого сечения}


На торцах круглого призматического бруса с осью $Ox_3$  действуют пары сил,
моменты которых равны по величине и имеют противоположные знаки.

\begin{figure}[h!]
  \centering
  \subfigure{
    \includegraphics[width=0.5\textwidth]{images/16.1.jpg}}
  \subfigure{\includegraphics[width=0.4\textwidth]{images/16.2.jpg}}  
\end{figure}


Боковая
поверхность свободна, массовые силы не учитываются. Предположим, что
возникшая деформация чистого кручения описывается тензором деформации:
$$
\left[\varepsilon_{i j}\right]=\left(\begin{array}{ccc}
0 & 0 & -\frac{\kappa}{2} x_2 \\
0 & 0 & \frac{\kappa}{2} x_1 \\
-\frac{\kappa}{2} x_2 & \frac{\kappa}{2} x_1 & 0
\end{array}\right)
$$
Исключим движение абсолютно твердого тела
и найдем вектор перемещения из формул Чезаро, приняв: $u_i^O=\omega_{ji}^O=x_i^O=0$
Учтем, что диагональные элементы равны нулю и нет зависимости от $x_3$:


$ \displaystyle
u_1=x_j \int_0^1 \varepsilon_{1 j} d t+\frac{x_k x_j}{2}\left(\varepsilon_{1 j, k}-\varepsilon_{j k, 1}\right)=x_3 \int_0^1 \varepsilon_{13} d t+\frac{x_2 x_3}{2} \varepsilon_{13,2}-\frac{x_3 x_2}{2} \varepsilon_{23,1}-\frac{x_2 x_3}{2} \varepsilon_{32,1}
=x_3 \int_0^1\left(-\frac{\kappa x_2 t}{2}\right) d t+\frac{x_2 x_3}{2} \varepsilon_{13,2}-x_2 x_3 \varepsilon_{23,1}=\frac{-\kappa}{2} x_2 x_3 \int_0^1 t d t-\frac{\kappa}{2} \frac{x_2 x_3}{2}-\frac{\kappa}{2} x_2 x_3=\left(-\frac{1}{2} \frac{1}{2}-\frac{1}{4}-\frac{1}{2}\right) \kappa x_2 x_3=-\kappa x_2 x_3$


$ \displaystyle
u_2=x_j \int_0^1 \varepsilon_{2 j} d t+\frac{x_k x_j}{2}\left(\varepsilon_{2 j, k}-\varepsilon_{j k, 2}\right)=x_3 \int_0^1 \varepsilon_{23} d t+\frac{x_1 x_3}{2} \varepsilon_{23,1}-\frac{x_3 x_1}{2} \varepsilon_{13,2}-\frac{x_1 x_3}{2} \varepsilon_{31,2}=x_3 \int_0^1 \frac{\kappa x_1 t}{2} d t
+\frac{x_1 x_3}{2} \varepsilon_{23,2}-x_1 x_3 \varepsilon_{13,2}=\frac{\kappa}{2} x_1 x_3 \int_0^1 t d t+\frac{\kappa}{2} \frac{x_1 x_3}{2}+\frac{\kappa}{2} x_1 x_3=\frac{\kappa}{2} x_1 x_3 \frac{1}{2}+\frac{\kappa x_1 x_3}{4}+\frac{\kappa}{2} x_1 x_3=\kappa x_1 x_3$


$\displaystyle
u_3= x_j \int_0^1 \varepsilon_{3 j} d t+\frac{x_k x_j}{2}\left(\varepsilon_{3 j, k}-\varepsilon_{j k, 3}\right)=x_1 \int_0^1 \varepsilon_{31} d t+x_2 \int_0^1 \varepsilon_{32} d t+\frac{x_2 x_1}{2} \varepsilon_{31,2}+\frac{x_1 x_2}{2} \varepsilon_{32,1}=x_1 \int_0^1-\frac{\kappa x_2 t}{2} d t+x_2 \int_0^1 \frac{\kappa x_1 t}{2} d t+\frac{x_2 x_1}{2}(-\kappa)+\frac{x_1 x_2}{2} \kappa=0 $


Получаем: $\quad u_1=-\kappa x_2 x_3 ; \quad u_2=\kappa x_1 x_3; \quad u_3=0 \quad \Rightarrow \quad$ Поперечные сечения остаются
плоскими и поворачиваются на
некоторый угол, который
различен для разных сечений.


Закон Гука при кручении круглого призматического бруса:
$$
-G \kappa I_O \vec{k}={\vec{L}=M \cdot(-\vec{k})} \Rightarrow \kappa=\frac{M}{G I_O}
$$


\section{Ослабление граничных условий: принцип Сен-Венана. Определение простейших задачах теории упругости. Формула Чезаро для простейших задач. Полуобратный метод Сен-Венана: решение задачи о чистом изгибе призматического бруса.}

\begin{figure}[h!]
  \centering
  \includegraphics[width=0.4\textwidth]{images/14.1.jpg} 
\end{figure}


Точный учёт граничных условий (усилий и перемещений) делает краевую
задачу очень сложной. Принцип Сен-Венана гласит, что \textbf{уравновешенная
система внешних сил, приложенная к упругому телу, когда все точки
приложения сил этой системы лежат внутри некоторой сферы, производит
пренебрежимо малые деформации в точках тела, на расстояниях от
сферы, достаточно больших по сравнению с её радиусом.} Таким образом,
при решении задач заданная система сил, приложенная к небольшой части
упругого тела, заменяется другой, удобной для упрощения задачи,
статически эквивалентной системой сил (т.е. имеющей те же главный
вектор и главный момент), приложенной к той же части тела.


\textbf{Простейшие задачи теории упругости} – это краевые задачи равновесия (также называются задачи статики или статические задачи) однородных изотропных тел, в которых в любой точке тела
компоненты напряжений (а значит и деформаций) постоянны или линейно зависят от координат.
Уравнения совместности выполнены тождественно, потому что вторые частные производные от констант
и линейных функций равны нулю. Уравнения Бельтрами-Митчелла выполняются тождественно (что не гарантирует выполнения уравнений равновесия), если
массовые силы являются константами, т. е. не зависят от координат точек тела. В ином случае эти
уравнения являются условиями, которым должны удовлетворять массовые силы, чтобы в теле было
реализовано простейшее напряженно-деформированное состояние.


\textbf{Полуобратный метод Сен-Венана} – метод решения задач, при котором частично задаются напряжения и
перемещения, а затем при помощи уравнений теории упругости определяются условия, которым должны
удовлетворять неизвестные компоненты напряжений и перемещений. На последнем этапе решения
можно воспользоваться формулами Чезаро:
$$
u_i^A=u_i^O+\omega_{j i}^O\left(x_j^A-x_j^O\right)+\int_O^A\left[\varepsilon_{i j}+\left(x_k^A-x_k\right)\left(\varepsilon_{i j, k}-\varepsilon_{j k, i}\right)\right] d x_j.
$$
Упростим формулы Чезаро для случая, когда все компоненты $\varepsilon_{i j, k}$ являются константами: 
Примем за $O$ начало координат и исключим движение абсолютно твердого тела: $x_i^O=u_i^O=\omega_{i j}^O=0$.
Криволинейный интеграл не зависит от пути интегрирования. Предполагая тело выпуклым, соединим точки $O$ и $A$ прямолинейным отрезком и параметризуем переменной $t \in[0 ; 1]$.

\begin{figure}[h!]
  \centering
  \includegraphics[width=0.2\textwidth]{images/14.2.jpg}
\end{figure}

Тогда   
$$
\begin{aligned}
& \begin{array}{l}
x_k=x_k^A t \\
x_j=x_j^A t
\end{array}: \int_0^A\left(x_k^A-x_k\right)\left(\varepsilon_{i j, k}-\varepsilon_{j k, i}\right) d x_j=\left(\varepsilon_{i j, k}-\varepsilon_{j k, i}\right) \int_0^1\left(x_k^A-x_k^A t\right) x_j^A d t= \\
& =\left(\varepsilon_{i j, k}-\varepsilon_{j k, i}\right) x_k^A x_j^A \int_0^1(1-t) d t=\frac{\left(\varepsilon_{i j, k}-\varepsilon_{j k, i}\right)}{2} x_k^A x_j^A \\
& u_i=x_j \int_0^1 \varepsilon_{i j}^1\left(x_1 t, x_2 t, x_3 t\right) d t+\frac{x_k x_j}{2}\left(\varepsilon_{i j, k}-\varepsilon_{j k, i}\right)
\end{aligned}
$$
Если все компоненты постоянны:$\displaystyle u_i=x_j \int_0^1 \varepsilon_{i j} d t =  x_j \varepsilon_{i j} \int_0^1 d t=\varepsilon_{i j} x_j$


$\displaystyle
u_i=\varepsilon_{i j} x_j, \vec{u}=\underset{\sim}{\varepsilon} \cdot \vec{r}=\vec{r} \cdot \underset{\sim}{\varepsilon} \Rightarrow
\vec{u}=\vec{u}^O-{\underset{\sim}{\omega}}^O \cdot \vec{r}+\underset{\sim}{\varepsilon} \cdot \vec{r}=\vec{u}^O+\left(\underset{\sim}{\varepsilon}-{\underset{\sim}{\omega}}^O\right) \cdot \vec{r}
$



\textbf{Задача о чистом изгибе призматического бруса}

\begin{figure}[h!]
  \centering
  \includegraphics[width=0.5\textwidth]{images/17.1.jpg}  
\end{figure}



Массовыми силами пренебрегаем $\vec{F}=\vec{0}$.
Примем $\sigma_{ij}=ax_2\delta_{i3}\delta_{j3}$.  Уравнения Бельтрами-Митчелла удовлетворяются тождественно. Уравнения
равновесия тоже выполнены: $\sigma_{ij,j}=ax_{2,j}\delta_{i3}\delta_{j3}=a\sigma_{2j}\delta_{i3}\delta_{j3}=a\delta_{i3}\delta_{23}=0$.


Найдем вектор напряжений на боковине с $\vec{n}=(n_1, n_2, 0): P_i^{(\vec{n})}=\sigma_{ij}n_j=0$ 


На торцах с $\vec{n}=(0, 0, \mp1): P_i^{(\mp\vec{k})}=\mp a x_2 \sigma_{i3}$ 


Боковая поверхность является свободной от внешних сил. На торцах
действуют линейно-распределенные нормальные напряжения. Определим главный вектор и главный
момент сил, действующих на торцах:
$$
\int_S P_3^{(\mp \vec{k})} d \Sigma=\mp a \int_S x_2 d \Sigma=\mp a S x_2^{\text{ЦМ}}=0=\int_S P_1^{(\mp \vec{k})} d \Sigma=\int_S P_2^{(\mp \vec{k})} d \Sigma
$$

$I=I_{22}$ - момент инерции
сечения относительно
нейтральной оси $Ox_1$.
(Т.к. $Ox_1$ и $Ox_2$ - главные оси
инерции сечения, то $I_{12}$ = 0.)

$$
\begin{aligned}
& \vec{L}=\int_S \vec{r} \times \vec{P}^{(\vec{k})} d \Sigma=\int_S\left|\begin{array}{ccc}
\vec{i} & \vec{j} & \vec{k} \\
x_1 & x_2 & 0 \\
0 & 0 & a x_2
\end{array}\right| d \Sigma=\left(a \int_S x_2^2 d \Sigma\right) \vec{i}-\left(a \int_S x_1 x_2 d \Sigma\right) \vec{j}= \\
& =a I_{22} \vec{i}-a I_{12} \vec{j}=a I_{22} \vec{i} \equiv a I \cdot \vec{i} 
\end{aligned}
$$
Практически приложить внешние силы вида $\vec{P}^{( \pm \vec{k})}= \pm a x_2 \vec{k}$ невозможно. Следуя принципу Сен-Венана, приложим нагрузку с главным изгибающим моментом: $\vec{M}=\vec{L} \Rightarrow M=a I \Rightarrow \sigma_{i j}=(M / I) x_2 \delta_{i 3} \delta_{j 3}$. По обобщенному закону Гука:
$$
\sigma=\sigma_{k k}=\frac{M}{3 I_{22}} x_2 ; \quad \varepsilon_{i j}=-\frac{3 v}{E} \sigma \delta_{i j}+\frac{1+v}{E} \sigma_{i j}=\left(-\frac{v}{E} \delta_{i j}+\frac{1+v}{E} \delta_{i 3} \delta_{j 3}\right) \frac{M}{I} x_2
$$
$
\displaystyle
 \text { Главные оси бруса совпадают с главными осями тензоров } \\ \text { напряжений и деформаций. Компоненты тензоров зависят } \\ \text { только от } x_2, \text { а матрица в главных осях является диагональной. } \\ \varepsilon_{12}=\varepsilon_{13}=\varepsilon_{23}=0
\varepsilon_{11}=\varepsilon_{22}=-\frac{v M}{E I} x_2 ; \varepsilon_{33}=\frac{M}{E I} x_2 ; \\ u_3=x_j \int_0^1 \varepsilon_{3 j} d t+\frac{x_k x_j}{2}\left(\varepsilon_{3 j, k}-\varepsilon_{j k, 3}\right)=x_3 \int_0^1 \varepsilon_{33}\left(x_2 t\right) d t+\frac{x_2 x_3}{2} \varepsilon_{33,2}=x_3 \frac{M}{E I} x_2 \int_0^1 t d t+\frac{x_2 x_3}{2} \frac{M}{E I}=\frac{M}{E I} x_2 x_3 \\ u_1=x_j \int_0^1 \varepsilon_{1 j} d t+\frac{x_k x_j}{2}\left(\varepsilon_{1 j, k}-\varepsilon_{j k, 1}\right)=x_1 \int_0^1 \varepsilon_{11}\left(x_2 t\right) d t+\frac{x_2 x_1}{2} \varepsilon_{11,2}=\frac{-v M}{E I} x_1 x_2\left(\int_0^1 t d t+\frac{1}{2}\right)=-\frac{v M}{E I} x_1 x_2 \\ u_2=x_j \int_0^1 \varepsilon_{2 j} d t+\frac{x_k x_j}{2}\left(\varepsilon_{2 j, k}-\varepsilon_{j k, 2}\right)=x_2 \int_0^1 \varepsilon_{22} d t+\frac{x_2 x_2}{2} \varepsilon_{22,2}-\frac{x_1 x_1}{2} \varepsilon_{11,2}-\frac{x_2 x_2}{2} \varepsilon_{22,2}-\frac{x_3 x_3}{2} \varepsilon_{33,2} =-\frac{v M}{E I} x_2^2 \int_0^1 t d t+\frac{x_1^2}{2} \frac{v M}{E I}-\frac{x_3^2}{2} \frac{M}{E I}=\frac{-M}{2 E I}\left(x_3^2+v\left[x_2^2-x_1^2\right]\right)$

$$\text{Получаем:} u_2=\frac{-M}{2 E I}\left(x_3^2+v\left[x_2^2-x_1^2\right]\right) ; \quad u_1=-\frac{v M}{E I} x_1 x_2 ; \quad u_3=\frac{M}{E I} x_3 x_2 $$

Найдем, как деформируется ось бруса:
$ x_1=x_2=0 \Rightarrow u_2=\frac{-M}{2 E I} x_3^2, u_1=u_3=0 \Rightarrow$ Ось бруса остается в плоскости изгиба $Ox_2x_3$ и
деформируется в параболу. Кривизна оси бруса с точностью до  малых высшего порядка равна: $ K=\frac{1}{R}=\frac{d^2 u_2}{d x_3{ }^2} \Rightarrow K=\frac{-M}{E I} \Rightarrow $ Закон Гука причистом изгибе: $\frac{1}{R}=\frac{-M}{E I}$



\section{Использование криволинейных координат: задача Ламе о деформировании толстостенной упругой однородной изотропной трубы под действием внешнего и внутреннего давления (кинематическая гипотеза; сведение задачи к краевой задаче обыкновенного дифференциального уравнения; уравнение Леви; формулы ненулевых компонент тензора напряжений).}
Задача Ламе о толстостенной трубе - задача о равновесии цилиндрической трубы, находящейся под действием внутреннего и внешнего давления. $\vec{F}=\overrightarrow{0}$


Кинематическая гипотеза: $\overrightarrow{\mathrm{u}}=u(\rho) \vec{e}_\rho$. 

\begin{figure}[h!]
  \centering
  \includegraphics[width=0.4\textwidth]{images/18.1.jpg}
\end{figure}



Запишем
уравнения Ламе и граничные условия в цилиндрической системе координат. 
$ \xi^1=\rho$, $ \xi^2=\varphi$, $\xi^3=z$; $\quad x_1 \equiv x=\rho \cos \varphi$, $x_2 \equiv y=\rho \sin \varphi$, $x_3 \equiv z=z$.
$(\Gamma_{12}^2=\Gamma_{21}^2=1 / \rho, \Gamma_{22}^1=-\rho) $ $(g_{11}=g_{33}=1, g_{22}=\rho^2 ; \sqrt{g}=\rho ; g^{11}=g^{33}=1, g^{22}=\rho^{-2})$
Проверяем кинематическую гипотезу:
$\left\{ \begin{array}{c}
     \vec{e}_\rho=\vec{e}_1=\vec{e^1}\\
     \vec{e}_\varphi=\rho^{-1} \vec{e}_2=\rho \vec{e^2}\\
      \vec{e}_z=\vec{e}_3=\vec{e^{3}}\\
        \end{array}\right.$
$\left\{ \begin{array}{c}
     u_{\rho}=u_1=u^1=u(\rho) \\
      u_{\varphi}=\rho u^2=\rho^{-1} u_2=0\\
      u_z=u^3=u_3=0\\
        \end{array}\right.$
 
$ 
\displaystyle
\mathrm{div} \vec{u}=\frac{\partial u_\rho}{\partial \rho}+\frac{1}{\rho} \frac{\partial u_{\varphi}}{\partial \phi}+\frac{\partial u_z}{\partial z}+\frac{u_\rho}{\rho}=\frac{d u}{d \rho}+\frac{u}{\rho} \Rightarrow \mathrm{grad} \mathrm{div} \vec{u}=\vec{e^i} \frac{\partial \mathrm{div} \vec{u}}{\partial \xi^i}=\vec{e}_\rho \frac{d}{d \rho}\left(\frac{d u}{d \rho}+\frac{u}{\rho}\right)
$

$$
u_{1,2}=\frac{\partial u}{\partial \varphi}-u_q \Gamma_{12}^q=-u_2 \Gamma_{12}^2 \equiv 0 \quad
u_{1,3}=\frac{\partial u}{\partial z}-u_q \Gamma_{13}^q \equiv 0
$$

$$
\mathrm{rot} \vec{u}=\frac{1}{\sqrt{g}} \mathrm{det}\left(\begin{array}{ccc}
\vec{e^1} & \vec{e^2} & \vec{e^3} \\
\nabla_1 & \nabla_2 & \nabla_3 \\
u_1 & 0 & 0
\end{array}\right)=\frac{1}{\sqrt{g}}\left(u_{1,3} \vec{e^2}-u_{1,2} \vec{e^3}\right) \equiv \overrightarrow{0}
$$

Уравнение Ламе в отсутствии массовых сил: $(\lambda+2 \mu) \mathrm{grad} \mathrm{div} \vec{u}+\mu \mathrm{rot} \mathrm{rot} \vec{u}=0$. 

$
\displaystyle
\begin{array}{l}

\mathrm{rot} \vec{u}=0 \stackrel{\vec{F}=\overline{0}}{\Rightarrow} \mathrm{grad} \mathrm{div} \vec{u}=0 \stackrel{\bar{u}={\bar{u_\rho}}(\rho)}{\Rightarrow} \frac{d}{d \rho}\left(\frac{d u}{d \rho}+\frac{u}{\rho}\right)=0 \Leftrightarrow \frac{d}{d \rho}\left(\frac{1}{\rho} \frac{d(\rho u)}{d \rho}\right)=0 \Leftrightarrow \frac{d(\rho u)}{d \rho}= \\ =A \rho \Leftrightarrow 
u=\frac{A \rho}{2}+\frac{B}{\rho} \quad u^{\prime}=\frac{A}{2}-\frac{B}{\rho^2}
\end{array}
$


$
\varepsilon_{i j} =\frac{1}{2}\left(u_{j, i}+u_{i, j}\right)=\frac{1}{2}(\frac{\partial u_j}{\partial \xi^i}-u_q \Gamma_{j i}^q+\frac{\partial u_i}{\partial \xi^j}-u_q \Gamma_{i j}^q)=\frac{1}{2}\left(\frac{\partial u_i}{\partial \xi^j}+\frac{\partial u_j}{\partial \xi^i}\right)-u_q \Gamma_{i j}^q= \frac{1}{2}\left(\frac{\partial u_1}{\partial \xi^1}+\frac{\partial u_1}{\partial \xi^1}\right) \delta_{i 1} \delta_{j 1}-u_1 \Gamma_{22}^1 \delta_{i 2} \delta_{j 2}=u^{\prime} \delta_{i 1} \delta_{j 1}+\rho u \delta_{i 2} \delta_{j 2}=\left(\frac{A \rho}{2}-\frac{B}{\rho^2}\right) \delta_{i 1} \delta_{j 1}+\left(\frac{A}{2} \rho^2+B\right) \delta_{i 2} \delta_{j 2}
$

$\left\{ \begin{array}{c}
    \varepsilon^{i j}=g^{i p} g^{j q} \varepsilon_{p q}\\
    \varepsilon_i^j=g^{j q} \varepsilon_{i q} \\
        \end{array}\right.
$ 

$\left\{ \begin{array}{c}
    \varepsilon_{11}=u^{\prime} =\frac{A}{2}-\frac{B}{\rho^2} \\
    \varepsilon_{22}=\rho u  \\
     \varepsilon_{12}=\varepsilon_{13}=\varepsilon_{23}=\varepsilon_{33}=0 \\
        \end{array}\right.
$

$\left\{\begin{array}{c}
   \varepsilon_1^1=\varepsilon^{11}=\varepsilon_{\rho \rho}=u^{\prime} \\
   \varepsilon^{22}=\rho^{-3} u \\
     \varepsilon_2^2=\rho^{-1} u=\varepsilon_{\varphi \varphi}\\
      \varepsilon_{z z}=\varepsilon_3^3=\varepsilon^{12}=\varepsilon^{13}=\varepsilon_{i=j}^{23}=\varepsilon^{33}=\varepsilon_2^1=\varepsilon_3^1=\varepsilon_3^2=\varepsilon_3^3=0 \\
        \end{array}\right.$


$\theta=\varepsilon_i^i=u^{\prime}+\rho^{-1} u=A \Rightarrow \theta=A $
(одинакова во всех точках)

Закон Гука:
$
\sigma_i^j=\lambda \theta \delta_i^j+2 \mu \varepsilon_i^j \Rightarrow \sigma_{i i}^{\phi \text { физ }}=A \lambda+2 \mu \varepsilon_{i i}^{\phi \text { физ }} \quad(j=i) \text {. } 
$
$\left\{ \begin{array}{c}
    \sigma_{\rho \rho}=A \lambda+2 \mu u^{\prime}\\
     \sigma_{\varphi \varphi}=A \lambda+2 \mu \rho^{-1} u  \\
   \sigma_{z z}=A \lambda \\
    \sigma_{\rho \varphi}=\sigma_{\rho z}=\sigma_{\varphi z}=0 \\
        \end{array}\right.
$ 


$
\sigma_{\rho \rho}+\sigma_{\varphi \varphi}=2 A \lambda+2 \mu\left(u^{\prime}+\rho^{-1} u\right)=2 A(\lambda+\mu)=\sigma_{z z} \cdot 2(\lambda+\mu) / \lambda=\sigma_{z z} / v
$


\textbf{Уравнение Леви:} $\quad \sigma_{z z}=v\left(\sigma_{\rho \rho}+\sigma_{\varphi \varphi}\right)$



$
\begin{aligned}
\vec{n} &= -\vec{e}_\rho \quad \Rightarrow \quad \vec{P}^{(-\vec{e}_\rho)}\Big|_{\rho=a} = \sigma^{ij} n_i \vec{e}_j = -\sigma^{1j} \vec{e}_j = -\sigma_{\rho\rho} \vec{e}_\rho \\
\vec{n} &= \vec{e}_\rho \quad \Rightarrow \quad \vec{P}^{(\vec{e}_\rho)}\Big|_{\rho=b} = \sigma^{ij} n_i \vec{e}_j = \sigma^{1j} \vec{e}_j = \sigma_{\rho\rho} \vec{e}_\rho
\end{aligned}$

\begin{figure}[h!]
  \centering
  \includegraphics[width=0.4\textwidth]{images/18.2.jpg}
\end{figure}


$\begin{aligned}
& \begin{array}{l}-\left.\sigma_{\rho \rho}\right|_{\rho=a}=p_a \Rightarrow A \lambda+\mu\left(A-2 a^{-2} B\right)=-p_a \\ \left.\sigma_{\rho \rho}\right|_{\rho=b}=-p_b \Rightarrow A \lambda+\mu\left(A-2 b^{-2} B\right)=-p_b\end{array} \quad \Rightarrow\left\{\begin{array}{l}2 \mu a^{-2} B-(\lambda+\mu) A=p_a \\ 2 \mu b^{-2} B-(\lambda+\mu) A=p_b\end{array}\right. \\ & A=\frac{p_a b^{-2}-p_b a^{-2}}{(\lambda+\mu)\left(a^{-2}-b^{-2}\right)} ; \quad B=\frac{p_a-p_b}{2 \mu\left(a^{-2}-b^{-2}\right)} \Rightarrow \\ & A=\frac{p_a a^2-p_b b^2}{(\lambda+\mu)\left(b^2-a^2\right)} ; \quad B=\frac{a^2 b^2\left(p_a-p_b\right)}{2 \mu\left(b^2-a^2\right)} \\ & \sigma_{\rho \rho} (\rho) =\frac{p_a a^2-p_b b^2}{b^2-a^2}-\frac{a^2 b^2\left(p_a-p_b\right)}{b^2-a^2} \frac{1}{\rho^2}=\frac{p_a a^2}{b^2-a^2}\left(1-\frac{b^2}{\rho^2}\right)-\frac{p_b b^2}{b^2-a^2}\left(1-\frac{a^2}{\rho^2}\right) \\ & \sigma_{\varphi\varphi}(\rho)=A \lambda+2 \mu \rho^{-1} u=A \lambda+\mu\left(A+2 \rho^{-2} B\right)=A(\lambda+\mu)+2 \mu B \rho^{-2}\
\\ &\sigma_{\varphi \varphi}(\rho)=\frac{p_a a^2}{b^2-a^2}\left(1+\frac{b^2}{\rho^2}\right)-\frac{p_b b^2}{b^2-a^2}\left(1+\frac{a^2}{\rho^2}\right)\ &
\\ & \end{aligned}$

При условии: $b>a, a \approx b, p_a \gg p_b \Rightarrow A>0, B>0$ 
$$
\sigma_{\varphi \varphi}>0, \max \left|\sigma_{\varphi \varphi}\right|=\sigma_{\varphi \varphi}(a)=\frac{p_a a^2+b^2\left(p_a-2 p_b\right)}{b^2-a^2}
$$
$$
\sigma_{z z}=v\left(\sigma_{\rho \rho}+\sigma_{\varphi \varphi}\right)=2 v \frac{p_a a^2-p_b b^2}{b^2-a^2}>0
$$
Компоненты завяисят только от геометрических характеристик трубы.

\section{Эффективные определяющие соотношения. Эффективный модуль Фойхта. Прямая формула смесей.}
- \textbf{Определяющие соотношения} - это соотношения между причинами и следствиями процессов, которые происходят в телах. Примеры: уравнение Менделеева-Клайперона; закон Гука; закон Навье-Стокса.

- \textbf{Эффективные определяющие соотношения} - это соотношения, связывающие между собой средние по представительному объёму значения причин и следствий. Материальные константы и материальные функции, входящие в эти соотношения, называются эффективными физико-механическими свойствами (характеристиками) композита.

- Экспериментальное определение всего комплекса эффективных свойств композита является дорогим и долгим процессом. Поэтому необходимы математические методы вычисления эффективных характеристик, позволяющие с разной степенью точности вычислять эффективные свойства.

- В МДТТ под эффективными определяющими соотношениями понимаются соотношения позволяющие выразить средние напряжения через средние деформации ( $\langle\underset{\sim}{\boldsymbol{\sigma}}\rangle \sim\langle\underset{\sim}{\boldsymbol{\varepsilon}}\rangle$ - это прямые соотношения), либо наоборот средние деформации через средние напряжения ($\langle\underset{\sim}{\boldsymbol{\varepsilon}}\rangle \sim\langle\underset{\sim}{\boldsymbol{\sigma}}\rangle$ - это обратные соотношения).

- Материальные константы и материальные функции, входящие в эти соотношения, называются эффективными константами и соответственно эффективными функциями композита.

- В общем случае эффективные материальные константы и функции, получаемые из прямых и обратных определяющих соотнощений, не являются взаимно обратными, то есть эффективные прямые и обратные соотношения не обязательно совпадают. 

$$
\langle\bullet\rangle \equiv \frac{1}{V} \int \limits_{V}(\bullet) d V \quad
\lambda=\frac{E \nu}{(1-2 \nu)(1+\nu)} \quad \mu=\frac{E}{2(1+\nu)}
$$

\textbf{Эффективный модуль Фойхта} определим при следующих предположениях: 1) фазы являются изотропными и линейно-упругими; 2) нет массовых сил; 3) тело находится в состоянии статического равновесия; 4) поле деформаций однородно:
$
\varepsilon_{ij}=const \Rightarrow
\left\langle\sigma_{i j}(\vec{x})\right\rangle=\left\langle C_{i j k l} \varepsilon_{k l}\right\rangle=\left\langle C_{i j k l}\right\rangle \varepsilon_{k l}=\left\langle C_{i j k l}\right\rangle\left\langle\varepsilon_{k l}\right\rangle \Rightarrow C_{i j k l}^{\mathrm{F}}=\left\langle C_{i j k l}\right\rangle
$

Рассмотрим линейно упругий однородный отрезок: 
\includegraphics[width=0.3\textwidth]{images/19.1.jpg}

$x \in(0 ; L), \varepsilon(x) \equiv \varepsilon=\langle\varepsilon\rangle_x$
$\sigma(x)=E(x) \varepsilon \Rightarrow\langle\sigma(x)\rangle_x=\langle E(x) \varepsilon\rangle_x=\langle E(x)\rangle_x \varepsilon=\langle E(x)\rangle_x\langle\varepsilon\rangle_x \Rightarrow E_x^F=\langle E(x)\rangle_x \equiv \frac{1}{L} \int_0^L E(x) d x $

\textbf{Эффективный модуль Фойхта} можно приближенно определить из решения 3D краевой задачи на одноосное растяжение неоднородного стержня, когда нагруженная грань остаётся плоской и перпендикулярной оси стержня. Рассмотрим градиентный по ширине прямоугольный стержень, такой, что коэффициент Пуассона меняются мало: $\left|v^{\prime}(x)\right| \ll 1$. Пусть $u_3=\varepsilon z$, $, u_1=-v \varepsilon x, u_2=-v \varepsilon y, \varepsilon=$ const - относительное удлинение стержня. Условия Сен-Венана выполнены тождественно. Убедимся, что тензор деформаций является приближенно постоянным и оси $O X Y Z$ - главные:

\begin{figure}[h!]
  \centering
\includegraphics[width=0.4\textwidth]{images/19.2.jpg}
\end{figure}



$
\varepsilon_{33}=u_{3,3}=\varepsilon,\quad
\varepsilon_{11}=u_{1,1}=-v^{\prime}(\varepsilon x)-v \varepsilon \approx-v \varepsilon,\quad \varepsilon_{22}=u_{2,2}=-v \varepsilon,\quad
2 \varepsilon_{12}=u_{1,2}+u_{2,1}=u_{2,1}=-v^{\prime} \varepsilon y \approx 0, \varepsilon_{13}=\varepsilon_{23}=0 \\
\sigma_{i j}=\lambda(x) \theta \delta_{i j}+2 \mu(x) \varepsilon_{i j} \Rightarrow \sigma_{33}=\lambda \theta+2 \mu \varepsilon_{33}=(\lambda(1-2 v)+2 \mu) \varepsilon=E(x) \varepsilon \\
\sigma_{11}=\lambda \theta+2 \mu \varepsilon_{11}=(\lambda(1-2 v)-2 \mu v) \varepsilon=\left(\frac{E v}{1+v}-\frac{E v}{1+v}\right) \varepsilon=0=\sigma_{22}=\sigma_{12}=\sigma_{13}=\sigma_{23}
$
($\triangle Z / Z=u_3(Z) / Z=\varepsilon=\left\langle\varepsilon_{33}\right\rangle$ )


Уравнение равновесия:

$
\sigma_{i j, 1}=0 \Rightarrow \sigma_{11,1}+\sigma_{12,2}+\sigma_{13,3}=0, \sigma_{21,1}+\sigma_{22,2}+\sigma_{23,3}=0, \sigma_{31,1}+\sigma_{32,2}+\sigma_{33,3}=0
$

Среднее по объему третьей компоненты напряжения: 

$\sigma_{33,3}=0 \Rightarrow\left\langle\sigma_{33}\right\rangle=\frac{1}{V} \int_{V} \sigma_{33} d V=\frac{1}{V} \int_{0}^{Z} d z \int_{S} \sigma_{33} d S=\frac{Z}{V} \int_{S} \sigma_{33} d S=\frac{1}{S} \int_{S} p d S=\frac{P}{S}$

Расчетное и экспериментальное определения эффективного модуля Фойхта:

$\left\langle\sigma_{33}\right\rangle=\langle E \varepsilon\rangle=\langle E\rangle \varepsilon^{V}=\langle E\rangle\left\langle\varepsilon_{33}\right\rangle \Rightarrow\langle E(x)\rangle=E^{\mathrm{F}}=\frac{P}{\varepsilon S}$

Кусочно-однородный стержень с сильно различными константами. Считаем, что в каждой из фаз приближенно реализовано состояние равномерного растяжения, так что:

\begin{figure}[h!]
  \centering
  \includegraphics[width=0.5\textwidth]{images/19.3.jpg}
\end{figure}




$\sigma_{i j}(x)=\sigma(x) \delta_{i3} \delta_{j3}, \sigma(x)=\left\{\begin{array}{l}
p_1, x \in\left[0 ; \ell_1\right) \\
p_2, x \in\left(\ell_1 ; L\right]
\end{array}, \frac{p_1}{E_{(1)}}=\varepsilon=\frac{p_2}{E_{(2)}}\right. (x=l_1)$



$
\displaystyle
    \frac{P}{S}=\frac{P_{(1)}}{S}+\frac{P_{(2)}}{S}=\frac{S_1}{S} \frac{P_{(1)}}{S_1}+\frac{S_2}{S} \frac{P_{(2)}}{S_2}=\gamma_1 P_1+\gamma_2 P_2=\gamma_1 E_{(1)} \frac{p_1}{E_{(1)}}+\gamma_2 E_{(2)} \frac{p_2}{E_{(2)}}=\left(\gamma_1 E_{(1)}+\gamma_2 E_{(2)}\right) \varepsilon\Rightarrow\gamma_1 E_{(1)}+\gamma_2 E_{(2)}=\frac{P}{\varepsilon S}
$


$
\displaystyle
    \left\langle\sigma_{33}\right\rangle=\frac{1}{V} \int \limits_{V} \sigma(x) d V=\frac{1}{V} \int \limits_{V_1} p_1 d V+\frac{1}{V} \int_{V_2} p_2 d V=\frac{V_1}{V} p_1+\frac{V_2}{V} p_2=\gamma_1 p_1+\gamma_2 p_2 
$


$
\displaystyle
    \Rightarrow\langle\sigma_{33}\rangle=\frac{P}{S} \Rightarrow E^F\equiv\frac{\langle\sigma_{33}\rangle}{\varepsilon}=\frac{PS}{\varepsilon S}=\gamma_1 E_{(1)}+\gamma_2 E_{(2)} \
$

$
\displaystyle
\varepsilon_{33}^{(1)}=\varepsilon_{33}^{(2)}=\ldots=\varepsilon_{33}^{(N)}=\varepsilon,
\left\langle\sigma_{33}\right\rangle=\sum_{i=1}^{N} \gamma_{i} p_{i}=\frac{P}{S}, 
$


\textbf{Прямая формула смесей:} 
$
\sum_{i=1}^{N} \gamma_{i} E_{(i)}=E^{\mathrm{F}}=\frac{Z \cdot P}{\Delta Z \cdot S}$

\section{Эффективные определяющие соотношения. Эффективный модуль Рейсса. Обратная формула смесей.}
- \textbf{Определяющие соотношения} - это соотношения между причинами и следствиями процессов, которые происходят в телах. Примеры: уравнение Менделеева-Клайперона; закон Гука; закон Навье-Стокса.

- \textbf{Эффективные определяющие соотношения} - это соотношения, связывающие между собой средние по представительному объёму значения причин и следствий. Материальные константы и материальные функции, входящие в эти соотношения, называются эффективными физико-механическими свойствами (характеристиками) композита.

- Экспериментальное определение всего комплекса эффективных свойств композита является дорогим и долгим процессом. Поэтому необходимы математические методы вычисления эффективных характеристик, позволяющие с разной степенью точности вычислять эффективные свойства.

- В МДТТ под эффективными определяющими соотношениями понимаются соотношения позволяющие выразить средние напряжения через средние деформации ($\langle\underset{\sim}{\boldsymbol{\sigma}}\rangle \sim\langle\underset{\sim}{\boldsymbol{\varepsilon}}\rangle$ - это прямые соотношения), либо наоборот средние деформации через средние напряжения ( $\langle\underset{\sim}{\boldsymbol{\varepsilon}}\rangle \sim\langle\underset{\sim}{\boldsymbol{\sigma}}\rangle$ - это обратные соотношения).

- Материальные константы и материальные функции, входящие в эти соотношения, называются эффективными константами и соответственно эффективными функциями композита.

- В общем случае эффективные материальные константы и функции, получаемые из прямых и обратных определяющих соотнощений, не являются взаимно обратными, то есть эффективные прямые и обратные соотношения не обязательно совпадают. 

\textbf{Эффективный модуль Рейсса} определим при следующих предположениях: 1) фазы являются изотропными и линейно-упругими; 2) нет массовых сил; 3) тело находится в состоянии статического равновесия; 4) поле напряжений однородно: 

$\sigma_{i j}=$ const. $\Rightarrow\left\langle\varepsilon_{i j}(\vec{x})\right\rangle=\left\langle C_{i j k l}^{-1} \sigma_{k l}\right\rangle=\left\langle C_{i j k l}^{-1}\right\rangle \sigma_{k l}=\left\langle C_{i j k l}^{-1}\right\rangle\left\langle\sigma_{k l}\right\rangle \Rightarrow C_{i j k l}^{\mathrm{R}}=\left\langle C_{i j k l}^{-1}\right\rangle^{-1}$. 
Рассмотрим сначала линейно-упруаий одномерный отрезок: 
\includegraphics[width=0.3\textwidth]{images/20.1.jpg}

$x \in(0 ; L), \sigma(x) \equiv \sigma=\langle\sigma\rangle$ 
$\displaystyle
\varepsilon(x)=\frac{\sigma}{E(x)} \Rightarrow\langle\varepsilon(x)\rangle_x=\left\langle\frac{1}{E(x)} \sigma\right\rangle_x=\left\langle\frac{1}{E(x)}\right\rangle_x \sigma=\left\langle\frac{1}{E(x)}\right\rangle_x\langle\sigma\rangle_x \Rightarrow E_x^{\mathrm{R}}=\left\langle\frac{1}{E(x)}\right\rangle_x^{-1} \equiv\left\langle\frac{1}{L} \int_0^L \frac{1}{E(x)} d x\right)
$

\vspace{0.5cm}

\textbf{Эффективный модуль Рейсса} можно приближенно определить из решения 3D краевой задачи на одноосное растяжение неоднородного стержня, когда нагруженная грань остаётся плоской и перпендикулярной оси стержня. Рассмотрим градиентный по длине прямоугольный стержень такой, что свойства меняются мало: 

$\left|v^{\prime}(x)\right| \ll 1, \left|E^{\prime}(x)\right| \ll 1$ 

\begin{figure}[h!]
  \centering
  \includegraphics[width=0.4\textwidth]{images/20.2.jpg}
\end{figure}



Предположим, что: $\sigma_{i j}=p \delta_{i 1} \delta_{j 1} \Rightarrow \mathrm{tr} \underset{\mathrm{\sigma }}{\mathrm{\sigma }}=p \Rightarrow \varepsilon_{i j}(x)=-\frac{v(x)}{E(x)} p \delta_{i j}+\frac{1+v(x)}{E(x)} \sigma_{i j}$
$
\varepsilon_{i j}(x)=\frac{p}{E(x)}\left(-v(x) \delta_{i j}+(1+v(x)) \delta_{i 1} \delta_{j 1}\right) \Rightarrow \varepsilon_{22}(x)=\varepsilon_{33}(x)=\frac{-v(x)}{E(x)} p, \varepsilon_{11}(x)=\frac{p}{E(x)}, \varepsilon_{12}=\varepsilon_{13}=\varepsilon_{23}=0
$


Уравнения совместности Сен-Венана: $\quad \varepsilon_{11,22}+\varepsilon_{22,11}=2 \varepsilon_{12,12} \Rightarrow \varepsilon_{22,11}=0 \quad$ 
$\quad\left(\frac{-v(x)}{E(x)} p\right)_{,11} \approx 0${\Large (?)}
Предложенное решение является приближенным. (Уравнения равновесия, краевые условия и четыре ур-ия Сен-Венана выполнены точно. Два ур-ия выполнены приближенно.)


Средняя по объему
продольная деформация:
$$
\varepsilon_{11,2}=\varepsilon_{11,3}=0 \Rightarrow\left\langle\varepsilon_{11}\right\rangle=\frac{1}{V} \int_V \varepsilon_{11} d V=\frac{1}{V} \int_S d S \int_0^L \varepsilon_{11} d x=\frac{S}{V} \int_0^L \varepsilon_{11} d x=\frac{\Delta L}{L}
$$
Расчетное и экспериментальное определения эффективного модуля Рейсса

$
\left\langle\varepsilon_{11}\right\rangle=\left\langle\frac{p}{E}\right\rangle=\left\langle\frac{1}{E}\right\rangle p=\left\langle\frac{1}{E}\right\rangle\left\langle\sigma_{11}\right\rangle \Rightarrow\left\langle\frac{1}{E(x)}\right\rangle^{-1}=E^{\mathrm{R}}=\frac{p L}{\Delta L} 
$

Кусочно-однородный стержень с сильно различными константами. Считаем, что в каждой из фаз приближенно реализовано состояние равномерного растяжения, так что:


\begin{figure}[h!]
  \centering
  \includegraphics[width=0.6\textwidth]{images/20.3.jpg}
\end{figure}



$\displaystyle
\sigma_{i j}=p \delta_{i 1} \delta_{j 1}, \varepsilon_{11}(x)=\left\{\begin{array}{l}
\varepsilon_{(1)}, x \in\left[0 ; \ell_1\right) \\
\varepsilon_{(2)}, x \in\left(\ell_1 ; L\right]
\end{array}, E_{(1)} \varepsilon_{(1)}=p=E_{(2)} \varepsilon_{(2)}\right.$

$\displaystyle
\frac{\Delta L}{L}=\frac{\Delta \ell_1}{L}+\frac{\Delta \ell_2}{L}=\frac{\ell_1}{L} \frac{\Delta \ell_1}{\ell_1}+\frac{\ell_2}{L} \frac{\Delta \ell_2}{\ell_2}= \gamma_1 \varepsilon_{(1)}+\gamma_2 \varepsilon_{(2)}=\frac{\gamma_1}{E_{(1)}} E_{(1)} \varepsilon_{(1)}+\frac{\gamma_2}{E_{(2)}} E_{(2)} \varepsilon_{(2)}=\left(\frac{\gamma_1}{E_{(1)}}+\frac{\gamma_2}{E_{(2)}}\right) p \Rightarrow \frac{\gamma_1}{E_{(1)}}+\frac{\gamma_2}{E_{(2)}}=\frac{\Delta L}{p L} $

$\displaystyle
\left\langle\varepsilon_{11}\right\rangle=\frac{1}{V} \iint_V \varepsilon(x) d V=\frac{1}{V} \int_{V_1} \varepsilon_{(1)} d V+\frac{1}{V} \int_{V_2} \varepsilon_{(2)} d V=\frac{V_1}{V} \varepsilon_{(1)}+\frac{V_2}{V} \varepsilon_{(2)}=\gamma_1 \varepsilon_{(1)}+\gamma_2 \varepsilon_{(2)} \Rightarrow\left\langle\varepsilon_{11}\right\rangle=\frac{\Delta L}{L} \Rightarrow \frac{1}{E^{\mathrm{R}}} \equiv \frac{\left\langle\varepsilon_{11}\right\rangle}{p}=\frac{\Delta L}{p L}=\frac{\gamma_1}{E_{(1)}}+\frac{\gamma_2}{E_{(2)}}$
 

 
$
\displaystyle
\sigma_{11}^{(1)}=\sigma_{11}^{(2)}=\ldots=\sigma_{11}^{(N)}=p,\left\langle\varepsilon_{11}\right\rangle=\sum_{i=1}^N \gamma_i \varepsilon_{(i)}, 
$


\textbf{Обратная формула смесей:}
$\sum_{i=1}^N \frac{\gamma_i}{E_{(i)}}=\frac{1}{E^{\mathrm{R}}}=\frac{S \cdot \Delta L}{P \cdot L}$


\section{Эффективные модули слоисто-волокнистого композита в плоско-напряженном состоянии в главных осях. Формула Акасаки для эффективного трансверсального модуля.}

\begin{figure}[h!]
  \centering
  \includegraphics[width=0.9\textwidth]{images/21.1.jpg}
\end{figure}

$$
\left(\sigma_L, \sigma_T, \sigma_{L T}\right)=\left(\sigma_{11}, \sigma_{22}, \sigma_{12}\right) ;\left(\varepsilon_L, \varepsilon_T, \varepsilon_{L T}\right)=\left(\varepsilon_{11}, \varepsilon_{22}, 2 \varepsilon_{12}\right)
$$

Эффективные плоские модули определим при следующих предположениях: 1) фазы являются линейно-упругими и изотропными; 2) нет массовых сил; 3) тело находится в состоянии статического равновесия; 4) состояние является плоско-напряженным и главные оси тензора напряжений сонаправлены со сторонами пластины.


($E_L$ эффективный продольный модуль Юнга, $E_T$ эффективный поперечный модуль Юнга)

При указанных условиях проверим гипотезу, что модель композита является ортотропной с главными осями \textit{LT}.

$
\displaystyle
\left(\begin{array}{l}
\varepsilon_1^{\prime} \\
\varepsilon_2^{\prime} \\
\varepsilon_6^{\prime}
\end{array}\right)=\left(\begin{array}{ccc}
1 / E^{\prime} & -v^{\prime} / E^{\prime} & 0 \\
-v^{\prime} / E^{\prime} & 1 / E^{\prime} & 0 \\
0 & 0 & 1 / G^{\prime}
\end{array}\right)\left(\begin{array}{l}
\sigma_1^{\prime} \\
\sigma_1^{\prime} \\
\sigma_6^{\prime}
\end{array}\right),
\left(\begin{array}{l}
\varepsilon_1^{\prime \prime} \\
\varepsilon_2^{\prime \prime} \\
\varepsilon_6^{\prime \prime}
\end{array}\right)=\left(\begin{array}{ccc}
1 / E^{\prime \prime} & -v^{\prime \prime} / E^{\prime \prime} & 0 \\
-v^{\prime \prime} / E^{\prime \prime} & 1 / E^{\prime \prime} & 0 \\
0 & 0 & 1 / G^{\prime \prime}
\end{array}\right)\left(\begin{array}{l}
\sigma_1^{\prime \prime} \\
\sigma_1^{\prime \prime} \\
\sigma_6^{\prime \prime}
\end{array}\right)
\textbf{???} 
\left(\begin{array}{l}
\varepsilon_L \\
\varepsilon_T \\
\varepsilon_{L T}
\end{array}\right)=\left(\begin{array}{ccc}
1 / E_L & -v_{T L} / E_T & 0 \\
-v_{L T} / E_L & 1 / E_T & 0 \\
0 & 0 & 1 / G_{L T}
\end{array}\right)\left(\begin{array}{l}
\sigma_L \\
\sigma_T \\
\sigma_{L T}
\end{array}\right) \textbf{???}$


Примем принцип суперпозиции двух одномерных задач. Это позволит нам определить средние по объёму значения диагональных компонент тензоров напряжений и деформаций. Средние значения совпадают со значениями в модели композита:
$$
\begin{aligned}
& \varepsilon_1^{\prime}=\varepsilon_1^{\prime \prime}=\left\langle\varepsilon_{11}\right\rangle \equiv \varepsilon_L ; \quad \gamma^{\prime} \sigma_1^{\prime}+\gamma^{\prime \prime} \sigma_1^{\prime \prime}=\left\langle\sigma_{11}\right\rangle \equiv \sigma_L \\
& \sigma_2^{\prime}=\sigma_2^{\prime \prime}=\left\langle\sigma_{22}\right\rangle \equiv \sigma_T ; \quad \gamma^{\prime} \varepsilon_2^{\prime}+\gamma^{\prime \prime} \varepsilon_2^{\prime \prime}=\left\langle\varepsilon_{11}\right\rangle \equiv \varepsilon_T
\end{aligned}
$$

Запишем получившуюся систему уравнений. Из этих равенств надо исключить: $\varepsilon_1^{\prime}, \varepsilon_2^{\prime}, \sigma_1^{\prime}, \sigma_2^{\prime}, \varepsilon_1^{\prime \prime}, \varepsilon_2^{\prime \prime}, \sigma_1^{\prime \prime}, \sigma_2^{\prime \prime}$.

$$\left\{\begin{array} { l } 
{ \varepsilon _ { 1 } ^ { \prime } = \varepsilon _ { 1 } ^ { \prime \prime } = \varepsilon _ { L } } \\
{ \gamma ^ { \prime } \sigma _ { 1 } ^ { \prime } + \gamma ^ { \prime \prime } \sigma _ { 1 } ^ { \prime \prime } = \sigma _ { L } } \\
{ \sigma _ { 2 } ^ { \prime } = \sigma _ { 2 } ^ { \prime \prime } = \sigma _ { T } } \\
{ \gamma ^ { \prime } \varepsilon _ { 2 } ^ { \prime } + \gamma ^ { \prime \prime } \varepsilon _ { 2 } ^ { \prime \prime } = \varepsilon _ { T } }
\end{array} \left\{\begin{array}{l}
E^{\prime} \varepsilon_1^{\prime}=\sigma_1^{\prime}-v^{\prime} \sigma_2^{\prime} \\
E^{\prime} \varepsilon_2^{\prime}=\sigma_2^{\prime}-v^{\prime} \sigma_1^{\prime} \\
E^{\prime \prime} \varepsilon_1^{\prime \prime}=\sigma_1^{\prime \prime}-v^{\prime \prime} \sigma_2^{\prime \prime} \\
E^{\prime \prime} \varepsilon_2^{\prime \prime}=\sigma_2^{\prime \prime}-v^{\prime \prime} \sigma_1^{\prime \prime}
\end{array}\right.\right.$$

$$
\left\{\begin{array}{l}
E^{\prime} \varepsilon_L=\sigma_1^{\prime}-v^{\prime} \sigma_T \mid \cdot \gamma^{\prime} \\
E^{\prime \prime} \varepsilon_L=\sigma_1^{\prime \prime}-v^{\prime \prime} \sigma_T \mid \cdot \gamma^{\prime \prime}
\end{array}\right. \Rightarrow$$

$$\left(\gamma^{\prime} E^{\prime}+\gamma^{\prime \prime} E^{\prime \prime}\right) \varepsilon_L \stackrel{{\gamma}^{\prime} \sigma_1^{\prime}+\gamma^{\prime \prime} \sigma_{1}^{\prime \prime}=\sigma_L}{=}
\quad \underbrace{\left(\gamma^{\prime} \sigma_1^{\prime}+\gamma^{\prime \prime} \sigma_1^{\prime \prime}\right)}\limits_{\sigma_L}-\left(\gamma^{\prime} v^{\prime}+\gamma^{\prime \prime} v^{\prime \prime}\right) \sigma_T \Rightarrow
$$
$$
\varepsilon_L=\frac{1}{\gamma^{\prime} E^{\prime}+\gamma^{\prime \prime} E^{\prime \prime}} \sigma_L-\frac{\gamma^{\prime} v^{\prime}+\gamma^{\prime \prime} v^{\prime \prime}}{\gamma^{\prime} E^{\prime}+\gamma^{\prime \prime} E^{\prime \prime}} \sigma_T \quad \Rightarrow \quad E_L=\gamma^{\prime} E^{\prime}+\gamma^{\prime \prime} E^{\prime \prime}, \quad \frac{v_{T L}}{E_T}=\frac{\gamma^{\prime} v^{\prime}+\gamma^{\prime \prime} v^{\prime \prime}}{\gamma^{\prime} E^{\prime}+\gamma^{\prime \prime} E^{\prime \prime}}
$$
$$
\left\{\begin{array}{l}
\varepsilon_2^{\prime}=1 / E^{\prime}\left(\sigma_T-v^{\prime} \sigma_1^{\prime}\right) \mid \cdot \gamma^{\prime} \\
\varepsilon_2^{\prime \prime}=1 / E^{\prime \prime}\left(\sigma_T-v^{\prime \prime} \sigma_1^{\prime \prime}\right) \mid \cdot \gamma^{\prime \prime}
\end{array}\right.$$
$$
\varepsilon_T=\left(\frac{\gamma^{\prime}}{E^{\prime}}+\frac{\gamma^{\prime \prime}}{E^{\prime \prime}}\right) \sigma_T-\frac{\gamma^{\prime} v^{\prime}}{E^{\prime}} \underbrace{\left(E^{\prime} \varepsilon_L+v^{\prime} \sigma_T\right)}_{\sigma_1^{\prime}}-\frac{\gamma^{\prime \prime} v^{\prime \prime}}{E^{\prime \prime}} \underbrace{\left(E^{\prime \prime} \varepsilon_L+v^{\prime \prime} \sigma_T\right)}_{\sigma_1^{\prime\prime}}
$$
$$
\begin{aligned}
& \varepsilon_T=\left(\frac{\gamma^{\prime}}{E^{\prime}}+\frac{\gamma^{\prime \prime}}{E^{\prime \prime}}-\frac{\gamma^{\prime}\left(v^{\prime}\right)^2}{E^{\prime}}-\frac{\gamma^{\prime \prime}\left(v^{\prime \prime}\right)^2}{E^{\prime \prime}}\right) \sigma_T-\left(\gamma^{\prime} v^{\prime}+\gamma^{\prime \prime} v^{\prime \prime}\right) \underbrace{\left(\frac{\sigma_L}{E_L}-\frac{\gamma^{\prime} v^{\prime}+\gamma^{\prime \prime} v^{\prime \prime}}{E_L} \sigma_T\right)} \limits_{\varepsilon_L} \\
\end{aligned}
$$
$$
\varepsilon_T=-\frac{\overbrace{\gamma^{\prime} v^{\prime}+\gamma^{\prime \prime} v^{\prime \prime}}^{v_{L T}}}{E_L} \sigma_L+\left(\frac{\gamma^{\prime}}{E^{\prime}}+\frac{\gamma^{\prime \prime}}{E^{\prime \prime}}-\frac{\gamma^{\prime}\left(v^{\prime}\right)^2}{E^{\prime}}-\frac{\gamma^{\prime \prime}\left(v^{\prime \prime}\right)^2}{E^{\prime \prime}}+\frac{\left(\gamma^{\prime} v^{\prime}+\gamma^{\prime \prime} v^{\prime \prime}\right)^2}{\underbrace{\gamma^{\prime} E^{\prime}+\gamma^{\prime \prime} E^{\prime \prime}}_{E_L}}\right) \sigma_T 
$$
$$
\Rightarrow 
(\varepsilon_T=-\frac{v_{LT}}{E_L}\sigma_L+\frac{1}{E_T}\sigma_T) \Rightarrow
$$
$$
\displaystyle
\begin{aligned}
& \frac{1}{E_T}=\left(\frac{\gamma^{\prime}}{E^{\prime}}+\frac{\gamma^{\prime \prime}}{E^{\prime \prime}}\right)-\frac{\gamma^{\prime}\left(v^{\prime}\right)^2}{E^{\prime}}-\frac{\gamma^{\prime \prime}\left(v^{\prime \prime}\right)^2}{E^{\prime \prime}}+\frac{\left(\gamma^{\prime} v^{\prime}+\gamma^{\prime \prime} v^{\prime \prime}\right)^2}{\gamma^{\prime} E^{\prime}+\gamma^{\prime \prime} E^{\prime \prime}}=\left(\frac{\gamma^{\prime}}{E^{\prime}}+\frac{\gamma^{\prime \prime}}{E^{\prime \prime}}\right)-\frac{*}{E^{\prime} E^{\prime \prime} E_L} \Rightarrow \\
& *=\gamma^{\prime}\left(v^{\prime}\right)^2 E^{\prime \prime}\left(\underline{\gamma^{\prime} E^{\prime}}+\gamma^{\prime \prime} E^{\prime \prime}\right)+\gamma^{\prime \prime}\left(v^{\prime \prime}\right)^2 E^{\prime}\left(\gamma^{\prime} E^{\prime}+\underline{\gamma^{\prime \prime} E^{\prime \prime}}\right)-\underline{E^{\prime} E^{\prime \prime}\left(\gamma^{\prime} v^{\prime}+\gamma^{\prime \prime} v^{\prime \prime}\right)^2}= \\
& \left(\left(\gamma^{\prime} v^{\prime}\right)^2+\left(\gamma^{\prime \prime} v^{\prime \prime}\right)^2-\left(\gamma^{\prime} v^{\prime}+\gamma^{\prime \prime} v^{\prime \prime}\right)^2\right) E^{\prime} E^{\prime \prime}+\gamma^{\prime} \gamma^{\prime \prime}\left(\left(v^{\prime} E^{\prime \prime}\right)^2+\left(v^{\prime \prime} E^{\prime}\right)^2\right)=-2 \gamma^{\prime} v^{\prime} \gamma^{\prime \prime} v^{\prime \prime} E^{\prime} E^{\prime \prime}+ \\
& +\gamma^{\prime} \gamma^{\prime \prime}\left(\left(v^{\prime} E^{\prime \prime}\right)^2+\left(v^{\prime \prime} E^{\prime}\right)^2\right)=\gamma^{\prime} \gamma^{\prime \prime}\left(v^{\prime} E^{\prime \prime}-v^{\prime \prime} E^{\prime}\right)^2 \Rightarrow 
\\& \textbf{Формула Акасаки:} \frac{1}{E_T}=\left(\frac{\gamma^{\prime}}{E^{\prime}}+\frac{\gamma^{\prime \prime}}{E^{\prime \prime}}\right)-\gamma^{\prime} \gamma^{\prime \prime} \frac{\left(v^{\prime} / E^{\prime}-v^{\prime \prime} / E^{\prime \prime}\right)^2}{\gamma^{\prime} / E^{\prime \prime}+\gamma^{\prime \prime} / E^{\prime}} \\
& v_{L T}=\gamma^{\prime} v^{\prime}+\gamma^{\prime \prime} v^{\prime \prime} ; \quad E_L=\gamma^{\prime} E^{\prime}+\gamma^{\prime \prime} E^{\prime \prime} ; \quad v_{T L}=\frac{E_T}{E_L} v_{L T} ; \quad \\& \frac{1}{G_{L T}}=\frac{\gamma^{\prime}}{G^{\prime}}+\frac{\gamma^{\prime \prime}}{G^{\prime \prime}} \Leftarrow \sigma_{12}^{\prime}=\sigma_{12}^{\prime \prime}=\sigma_{L T} \\
&
\end{aligned}
$$
