\section{Колебания атомов одномерной двухатомной решетки. Акустические и оптические фононы, температура Дебая. Колебания атомов трехмерной 
решетки (зоны Бриллюэна, законы дисперсии и поверхности постоянной частоты).}

Рассмотрим одномерную решетку Бравэ с двумя ионами в элементарной ячейке, равновесные положения которых есть $na$ и $na+d$. Считаем, что оба иона идентичны, но что $d \leq a/2$, поэтому сила между соседними ионами зависит от того, равно ли это расстояние между ними $d$ или же $a-d$. 

\begin{figure}[h!]
    \includegraphics[width=\textwidth]{phys_1_1.png}
\end{figure}

Предположим, что взаимодействуют ближайшие соседи, причем сила взаимодействия больше для пар с расстоянием $d$, чем для пар с расстоянием $a-d$. Гармоническую потенциальную энергию можно представить в виде:

\begin{equation}
    U^{\text {harm }}=\frac{K}{2} \sum\left[u_1(n a)-u_2(n a)\right]^2+\frac{G}{2} \sum\left[u_2(n a)-u_1([n+1] a)\right]^2
\end{equation}

\noindent где через $u_1(na)$ обозначено смещение иона, совершающего колебания вблизи узла na, а через $u_2(na)$ --- смещение иона, колеблющегося вблизи узла na+d. Положим $K>=G$. Уравнения движения 

\begin{equation}
    \begin{aligned}
    M \ddot{u}_1(n a) & =-\frac{\partial U^{h a r m}}{\partial u_1(n a)}=-K\left[u_1(n a)-u_2(n a)\right]-G\left[u_1(n a)-u_2([n-1] a)\right] \\
    M \ddot{u_2}(n a) & =-\frac{\partial U^{h a r m}}{\partial u_2(n a)}=-K\left[u_2(n a)-u_1(n a)\right]-G\left[u_2(n a)-u_1([n+1] a)\right]
    \end{aligned}
\end{equation}


Ищем решение, представляющее собой волну с частотой $\omega$ и волновым вектором $k$: $u_1(n a)=\varepsilon_1 e^{i(k n a-\omega \mathrm{t})}$ и $u_2(n a)=\varepsilon_2 e^{i(k n a-\omega t)}$. Здесь $\epsilon_1$ и $\epsilon_2$ --- требующие определения постоянные, отношение которых дает относительные амплитуды и фазу колебаний ионов в каждой элементарной ячейке. Как и в моноатомном случае, периодическое граничное условие Борна-Кармана приводит к $N$ неэквивалентным значениям $k$. 

\begin{figure}[h!]
    \begin{center}
        \includegraphics[width=0.7\textwidth]{phys_1_2}
    \end{center}
\end{figure}


\begin{equation}
    k=\frac{1\pi}{a} \frac{n}{N}
\end{equation}

Граничное условие (формулировка для линейной цепочки): соединяем два противоположных конца цепочки с помощью одной такой же <<пружинки>>, как и те, которые соединяют ионы внутри цепочки. 

Из полученных выше выражений получим 

\begin{equation}
    \begin{gathered}
    {\left[M \omega^2-(K+G)\right] \varepsilon_1+\left(K+G e^{-i k a}\right) \varepsilon_2=0} \\
    {\left[M \omega^2-(K+G)\right] \varepsilon_2+\left(K+G e^{i k a}\right) \varepsilon_1=0}
    \end{gathered}
\end{equation}

Данная пара уравнений имеет решение, если обращается в нуль детерминант, составленный из их коэффициентов, т.е. если выполняется условие 

\begin{equation}
    \left[M \omega^2-(K+G)\right]^2=\left|K+G e^{-i k a}\right|^2=K^2+G^2+2 K G \cos k a
\end{equation}

Данные уравнение выполняется для двух положительных значений $\omega$, для которых 

\begin{equation}
    \omega^2=\frac{K+G}{M} \pm \frac{1}{M} \sqrt{K^2+G^2+2 K G \cos k a}
\end{equation}
Причем $\displaystyle \frac{\varepsilon_1}{\varepsilon_2}=\mp \frac{K+G e^{-i k a}}{\left|K+G e^{-i k a}\right|}$

Для каждого из $N$ значений $k$ имеется, таким образом, 2 решения, что дает в целом $2N$ нормальных мод, ка и должно быть при $2N$ степенях свободы (по 2 иона в каждой из N элементарных ячеек). Две кривые зависимости $\omega$ от $k$ носят название 2 ветвей закона дисперсии.

Частота первой ветви $\omega$ пропорциональна $k$ при малых $k$, а на краях зоны Бриллюэна кривая становится почти горизонтальной. Эту ветвь называют акустической, так как ее закон дисперсии при малых $k$ имеет форму $\omega=ck$, характерную для звуковых волн. Вторая ветвь начинается при $k=0$ от значения  $\omega = \sqrt{2(K+G)/M}$ и опускается вниз с ростом $k$, достигая значения $\sqrt{2K/M}$ на границе зоны. Ее называют оптической ветвью, поскольку длинноволновые оптические моды в ионных кристаллах могут взаимодействовать с электромагнитным излучением. 

\textbf{Модель Дебая}

\begin{itemize}
    \item Колебания решетки рассматриваются как фононный газ
    \item Закон дисперсии фононов предполагается изотропным и линейным, то есть $\omega=u_sk$ ($u_s = const$)
    \item Первая зона Бриллюэна, которая представляет собой многогранник в пространстве волновых векторов, заменяется на шар того же объема. Его радиус $q_0$ находится из условия равенства объемов: $V_{\text{з.Бp.}}=\frac{4 \pi q_0^3}{3} \Rightarrow q_0=\left(\frac{6 \pi^2}{V_{\text {яч. }}}\right)^{1 / 3}$
    \item $0<\omega<\omega_{max}=\omega_D$
\end{itemize}

Для сплошной среды число состояний на элемент $dk$: 
\begin{equation}
    \frac{V 4 \pi k^3 d k}{(2 \pi)^3}
\end{equation}

При $\mathrm{T} \gg \Theta_{\mathrm{D}}$, где $\Theta_{\mathrm{D}}=\hbar \omega_{\mathrm{D}}$, температура Дебая, все осцилляторы допускают классическое описание.

$\mathrm{T} \ll \Theta_{\mathrm{D}}$, всегда найдутся осцилляторы, чье движение можно описывать классическими формулами, но остальные требуется описать квантово. 

\section{Спектральная плотность фононов. Модели Эйнштейна и Дебая. 
Теплоемкость кристаллической решетки.}

\textbf{Спектральная плотность колебаний решетки} (фононов) или плотность фононных состояний: $v(\omega)=\frac{1}{V} \frac{d N \omega}{d \omega}$, где $dN_\omega$ --- число фононных мод с частотами, лежащими в интервале ($\omega$; $\omega + d \omega$).

Спектральная плотность удовлетворяет условию нормировки $\displaystyle \int_0^{\infty} v(\omega) d \omega=\frac{3 N}{V_{\text {ячейки }}}$ (здесь $N$
--- число атомов в элементарной ячейке $3D$ кристалла, то есть $3N$ --- число степеней свободы). Зная законы дисперсии фононных ветвей, можно определить плотность фононных состояний как $\displaystyle v(\omega)=\sum_p \int \frac{d^3 \bar{k}}{(2 \pi)^2} \delta\left(\omega-\omega_p(\bar{k})\right)$, где $\delta(\omega)$ --- дельта-функция Дирака. Интеграл взят по первой зоне Бриллюэна. 

\begin{equation}
    \delta(\omega)=\left\{\begin{array}{c}
    +\infty, x=0 \\
    0, x \neq 0
    \end{array}\right.
\end{equation}

\textbf{Модель Эйнштейна}

\begin{itemize}
    \item Атомы в решетке ведут себя как $3N_a$ гармонических осцилляторов, не
    взаимодействуя друг с другом;
    Частота колебаний всех осцилляторов одинакова: $\omega_p(\bar{k})=\omega_E=const$. Модель качественно описывает оптические ветви, для которых $\omega_{\min}<\omega_p(\bar{k})<\omega_{\max}$, причем $\omega_{\min}$ и $\omega_{\max}$ --- величины одного порядка

    Для акустических ветвей модель Эйнштейна в области низких температур неприменима

    \item $\hbar \omega_p(\bar{k}) \geq T$. Так как $\omega_E=const$, $\nu(\omega)=3N\delta(\omega-\omega_E)$
\end{itemize}

\textit{Теплоемкость:}

Вычислим среднюю энергию осциллятора с частотой $\omega_i$ и полную энергию набора осцилляторов, так как фононы подчиняются статистике Бозе-Эйнштейна. 

\begin{equation}
    \bar{\varepsilon}_l=\frac{\hbar \omega_i}{e^{\hbar \omega_i / T}-1} ; \quad E_{\text {полн }}=\sum_{i=1}^{3 N} \frac{\hbar \omega_i}{e^{\hbar \omega_i / T}-1}
\end{equation}

Среднее число заполнения: $\langle n\rangle=\frac{1}{e^{\frac{\hbar \omega_i}{T}}-1}$

\begin{equation}
    E_{\text {полн }}=\int_0^{\omega_{\max }} \frac{\hbar \omega_i}{e^{\frac{\hbar \omega_i}{T}}-1} v(\omega) \mathrm{d} \omega = \int_0^{\omega_{\max }=\omega_E} \frac{\hbar \omega_i * 3 N}{e^{\frac{\hbar \omega_i}{T}}-1} \delta\left(\omega-\omega_E\right) \mathrm{d} \omega= \frac{\hbar \omega_E * 3 N}{e^{\hbar \omega_i / T}-1}
\end{equation}



\begin{equation}
    C_v=\frac{\partial E}{\partial T}=\frac{\hbar \omega_E \cdot 3 N-e^{\frac{\hbar \omega_E}{T}} \cdot \frac{\hbar \omega_E}{T^2}}{\left(e^{\frac{\hbar \omega_E}{T}}-1\right)^2}
\end{equation}


\textbf{Модель Дебая}

\begin{itemize}
    \item Колебания решетки рассматриваются как фононный газ.
    \item Закон дисперсии фононов предполагается изотропным и линейным, то есть $\omega = u_{s}k$, $(u_s=const)$.
    \item Первая зона Бриллюэна, которая представляет собой многогранник в пространстве волновых векторов, заменяется на шар того же объема. Его радиус $q_0$ находится из условия равенства объемов: $V_{\text{з.Бp.}}=\frac{4 \pi q_0^3}{3} \Rightarrow q_0=\left(\frac{6 \pi^2}{V_{\text {яч. }}}\right)^{1 / 3}$
    \item $0<\omega<\omega_{max}=\omega_D$
\end{itemize}

Для сплошной среды число состояний на элемент $dk=\frac{V 4 \pi k^3 d k}{(2 \pi)^3}$. Усредним $\omega=\overline{u_s}k$; $\frac{3}{\overline{u_{s}}^3} = \frac{1}{\overline{u_{e}}^3}+\frac{2}{\overline{u_{t}}^3}$, ($\omega_t=\overline{u_t}k$, $\omega_c=\overline{u_c}$ --- продольные и поперечные волны).
Тогда $\displaystyle \frac{V 4 \pi k^3 d k}{(2 \pi)^3} \rightarrow \frac{V 3 \omega^2 d \omega}{2 \pi^2 \bar{u}_s{ }^3} \rightarrow \nu(\omega)=\left\{\begin{array}{c}\frac{3}{2 \pi^2} \cdot \frac{\omega^2}{\overline{u_s}^3} \omega \leqslant \omega_D \\ 0, \quad \omega>\omega_D\end{array}\right.$

\textit{Теплоёмкость}

\begin{equation}
    C_v=\int_0^{\omega_D} \frac{(-1)  e^{\hbar \omega / T} \cdot \left(\frac{1}{T^2}\right) *(h \omega)^2}{\left(e^{\hbar \omega / T}-1\right)^2} v(\omega) d \omega,
\end{equation}

Новая переменная $\displaystyle \left\{\begin{array}{c}\varkappa=\frac{\hbar \omega}{T} \quad \mid \omega \rightarrow 0 \; \varkappa \rightarrow 0 \\ d \varkappa=\frac{\hbar}{T} d \omega \quad \mid \omega \rightarrow \omega_D \; \varkappa \rightarrow \frac{\Theta_D}{T}\end{array}\left(\Theta_D=\omega_D \hbar\right)\right.$

Тогда 
\begin{equation}
    \begin{aligned} & \left.C_v=\int_0^{\theta_D / T} \frac{e^\chi \chi^2}{\left(e^x-1\right)^2} * \frac{3 T^2 \varkappa^2}{2 \pi^2 \overline{u}_s{ }^2 \hbar^2} * \frac{T}{\hbar} d \varkappa=\frac{3 T^3}{2 \pi^2 \overline{u}^3 \mathrm{\hbar}^3} \int_0^{\theta_D / T} \frac{e^\chi \varkappa^4}{\left(e^x-1\right)^2}=\left( \begin{array}{l} \omega_D=\overline{u}_s k_D \\ \overline{u}_s=b\omega_D / k_D  \end{array}\right. \right) = \\ & =\frac{3}{2 \pi^2} *\left(\frac{T}{\theta_D}\right)^3 k_D{ }^3\left\{\begin{array}{c}T \ll \theta_D \Rightarrow \theta_D / T \rightarrow \infty \Rightarrow \int_0^{\infty} \varkappa^2 d \varkappa=\frac{4 \pi^2}{15} \\ T \gg \theta_D \Rightarrow \int_0^{\theta_D / T} e^{\varkappa} \varkappa^2 d \varkappa \approx \int_0^{\theta_D / T} \varkappa^2 d \varkappa=\frac{1}{3}\left(\frac{\theta_D}{T}\right)^3\end{array}\right. \\ & \end{aligned}
    \end{equation}
    
\begin{equation}
    C_v=\left\{\begin{array}{c}T \ll \theta_D \Rightarrow \frac{3}{2 \pi}\left(\frac{T}{\theta_D}\right)^3 k_D{ }^3 \frac{4}{15} \pi^2=\frac{2 \pi}{5}\left(\frac{T}{\theta_D}\right)^3 k_D{ }^3 \\ T \gg \theta_D \Rightarrow \frac{k_D{ }^3}{2 \pi} \Rightarrow\left\langle\int_0^{\omega_D} v(\omega) d \omega=3 N\right\rangle \Rightarrow \frac{3 N \pi^2}{2 \pi^2}=9 N\end{array}\right. 
\end{equation}


\section{Приближение почти свободных электронов. Поверхности Ферми металлов. Метод Харрисона.}

\begin{figure}[h!]
    \centering
    \includegraphics[width=0.8\textwidth]{phys_1_3.png}
\end{figure}

Применим для металлов; Электроны ведут себя почти как свободные; $U(r)$ мало 
(значительно меньше кинетической энергии); Используются методы теории возмущений 

Рассматривается движение электронов в линейной цепочке прямоугольных потенциальных ям шириной $a$, отделённых барьерами толщиной $b$ и высотой U0. Длина цепочки $L$, период $a+b$.
Решение уравнения Шрёдингера $\frac{\partial^2 \psi}{\partial x^2}+\frac{2 m}{\hbar^2}(E-U) \psi=0$ разбивается на две части

\begin{enumerate}
    \item $\mathrm{U}=0$, волновая функция представляется в виде $\psi_1(x)=A e^{i \alpha x}+B e^{-i \alpha x}$ (первое слагаемое - прямая волна, второе - отраженная от барьера)
    \item $\mathrm{U}=\mathrm{U}_0$, волновая функция: $\psi_2(x)=C e^{\beta x}+B e^{-\beta x}$.
    При этом $\alpha=\sqrt{\frac{2 m e}{\hbar^2}} \quad; \beta=\sqrt{\frac{2 m e\left(U_0-E\right)}{\hbar^2}}$.
\end{enumerate}

Подставляя в одномерную функцию Блоха $\psi(x)=U(x) \exp (i k x)$
$$
U_1(x)=A e^{(\alpha-i k) x}+B e^{-(\alpha+i k) x} ; \quad U_2(x)=C e^{(\beta-i k) x}+B e^{-(\beta+i k) x} .
$$
Oпределим A, B, C, D из непрерывности $U(x)$ и $U^{\prime}(x)$ в местах скачка потенциала. Т.е. для $\mathrm{x}=0$
$$
U_1(0)=U_2(0) ;\left(\frac{\partial U_1}{\partial x}\right)_{x=0}=\left(\frac{\partial U_2}{\partial x}\right)_{x=0}
$$ 

Также воспользуемся периодичностью $U(x): U_1(a)=U_2(-b) ;\left(\frac{\partial U_1}{\partial x}\right)_{x=a}=\left(\frac{\partial U_2}{\partial x}\right)_{x=-b}$

\begin{figure}[h!]
    \centering
    \includegraphics{phys_2_1}
\end{figure}


Подставляя в уравнения для $U_i(x)$, получим
$$
\frac{\beta^2-\alpha^2}{2 \alpha \beta} \mathrm{sh} (\beta b) \sin (\alpha a)+\mathrm{ch}(\beta b) \cos (\alpha a)=\cos k(a+b)
$$

Рассмотрим предельный случай $b \rightarrow 0, U_0 \rightarrow \infty$ при $b U_0= const =\frac{\hbar^2}{m a} P$. $\displaystyle P=\frac{m a}{h^2 b v_0}$ ($\mathrm{P}$ --- прозрачность барьера). Тогда: $\displaystyle P \frac{\sin (\alpha a)}{a d}+\cos (\alpha a)=\cos (k a) \text {. }$

Так как $-1<\cos<1$, левая часть уравнения от -1 до 1. Это и определяет область разрешённых состояний.

Таким образом, при построении зон Бриллюэна в приближении ПСЭ рассматривается чередование запрещенных и разрешенных зон. При этом

\begin{itemize}
    \item на границе $\displaystyle \frac{\partial E}{\partial k} =0$
    
    \includegraphics[width=0.4\textwidth]{phys_2_2} \includegraphics[width=0.2\textwidth]{phys_2_5}
    
    \item размеры зон одинаковы, поэтому число состояний в зонах конечно и одинаково
    \item в расширенной схеме каждая зона повторяется при трансляции на $\bar{a}$ : схема периодична, первая зона --- приведенная зонная схема 
\end{itemize}

\textbf{Заполнение зон Бриллюэна электронами в металлах:}

\begin{enumerate}
    \item Двухкомпонентная квадратичная решётка: 1-я зона --- куб (квадрат), 2-я зона --- призма (4 треугольника)
    \item $E_{\min} ^2 < E_{\max} ^1$ --- металлы Mg, Be (двухзонные). Проводимость: электронная по второй зоне, дырочная по первой (вторая зона заполняется после первой).
\end{enumerate}

\begin{figure}[h!]
    \centering
    \includegraphics[width=0.5\textwidth]{phys_2_3}
\end{figure}



\textbf{Поверхность Ферми} --- поверхность с энергией Ферми, отделяющая пустые состояния от занятых.

\begin{itemize}
    \item Металлы 1-й группы. Na, Li, K, Au, Ag, Cu: первая зона заполнена на половину: $\displaystyle R^2 \pi=\frac{\pi^2}{2 a^2} \Rightarrow R^2=\frac{\pi}{2 a^2} \Rightarrow R=\frac{\pi}{a} \frac{1}{\sqrt{2 \pi}}$, $\displaystyle \mathrm{K}_{\mathrm{F}}=\frac{\pi}{a} \frac{1}{\sqrt{2 \pi}}$
    \item Металлы 2-й группы. Mg, Be, Zn, Cd: $R^2 \pi= \frac{4\pi^2}{a^2} \Rightarrow R^2=\frac{4 \pi}{a^2} \Rightarrow K_F=\frac{2 \sqrt{\pi}}{a}$

\end{itemize}

\begin{figure}[h!]
    \centering
    \includegraphics[width=0.2\textwidth]{phys_2_4}
\end{figure}

\textbf{Метод Харрисона построения поверхностей Ферми металлов}

\begin{figure}[h!]
    \centering
    \includegraphics[width=0.4\textwidth]{phys_2_6}
\end{figure}


\begin{enumerate}
    \item Построение мозаики зон Бриллюэна
    \item Расчёт размеров и построение сфер Ферми с центрами из каждого элемента мозаики
    \item Корректировка границ сфер (пересечение с границами зон под прямым углом)
    \item Классификация полученных объёмов по зонам Бриллюэна (вогнутые поверхности --- к сферам $k+1$-й зоне (дырочные); выпуклые --- электронные в $k$-й зоне)
\end{enumerate}


\section{Предположения и следствия метода сильной связи. Заполнение энергетических зон электронами (металлы, полупроводники и диэлектрики, полуметаллы).}

\begin{figure}[h!]
    \centering
    \includegraphics[width=0.8\textwidth]{phys_1_3.png}
\end{figure}

\textbf{Метод сильной связи:} электрон в твёрдом теле ведёт себя так же, как в изолированном атоме.

\begin{figure}[h!]
    \centering
    \includegraphics{phys_3_1}
\end{figure}

$\displaystyle -n=1,2,3 \ldots\left(H: E=-\frac{R_y}{n^{2}}\right)$ ($E=E(n, \;l)$ в отсутствие полей)

$\displaystyle -l=0,1,2 \ldots \mathrm{n}-1\left(M_{e}=\hbar \sqrt{l(l+1}\right)$

$\displaystyle -m=-l \ldots 0 \ldots l$ ($(2 l+1)$ значений)

$\displaystyle -s= \pm 1(1 / 2)$ (s-подуровень --- вырожден, $p-$ --- трёхкратно вырожден)

Рассмотрим на примере атома натрия.

При переходе от набора изолированных атомов к кристаллу 3s-подуровень оказывается выше барьера. Поэтому движение электронов этого подуровня по всему кристаллу не ограничено. К снижению потенциальных барьеров привело перекрытие волновых функций.

\begin{wrapfigure}{R}{0.4\textwidth}
    \includegraphics[width=0.4\textwidth]{phys_3_2}
\end{wrapfigure}

Уравнение Шрёдингера $\quad\left\{-\frac{\hbar^{2}}{2 m} \nabla^{2}+U(\bar{r})\right\} \psi(\bar{r})=E \psi(\bar{r}) \quad$ для изолированного атома с номером $g$ будет принимать вид $\left\{-\frac{\hbar^{2}}{2 m} \nabla^{2}+U_g(\bar{r})\right\} \psi_{g}\left(\bar{r}-\overrightarrow{R_{g}}\right)=E_{a} \psi_{g}\left(\bar{r}-\overrightarrow{R_{g}}\right)$ ($\overrightarrow{R_{g}}$ --- радиус-вектор узла решётки).

В качестве предполагаемого вида $\psi(\bar{r})$ выберем линейную комбинацию $\sum \limits_{g} a_{g} \psi_{g}$.

Тогда $\psi(\bar{r})=\sum \limits_{g} a_{g} \psi_{g}=\sum \limits_{g} e^{i k \overrightarrow{R_{g}}} \psi_{g}\left(\bar{r}-\overrightarrow{R_{g}}\right)$. При этом около узла $g$ $\psi(\bar{r}) \approx e^{i k \overrightarrow{R_{g}}} \psi_{g}\left(\bar{r}-\overrightarrow{R_{g}}\right)$

$\sum \limits_{g}\left\{\left[-\frac{\hbar^{2}}{2 m} \nabla^{2}+U_g(\bar{r})\right]+\left[U(\bar{r})-U_{g}(\bar{r})\right]-E\right\} e^{i k \overrightarrow{R_{g}}} \psi_{g}=0 \quad$ (вид, $\quad$ чтобы $\quad$ был $\quad$ известный гамельтониан $\Rightarrow \sum \limits_{g}\left\{\left(E_{a}-E\right)+W(\bar{r})\right\} e^{i k \overrightarrow{R_{g}}} \psi_{g}=0$.

$\displaystyle \int \psi_{g}^{*} \psi_{g} d \bar{r}=S\left(R_{g}-R_{g}{ }^{\prime}\right)$ --- интеграл перекрытия


$\displaystyle \int \psi_{g}^{*} w(\bar{r}) \psi_{g} d \bar{r}=A\left(R_{g}-R_{g}{ }^{\prime}\right)$ --- обменный интеграл 

Тогда $\displaystyle \sum \limits_{g} e^{i k \overrightarrow{R_{g}}}\left\{\left(E_{a}-E\right) S\left(\overrightarrow{R_{g}}-\overrightarrow{{R_{g}}^{\prime}}\right)+A\left(\overrightarrow{R_{g}}-\overrightarrow{R_{g}}\right)\right\}=0$.

И, вводя $\bar{q}=\overrightarrow{R_{g}}-\overrightarrow{R_{g}}:\left(E_{a}-E\right) \sum \limits_{g} e^{i k \bar{q}} S(\bar{q})+\sum \limits_{g} A(\bar{q}) e^{i k \bar{q}}=0$,

\noindent где $E$ периодичная по волновому вектору добавка: $E=E_{a}+\frac{\sum \limits_{g} A(\bar{q}) e^{i k \vec{q}}}{\sum \limits_{g} S(\bar{q}) e^{i k \vec{q}}}$.

В суммы входят сначала слагаемые I-го типа (ближайшие соседа, т.е. I сфера), затем вторая сфера и так далее. В числителе учитывается перекрывание волновых функций соседи, в знаменателе для простоты перекрывание не учитывается:

$
\displaystyle
\sum \limits_{g} S(\bar{q}) e^{i k \bar{q}} \int \psi_{g}^{*} \psi_{g} d \bar{r}+\cdots=1$ (остальные слагаемые не учитываются)

$
\displaystyle
\sum_{g} A(\bar{q}) e^{i k \bar{q}} \int \limits _{\bar{q}=0}^{\bar{q}=0} \psi_{g}^{*} w(\bar{r}) \psi_{g} d \bar{r} * 1+A_{1} \sum_{i k} e^{i \bar{k} \bar{q}}$  (второе слагаемое отвечает за соседей из первой координационной сферы)


\begin{figure}[h!]
    \centering
    \includegraphics[width=0.2\textwidth]{phys_3_3}
\end{figure}

Конечное уравнение для $E(\bar{k}) = E_a+\underbrace{C}_{<0}+A_1 \sum e^{i\bar{k} \bar{q}}$.  В простой кубической решётке $E(\bar{k}) = E_a+C+A_1 \left( e^{i k_x a} + e^{-i k_x a} + e^{i k_y a} + e^{-i k_y a} + e^{i k_x a} + e^{-i k_z a}\right) = E_a+C+2A_1 \left( \cos k_x a + \cos k_y a + \cos k_z a\right) $

Следствия МСС:

\begin{enumerate}
    \item При образовании кристаллов из изолированных атомов каждый атомный уровень движется вниз и расширяется $\Rightarrow$ образуются зоны.
    
    \begin{figure}[h!]
        \centering
        \includegraphics[width=0.4\textwidth]{phys_stripes}
    \end{figure}

    \item Каждая зона ограничена $\mathrm{E}_{\min }$ и $\mathrm{E}_{\max .}$
    
    $$
    \begin{aligned}
    & E_{\min }=E_{a}+C-6|A| \quad (\mathrm{C}<0) \\
    & E_{\max }=E_{a}+C+6|A| .
    \end{aligned}
    $$


    Таким образом, ширина зоны $12 \mathrm{~A}_{1}$ $\left(\Delta \mathrm{E}=12\left|\mathrm{~A}_{1}\right|\right)$


    \item  При движении по энергии вверх (при переходе от нижних уровней к верхним) ширина разрещённых зон растёт (внешние оболочки перекрываются сильнее, чем внутренние). При этом ширина запрещенных зон уменьшается.
    \item Разрешённые и запрещенные зоны чередуются, возможно перекрытие зон.
    \item $E(k)$ --- чётная по $k$.
    \item Качественное изменение внешних условий
    
    $\mathrm{T} \uparrow \rightarrow \mathrm{a} \uparrow \rightarrow\left|\mathrm{A}_{1}\right| \downarrow \rightarrow \Delta \mathrm{E} \downarrow \rightarrow \mathrm{E}_{\mathrm{g} \uparrow}$

    $\mathrm{P} \uparrow \rightarrow \mathrm{a} \downarrow \rightarrow\left|\mathrm{A}_{1}\right| \uparrow \rightarrow \Delta \mathrm{E} \uparrow \rightarrow \mathrm{Eg}_{\mathrm{g}} \downarrow$

    \item Вырождение может честично или полностью сниматься при образовании кристалла из атомов.
    \item Заполнение энергетических зон.
\end{enumerate}







C ростом энергии растёт ширина разрешённых зон, а ширина запрещённых зон уменьшается. Из-за высокого положения уровней в какой-то момент возможно перекрывание. При образовании кристаллов вырождение может частично или даже полностью сниматься. Отсюда качественное понимание, как меняются свойства при изменении внешних условий:

$\mathrm{T} \uparrow \rightarrow \mathrm{a} \uparrow \rightarrow\left|\mathrm{A}_{1}\right| \downarrow \rightarrow \Delta \mathrm{E} \downarrow \rightarrow \mathrm{E}_{\mathrm{g} \uparrow}$

$\mathrm{P} \uparrow \rightarrow \mathrm{a} \downarrow \rightarrow\left|\mathrm{A}_{1}\right| \uparrow \rightarrow \Delta \mathrm{E} \uparrow \rightarrow \mathrm{Eg}_{\mathrm{g}} \downarrow$

\textbf{Заполнение энергетических зон: металлы, диаэлектрики, полупроводники}

\begin{table}[h!]
    \centering
    \resizebox{\textwidth}{!}{\begin{tabular}{|c|c|}
        \hline
        N изолированных атомов & Кристалл из N атомов \\
        \hline
        $N * 2(2 l+1)$ & $2 N(2 l+1)$ (число состояний сохраняется) \\
        \hline
        \multicolumn{2}{|c|}{Число уровней сохраняется+сохраняется степень заполнения состояний}  \\
        \hline
        \end{tabular}}
\end{table}

Если в системе из N атомов уровень был заполнен наполовину, то и в кристалле зона будет заполнена наполовину.

Целиком заполненная зона не даёт вклада в проводимость.

\textbf{Классификация твёрдых тел}

\begin{figure}[h!]
    \centering
    \includegraphics[width=0.8\textwidth]{images/phys_3_4.png}
\end{figure}

Полупроводник отличается от диэлектрика только размером запрещённой зоны (у диэлектрика $E_g > 5$эВ, у полупроводника $E_g < 5$эВ).


\section{Волновая функция электрона в кристалле. Теорема Блоха. Квазиимпульс электрона в кристалле. Зоны Бриллюэна.}

Для одного электрона, движущегося в кристалле, уравнение Шрёдингера имеет вид:
$$
\begin{aligned}
& \left\{-\frac{\hbar^{2}}{2 m} \nabla^{2}+U(\bar{r})\right\} \psi(\bar{r})=E \psi(\bar{r}) \\
& U\left(\bar{r}+\bar{a}_{n}\right)=U(\bar{r}), \quad \bar{a}_{n}=n_{1} \bar{a}_{1}+n_{2} \bar{a}_{2}+n_{3} \overline{a_{3}}
\end{aligned}
$$

\noindent U --- потенциал кристаллической решётки, объединяет в себе взаимодействие с решёткой и взаимодействие с другими электронами, $\bar{a_n}$ --- период кристаллической решётки. 

\textbf{Теорема Блоха:} волновая функция электрона в поле периодичного потенциала (в частности в кристалле) может быть предстравлена в виде произведения плоской волны и функции, обладающей той же периодичностью, что и кристаллическая решётка.

$$
\psi_{n k}=e^{i k r} U_{n k}(r), \quad U_{n k}(\bar{r}+\bar{a})=U_{n k}(\bar{r}) \text { --- блоховская функция }
$$

\noindent $\bar{a}$ --- период кристаллической решётки, $n$ --- номер зоны.

$\psi(\bar{r}+\bar{a})=e^{i \bar{k} \bar{a}} \psi(\bar{r})$ --- условие трансляционной симетрии для волновой функции в кристалле.

%$\psi _{\bar{k}} (\bar{r})=e ^{i \bar{k} \bar{a}} U_{\bar{k}} (\bar{r})$ --- будем искать решения уравнения Шрёдингера в виде $\widehat{H} \psi=\varepsilon \psi$

% $
% \begin{aligned}
% & U_{\bar{k}}(\bar{r})=U_{\hat{k}}(\bar{r}+\bar{a}) \text {--- периодичность функции} \\
% & \widehat{H} \psi_{\bar{k}}(\bar{r})=\hat{H} e^{i \bar{k} \bar{r}} U_{k}(\bar{r})=\hat{H} \varepsilon e^{i \bar{k} \bar{r}} U_{k}(\bar{r}) %\\
% & e^{-i \bar{k} \bar{r}} \widehat{H} e^{i \bar{k} \bar{r}} U_{\bar{k}}(\bar{r})=e^{-i \bar{k} \bar{r}} \hat{H} \varepsilon e^{i \bar{k} \bar{r}} U_{k}(\bar{r}) \\
% & \widehat{H}_{k}\left(U_{\bar{k}}(\bar{r})\right)=\varepsilon(\bar{k}) U_{\bar{k}}(\bar{r})\text { (1) (новая задача) }
% \end{aligned}
% $

% Представление волновой функции : 
% $$\psi _{n\bar{k}} = e ^{i\bar{k}\bar{r}} U_{n \bar{k}} (\bar{r}) $$


\textbf{Квазиимпульс в кристалле}


    
% $\hat{\bar{v}}=\frac{\hat{\bar{p}}}{m}=-i \frac{\hbar}{m} \nabla$

Скорость свободного электрона $ \displaystyle \langle\bar{V}\rangle=\int \limits_V A e^{-i \bar{k} \bar{r}}\left(-i \frac{\hbar}{m} \nabla\right) A e^{i \bar{k} \bar{r}} d r =  \frac{-i\hbar}{m}(i k) \underbrace{\int\limits _V A e^{-i \bar{k} \bar{r}} A e^{i \bar{k} \bar{r}} d r}_1 =\frac{\hbar k}{m}=\frac{p}{m}$

p--- импульс.


Скорость электрона в кристалле $\langle \bar{V} \rangle=\int \limits_V \psi^*(\bar{r}) \hat{\bar{V}} \psi(\bar{r}) d \bar{r}$

$\displaystyle \langle\bar{V}\rangle=\int u_k^*(\bar{r}) e^{-i \bar{k} \bar{r}}\left(-i \frac{\hbar}{m} \nabla\right) U_k(\bar{r}) e^{i \bar{k} \bar{r}} d r=  -\left(i \frac{\hbar}{m}\right)(i k) \int u_k^*(r) e^{-i k r} u_k(r) e^{i k r} d r+  \left(-i \frac{\hbar}{m}\right) \int u_k^*(r) e^{-i k r} e^{i k r} \nabla u_k(r) d r= \frac{\hbar k}{m}-i \frac{\hbar}{m} \int u_k^*(r) \nabla u_k(r) d r \neq \frac{p}{m}$

\bigskip

Для описания поведения электрона в кристалле вводится квазиимпульс $P$ такой, что $\displaystyle V=\frac{dE}{dP}$. Вместе с квазиимпульсом вводится квазиволновой вектор $k=P/\hbar$. Далее кразиимпульс будем обозначать $p$.

\textbf{Свойства квазиимульса:}
\begin{enumerate}
    \item $\displaystyle \frac{dp}{dt}=F=\nabla U$, причём $F$ --- внешняя сила.
    \item При отсутствии внешних сил квазиимпульс сохраняется.
    \item Любое искажение периодичности решётки, приводящее к процессам рассеяния электронов на дефектах, можно представить как наложение локальной внешней силы.
    \item Квазиимпульс дискретен ($\Delta p = 2\pi \hbar /a$).
    \item Квазиимпульс изменяется в пределах зоны Бриллюэна (в то время как импульс свободного электрона может принимать значения $[0; +\infty)$)
\end{enumerate}

Возьмём большой кристалл ($L \gg a$). Внутреннюю область разбиваем на блоки размером $L_x$, $L_y$, $L_z$.
$\psi(x; y; z) = \psi(x +L_x; y; z) = \psi(x; y+L_y; z) = \psi(x; y; z+L_z)$

$\psi(x+L_x)=\psi(x) \Rightarrow U_{k}(x+L_x)e^{ik_{x} (x+L_x)}=U_k(x)e^{ik_{x} x}$

Так как $e^{ik_{x} L_x}=1$, то $k_x L_x=2 \pi n$ ($n$ --- целое). Из этого следует дискретность квазиволнового вектора, так как $p=\hbar k$, то и дискретность квазиимпульса.

\textbf{Зона Бриллюэна} --- отображение ячейки Вигнера-Зейтса в обратном пространстве. В приближении Блоха волновая функция электрона для периодического потенциала решетки полность описывается ее поведением в 1 зоне Бриллюэна. 1я зона Бриллюэна содержит точку $k=0$ (2я зона непосредственно прилегает к 1 и так далее); она может быть построена как объем, ограниченный плоскостями, которые отстоят на равные расстояния от рассматриваемого узла обратной решетки до соседних узлов. Для простой кубической решетки зона Бриллюэна --- кубооктаэдр (14-граник).

\begin{figure}[h!]
    \centering
    \includegraphics[width=0.8\textwidth]{phys_4_1.png}
\end{figure}

Число разрешённых (возможных) состояний квазиволнового вектора в зон Бриллюэна конечно. Число разрешённых состояний в кристалле с параметром решётки $a$ и периодичностью квазиволнового вектора $2\pi /L$: $\frac{2\pi}{a} / \frac{2\pi}{L} = 2 L/a = 2N$. $N$ --- количество атомов в атомной решётке, 2 из-за возможных ориентаций спина.

Все зоны Бриллюэна имеют одинаковый объём.



\section{Закон дисперсии, изоэнергетическая поверхность, эффективная масса электрона в кристалле}

\textbf{Закон дисперсии} --- зависимость энергии от квазиимпульса в разрешенной зоне ($E(p)$).

\textit{Классификация:}

\begin{enumerate}
    \item Квадратичные ($E \sim p^2$)
    \begin{enumerate}
        \item Квадратичный изотропный $E=\frac{p^2}{2 m^*}$ (изоэнергетическая поверхность --- сфера, $R=\sqrt{2m^*E}$)
        \item Квадратичный анизотропный $E=\frac{p_x^2}{2 m_x}+\frac{p_y^2}{2 m_y}+\frac{p_z^2}{2 m_z}$ (изоэнергетическая поверхность --- трёхосный эллипсоид. Для случая $m_x=m_y$ --- эллипсоид вращения).
    \end{enumerate}
    \item Неквадратичные ($E \nsim p^2$, $m^* \neq const \quad m^* \uparrow \uparrow E$)
    \begin{enumerate}
        \item Закон Кейна (изотропный) $E(1+\frac{E}{E_g})=\frac{p^2}{2m^*}$ (изоэнергетическая поверхность в близи экстремума --- сфера $R=\sqrt{2m^*(0)E(1+E/E_g)}$)
        \item Закон дисперсии в методе сильной связи (для ПКР) $E=E_a+C+2 A_1\left(\cos \left(k_x a\right)+\cos \left(k_y a\right)+\cos \left(k_z a\right)\right)$ (В окрестностях экстремумов выраждается в квадратичный закон $E=|A_1|a^2k^2$, изоэнергетическая поверхность в близи экстремума --- сфера $R=\sqrt{E/(|A_1|a^2)}$)
    \end{enumerate}
\end{enumerate}

\textbf{Эффективная масса электрона}

Введение эффективной массы электрона позволяет описывать его движение в кристалле при помощи законов механики. $\displaystyle \frac{1}{m^*} = \frac{d^2E}{dp^2}$. В близи минимума энергии $m^*>0$, в близи максимума энергии $m^*<0$. 

В трёхмерном случае $m^*$ --- тензор, который можно диагонализировать правильным выбором точки отсчёта.


$$
    \widetilde{m^*}^{-1}=\left(\begin{array}{ccc}
    \frac{\partial^2 E}{\partial p_x^2} & \frac{\partial^2 E}{\partial p_x \partial p_y} & \frac{\partial^2 E}{\partial p_x \partial p_z} \\
    \frac{\partial^2 E}{\partial p_y \partial p_x} & \frac{\partial^2 E}{\partial p_y^2} & \frac{\partial^2 E}{\partial p_y \partial p_z} \\
    \frac{\partial^2 E}{\partial p_z \partial p_x} & \frac{\partial^2 E}{\partial p_z \partial p_y} & \frac{\partial^2 E}{\partial p_z^2}
    \end{array}\right) \rightarrow \left(\begin{array}{ccc} m_1^{-1} & 0 & 0 \\ 0 & m_2^{-1} & 0 \\ 0 & 0 & m_3^{-1} \end{array}\right)
$$

$\bar{F}_\text{внеш} =  \widetilde{m^*}\bar{a}$

\section{$\rm k \cdot p$--метод. Энергетический спектр электрона в однозонном приближении. Правила сумм. Взаимодействие энергетических зон}

\textbf{$\rm k \cdot p$--метод.}

Методы расчёта <<из первых принципов>> не обладают достаточной точностью для расчёта энергетических зон узкощелевых полупроводников так как:

\begin{enumerate}
    \item $E_g$ в данных материалах может быть <0,1эВ (или 0эВ)
    \item Результат получается в численном виде (таблица с числами в пределах зоны Бриллюэна). В УЩПП необходимо знать закон дисперсии в окрестности экстремумов зон ($E_c (k)$ и $E_v (k)$) в аналитическом виде.
\end{enumerate}

Преимущество $\rm k \cdot p$--метода --- при расчёте используются экспериментальные данные, а в результате получается аналитический вид закона дисперсии.

\textbf{Предположения:}

\begin{itemize}
    \item $\psi _{n k_0}(\bar{r})$ --- волновые функции в экстремумах зон
    \item $E_n(\bar{k_0})$ --- энергия краёв зон
    \item Матричные элементы оператора импульса (из эксперимента)
\end{itemize}
 То есть известны решения уравнения Шрёдингера $\displaystyle \left\{\frac{\widehat{\bar{p}^2}}{2 m}+U(\bar{r})\right\} \psi_{n k_0}(\bar{r})=E_n\left(\bar{k}_0\right) \psi_{n k_0}(\bar{r})$. Можно найти $\displaystyle \left\{\frac{\widehat{\bar{p}^2}}{2 m}+U(\bar{r})\right\} \psi_{n k_0}(\bar{r})=E_n\left(\bar{k}_0\right) \psi_{n k}(\bar{r})$ в малых окрестностях экстремумов зон $k_0$. То есть можно найти законы дисперсии.

 \begin{figure}[h!]
    \centering
    \includegraphics[width=0.2\textwidth]{phys_7_1}
    
 \end{figure}

 \begin{enumerate}
    \item Переход от уравнения для $\psi_{nk}$ к уравнению $U_{nk}$. 
    
    $\psi_{n k}(\bar{r})=U_{n k}(\bar{r}) e^{i \bar{k} \bar{r}}$ 
    
    $\displaystyle \psi_{n k}(\bar{r})=U_{n k}(r) e^{i \bar{k} \bar{r}} \hat{\bar{p}}\left(U_{n k} e^{i \bar{k} \bar{r}}\right)=e^{i \bar{k} \bar{r}} \hat{\bar{p}} \left( U_{n k} \right) + U_{n k} \hat{\bar{p}} \left( e^{i \bar{k} \bar{r}} \right) =e^{i \bar{k} \bar{r}} \hat{\bar{p}} \left( U_{n k} \right)+ \hbar k  e^{i \bar{k} \bar{r}}U_{n k}= e^{i \bar{k} \bar{r}}(\hat{\bar{p}}+\hbar \bar{k}) U_{n k}$
    
    $\hat{\bar{p}} \left( \hat{\bar{p}}\left(u_{n k} e^{i \bar{k} \bar{r}}\right) \right) =\ldots=e^{i \bar{k} \bar{r}}(\hat{\bar{p}}+ \hbar \bar{k})^2 U_{n k}$
    
    $\displaystyle \underbrace{\left\{\frac{(\hat{\bar{p}}+\hbar k)^2}{2 m}+U(\bar{r})\right\}}_{\hat{H}_k} U_{nk}(\bar{r}) =E_n(\bar{k}) U_{n k}(\bar{r})$


    $\hat{\bar{p}} \overset{\psi_{nk} \rightarrow U_{nk}}{\longrightarrow} (\hat{p}+\hbar k)$

    $\left\{\frac{(\hat{\hat{p}}+\hbar k)^2}{2 m}+U(\bar{r})\right\}=\hat{\mu}_k$

    \item Выделение оператора $\hat{H}_{k_0}$ и дополнительного возмущения зависящего от $(k-k_0)$
    
    $\displaystyle \hat{H}_k=\frac{\hat{\bar{p}}^2}{2 m}+\frac{\hbar}{m} \bar{k}\bar{p}+\frac{\hbar^2 k^2}{2 m} +U(\bar{r})$
    
    $\displaystyle \hat{H}_k=\hat{H}_{k_0}+\hat{V}$
    
    $\displaystyle \hat{H}_k=\hat{H}_{k_0}+\frac{\hbar}{m}\left(\bar{k}-\hat{k}_0\right) \hat{\bar{p}}+\frac{\hbar^2}{2 m}\left(\bar{k}-\bar{k}_0\right)$
    
    $\displaystyle \hat{V}=\frac{\hbar}{m}\left(\bar{k}-\bar{k}_0\right) \hat{\bar{p}}+\frac{\hbar^2}{2}\left(\bar{k}-\bar{k}_0\right) $

    $\hat{V}$ --- возмущение.

    $\displaystyle \hat{V}$ возрастает при удалении $\bar{k}$  от $\bar{k}_0$.

    \item Применяем теорию возмущений. Разложение $U_{nk}(\bar{r})$ в ряд по известным $U_{nk_0}(\bar{r})$. Подставляем в уравнение, умножаем на $U^*_{nk_0}$, интегрируем по элементарной ячейке. Переходим к системе уравнений относительно $C_{nn^\prime}$
    
    $\sum_{n^{\prime}} C_{n n^{\prime}}\left[\int U_{n k_0}^* \hat{H}_{k_0} U_{n^{\prime} k_0} d \bar{r}\right. +\frac{\hbar}{m}\left(\bar{k}-\bar{k}_0\right) \int U_{n k_0}^* \hat{\bar{p}} U_{n^{\prime} k_0} d \bar{r}^*\left.+\frac{\hbar^2}{2 m}\left(\bar{k^2}- \bar{k^2}_0\right) \int U_{n k_0}^* U_{n^{\prime} k_0} d r\right]= \sum_{n^{\prime}} C_{n n^{\prime}} E_n(\bar{k}) \underbrace{\int U_{n k_0}^* U_{n^{\prime} k_0} d r}_{\delta_{n n_1}}$

    $\int U_{n k_0}^* \hat{H}_{k_0} U_{n^{\prime} k_0} d \bar{r} = E_n (\bar{k}_0)\delta_{nn^\prime}$
    
    $\left\langle U_{n k_0}|\hat{\bar{p}}| U_{n^{\prime} k_0}\right\rangle=p_{n n^{\prime}}\left(\bar{k}_0\right)$ --- матричный оператор импульса.

    \item Определитель системы должен быть равен <<0>>
    
    $\mathrm{det} \Bigg| \left[\left\{E_1(\bar{k}_0)-E_n(\bar{k})+\frac{\hbar^2}{2 m} \right\} \delta_{n n^{\prime}}+\frac{\hbar}{m}(\bar{k}-\bar{k}_0) p_{n n^{\prime}}(\bar{k}_0)\right]  \Bigg| =0$

    В общем случае $n=\infty$

 \end{enumerate}

Решая определитель можно получить законы дисперсии для $E_n(\bar{k})$. Мешает то, что надо знать много значений оператора импульса и наличие большого количества зон.

\textbf{Однозонное приближение}

1 зона с номером $n$. В идём в конец 3 пункта, $n=n^\prime$. $E(\bar{k}_0) - E(\bar{k}) + \frac{\hbar}{2m}(\bar{k}^2 - \bar{k}^{2}_0) + \frac{\hbar}{m}(\bar{k} - \bar{k}_0)\bar{p}_{nn}(\bar{k}_0)=0$

Воспользовавшись $p_{A n}(k_0)=-\hbar k_0$ получаем $E(\bar{k})=E\left(\bar{k}_0\right)+\frac{\hbar^2\left(\bar{k}-\bar{k}_0\right)^2}{2 m}$.

\textbf{Правила сумм}

Вернёмся в конец пункта 2.

$\mathrm{H}=\hat{H}_{k_0}+\frac{\hbar}{m}\left(\bar{k}-\bar{k}_0\right) \hat{\bar{p}}+\frac{\hbar^2}{2 m}\left(\bar{k}^2-\bar{k}_0^2\right)$

В теории возмущений для невырожденных состояний 

$\left(\hat{H_0}+\hat{U}\right) \psi=E \psi \quad E=E_0+E_1+E_2+\ldots$

$E_1=V_{n n}=\int \psi_n^* \hat{V} \psi_n d r$

$E_L=\sum_{n^{\prime} \neq n} \frac{V_{n n^{\prime}} \cdot V_{n^{\prime} n}}{E_n(0)-E_{n^{\prime}}(0)}$

$E_n(\bar{k})=E_n\left(\bar{k}_0\right)+\underbrace{\frac{\hbar}{m}\left(\bar{k}-\bar{k}_0\right) \int U_n^* \hat{М} U_n d r+\frac{\hbar^2}{2 m}\left(k^*-k_0^2\right) \int U^* U d r}_{E_1} + \underbrace{\frac{\hbar^2}{m^2} \sum_{n^{\prime} \neq n} \frac{\left(k-k_0\right) p_{n n^{\prime}}\left(\bar{k}_0\right) \cdot\left(k-k_0\right) p_{n^{\prime} n}\left(\bar{k}_0\right)}{\left(E_n \bar{k}_0\right)-E_{n^{\prime}}\left(\bar{k}_0\right)}}_{E_2} + E_3+ \ldots$

$\hbar (\bar{k}-\bar{k}_0)=\bar{p}$ --- квазиимпульс.

$E=E_n\left(E_0\right)+\frac{p^2}{2 m^*}$

$\displaystyle 
\left.\begin{array}{l}\frac{1}{m^*}=\frac{1}{m}+\frac{2}{m^2} \sum \frac{p_{n n}\left(\bar{k}_0\right) p_{n^\prime n}\left(k_0\right)}{E_n\left(k_0\right)-E_{n^{\prime}}\left(k_0\right)}+\ldots \\ E_n(\bar{k})=E_n\left(\bar{k}_0\right)+\frac{p^2}{2 m}+\frac{1}{m^2} \sum_{n^\prime \neq n} \frac{\bar{p} \bar{p}_{n^\prime n}\left(k_0\right) \cdot \bar{p} \bar{p}_{n^\prime n }\left(k_0\right)}{E_n\left(k_0\right)-E_{n^\prime}\left(k_0\right)}+\ldots\end{array}\right\} \text{--- правила сумм}$

\textit{Правила сумм} для $E_n(\bar{k})$ и $m^*$:

\begin{enumerate}
    \item Отклонение сумм для $E_n(\bar{k})$ от квадратичного изотропного закона и $m^*$ от $m$ --- следствие учёта наличия других зон.
    \item Вклад других зон --- знакопеременный.
    \item Количественно ряд зависит от $E_n-E_{n^\prime}$ и от величины $p_{nn^\prime}(\bar{k}_0)$.
\end{enumerate}

Если $p_{nn^\prime}(\bar{k}_0)=0$, то зоны не взаимодействуют. Если $p_{nn^\prime}(\bar{k}_0)\neq 0$, то говорят, что зоны взаимодествуют.


\section{Параметры зонной структуры полупроводников. Зонная структура основных полупроводниковых материалов.}

\textbf{Основные параметры зонной структуры:}

\begin{enumerate}
    \item Кристаллическая решётка
    \item Форма зоны Бриллюэна и расположение в ней основных точек симметрии и направлений
    \item Положение экстремумов зон в зоне Бриллюэна, величина $E_g$.
    \item Законы дисперсии и изоэнергетические поверхности в окрестностях экстремумов зон.
\end{enumerate}

\textit{//для справки//}

Для ГЦК решётки 1 зона Бриллюэна --- кубооктаэдр в ней

$\Gamma$ --- центр зоны Бриллюэна

$X$ --- центры квадратных граней (3 штуки)

$L$ --- центры гексогональных граней (4 штуки)

$\Sigma$ --- внутри зоны Бриллюэна в направлении типа $\langle 110 \rangle$ (12 штук)

$\Delta$ --- внутри зоны Бриллюэна в направлении типа $\langle 100 \rangle$ (6 штук)



\textbf{Зонная структура Ge и Si}

\begin{enumerate}
    \item Кристаллическая решётка типа алмаза (2 ГЦК подрешётки, сдвинутые друг относительно друга на $\frac{1}{4}$ пространственной диагонали куба)
    \item Зона Бриллюэна такая же как и у ГЦК решётки Браве --- 14-гранник
    \item Ge: минимум зоны проводимости в точках $L$ (4 точки на зону Бриллюэна). максимум валентной зоны в точке $\Gamma$ (1 точка)
    
    $E_g \approx 0,67$~эВ при $T=300$~К

    Si: минимум зоны проводимости в точках $\Delta$ (6 штук на зону Бриллюэна), максимум валентной зоны в точке $\Gamma$ (1 точка)
    
    $E_g \approx 1,1$~эВ при $T=300$~К

    \begin{figure}[h!]
        \centering
        \includegraphics[width=0.5\textwidth]{phys_8_1}
    \end{figure}

    \item Зона проводимости: $\displaystyle E(\bar{k}) = \frac{\hbar^2 k_\perp ^2}{2m_t^*} + \frac{\hbar^2 k_\parallel ^2}{2m_l^*}$
    
    $k_\perp ^2 = k_x ^2 + k_y ^2$, $k^2_\parallel = k^2 _z$

    Ge: $k_\parallel \parallel \langle 111 \rangle$

    Si: $k_\parallel \parallel \langle 100 \rangle$

    Изоэнергетические поверхности --- эллипсоиды вращения.

    \item Валентная зона: $\displaystyle E_{1;2}=-\frac{\hbar^2}{2m}\left[ Ak^2 \pm \left( B^2k^4 + c^2(k_x^2k_y^2+ k_y^2k_z^2 + k_x^2k_x^2) \right)^0,5 \right]$, (A, B, C --- безразмерные константы). Изоэнергетические поверхности --- гофрированные сферы.
    
    $E_3 (\bar{k})=-\frac{\hbar^2 \bar{k}^2}{2m^*_3}$, $m^*_3=\frac{m}{A}$. Изоэнергетические поверхности --- сферы.
\end{enumerate}





\textbf{Зонная структура $A^3B^5$}

A: In, Ga, Al...

B: Sb, As, P, N...

Общая (наиболее распространённа)я ситуация

\begin{enumerate}
    \item Кристаллическая решётка типа сфалерита (цинковой обманки)
    \item Зона Бриллюэна --- кубооктаэдр
    \item Закон дисперсии Кейна
\end{enumerate}

\begin{figure}[h!]
    \centering
    \includegraphics[width=0.4\textwidth]{phys_8_2}
\end{figure}

Отклонение от этой системы:

\begin{itemize}
    \item GaN, InN, AlN кристаллизуются в решётки вюрцита (ГПУ)
    \item Наличие у ряда полупроводников дополнительных экстремумов в зоне проводимости
\end{itemize}

\begin{figure}[h!]
    \centering
    \includegraphics[width=0.8\textwidth]{phys_8_3}
\end{figure}

\begin{figure}[h!]
    \centering
    \includegraphics[width=0.7\textwidth]{phys_8_4}
\end{figure}

Прямозонные используются в оптике.


\textbf{Зонная структура $A^2B^6$}

A: Zn, Cd, Hg...

B: O, Se, S, Te...

\begin{enumerate}
    \item Кристаллическая решётка типа сфалерита или вюрцита
    \item Зона Бриллюэна --- кубооктаэдр
    \item Выполняется закон Кейна
\end{enumerate}

Особенности полупроводников на основе Hg:

\begin{enumerate}
    \item При низких температурах существуют свободные носители заряда (как в металлах)
    \item Инверсная зонная структура ($E_g<0$ в законе Кейна)
\end{enumerate}

\begin{figure}[h!]
    \centering
    \subfigure{
    \includegraphics[width=0.4\textwidth]{phys_8_5}}
    \subfigure{
    \includegraphics[width=0.5\textwidth]{phys_8_6}}
\end{figure}


\textbf{Зонная структура $A^4B^6$}

A: Sn, Ge, Pb...

B: O, S, Se, Te...

\textit{Кристаллическая структура}

\begin{enumerate}
    \item Кубические модификации (решётки типа NaCl). Зона Бриллюэна --- кубооктаэдр
    \item Ромбическая решётка (деформация куба вдоль $\langle 111 \rangle$) (CdTe)
    \item Орторомбическая (деформация куба вдоль $\langle 110 \rangle$) (PbTe)
\end{enumerate}


\vspace{2cm}

\begin{figure}[h!]
    \centering
    %
    \subfigure{
        \includegraphics[width=0.2\textwidth]{phys_8_7}}
    \subfigure{
    \includegraphics[width=0.2\textwidth]{phys_8_8}}
\end{figure}



\vfill

\textbf{Модель инверсии зон Диммака}

\begin{figure}[h!]
    \centering
    \subfigure{
    \includegraphics[width=0.45\textwidth]{phys_8_9}}
    \subfigure{\includegraphics[width=0.45\textwidth]{phys_8_9.2}}

\end{figure}

\begin{figure}[h!]
    \centering\subfigure{
    \includegraphics[width=0.45\textwidth]{phys_8_10}}
    \subfigure{
    \includegraphics[width=0.45\textwidth]{phys_8_11}}
\end{figure}

Справа от весщелевого состояния $E_g \downarrow \uparrow T$, слева от бесщелевого сотояния $E_g \uparrow \uparrow T$.

Для полупроводников на основе свинца наблюдается рост с насыщением запрещённой зоны от температуры (для большинства других полупроводников с ростом температуры запрещённая зона уменьшается). С ростом температуры края зон $L{6}$ и $L_{\bar{6}}$ расходятся (увеличение запрещённой зоны), а край зоны $\Sigma$ движется вверх (приблизительно с такой же скорость, что и $L_6$). С некоторого момента зона $\Sigma$ оказывается выше $L{6}$ (выход на насыщение).

До точки $x\leq 0,3$ можно использовать модифицированный закон Кейна ($\displaystyle E\left(1+\frac{E}{E_g}\right)=\frac{\hbar^2 k_\perp^2}{2 m_t^*(0)}+\frac{\hbar^2 k_\parallel^2}{2 m_l^*(0)}$, 2 параметра $m_t$ и $m_l$).

Далее лучше закон дисперсии Диммака ($\left(\frac{E_g}{2}+\frac{p_\perp^2}{2 m_t^{-}}+\frac{p_\parallel^2}{2 m_t^{-}}-E\right)\left(-\frac{E_g}{2}-\frac{p_\perp^2}{2 m_t^{+}}-\frac{p_\parallel^2}{2 m_t ^+}-E\right)=E_\perp \frac{p_\perp^2}{2 m}+E_\parallel \frac{p_\parallel^2}{2 m}$, $m_t^{ \pm}$, $m_l^{ \pm}$, $E_\perp$, $E_\parallel$ ---  6 параметров, определяются только экспериментально).


\begin{figure}[h!]
    \centering
    \includegraphics[width=0.6\textwidth]{phys_8_12}
\end{figure}

\section{Прямой и инверсный спектры Кейна для полупроводников $A^3B^5$ и $A^2B^6$.}

\textbf{Зонная структура $A^3B^5$}

A: In, Ga, Al...

B: Sb, As, P, N...

Общая (наиболее распространённа)я ситуация

\begin{enumerate}
    \item Кристаллическая решётка типа сфалерита (цинковой обманки)
    \item Зона Бриллюэна --- кубооктаэдр
    \item Закон дисперсии Кейна:

    $$\left\{ \begin{array}{c}
        E_c=E_g+\frac{1}{2} E_g \left( \sqrt{1+\frac{8k^2P^2}{3E_g^2}} -1  \right) \\
        E_{v_1} = -\frac{\hbar^2 k^2}{2m}\\
        E_{v_2} = -\frac{E_g}{2}\left( \sqrt{1+\frac{8k^2P^2}{3E_g}} -1 \right)\\
        E_{v_3} = -\Delta - \frac{P^2k^2}{3(E_g+\Delta)}
    \end{array} \right.$$
    
    $\Delta$ --- энергия спинорбитального взаимодествия, $P=\frac{\hbar}{m} \int U^*_c \hat{\bar{p}} U_{v_2} d\bar{r}$ --- матричный элемент оператора импульса ($\hat{\bar{p}}=-i\hbar \nabla$), $U_c$, $U_{v_2}$ --- блоховские амплитуды.
    
    Неквадратичный закон для электронов и лёгких дырок (зоны проводимости и лёгких дырок зеркальны ($m_c^*=m_{v_2}^*$)).
    
    При выборе начала отсчёта на уровне Ферми закон превращается в $E\left(1+\frac{E}{E_g}\right)=\frac{2 k^2 p^2}{3 E_g} \leftrightarrow \frac{\hbar^2 k^2}{2 m^*}$. В узкощелевых полупроводниках необходимо учитывать вклад $1+\frac{E}{E_g}$, эффективная циклотронная масса  $m^*_c(E)=m^*(0)(1+\frac{2E}{E_g})$.
\end{enumerate}

\begin{figure}[h!]
    \centering
    \includegraphics[width=0.4\textwidth]{phys_8_2}
\end{figure}

Отклонение от этой системы:

\begin{itemize}
    \item GaN, InN, AlN кристаллизуются в решётки вюрцита (ГПУ)
    \item Наличие у ряда полупроводников дополнительных экстремумов в зоне проводимости
\end{itemize}

\begin{figure}[h!]
    \centering
    \includegraphics[width=0.8\textwidth]{phys_8_3}
\end{figure}

\begin{figure}[h!]
    \centering
    \includegraphics[width=0.7\textwidth]{phys_8_4}
\end{figure}

Прямозонные используются в оптике.


\textbf{Зонная структура $A^2B^6$}

A: Zn, Cd, Hg...

B: O, Se, S, Te...

\begin{enumerate}
    \item Кристаллическая решётка типа сфалерита или вюрцита
    \item Зона Бриллюэна --- кубооктаэдр
    \item Выполняется закон Кейна
\end{enumerate}

Особенности полупроводников на основе Hg:

\begin{enumerate}
    \item При низких температурах существуют свободные носители заряда (как в металлах)
    \item Инверсная зонная структура ($E_g<0$ в законе Кейна)
\end{enumerate}

\begin{figure}[h!]
    \centering
    \subfigure{
    \includegraphics[width=0.4\textwidth]{phys_8_5}}
    \subfigure{
    \includegraphics[width=0.5\textwidth]{phys_8_6}}
\end{figure}


$\displaystyle E=\frac{E_g}{2}\left(\sqrt{1+\frac{8 k^2 p^2}{3 E_g^2}}-1\right) \underset{E_g=0}{\longrightarrow} \quad E=\sqrt{\frac{2}{3}} pk$ --- Линейный закон дисперсии (как у графена. Обладает нулевой запрещённой зоной).

\section{Энергетические уровни дефектов в полупроводниках. Водородоподобная модель примеси. Примесные зоны.}

\textbf{Собственная проводимость.}

Собственный полупроводник --- полупроводник без дефектов и примесей.

При $T=0$ проводимость нулевая. При повышении температуры появляются свободные электроны (возникает проводимость $\sigma = e \mu n$). На место ушедшего электрона может придти электрон из соседней связи. Разорванная связь --- дырка тоже вносит вклад в проводимость. Концентрация электронов и дырок одинакова $\sigma = e \mu n + e \mu p$. Механизм проводимости --- по незаполненным связям.

\begin{figure}[h!]
    \centering
    \includegraphics[width=0.5\textwidth]{intrinsic}
\end{figure}

\textbf{Примесная проводимость}

\textit{Донорный полупроводник (n-тип)}

\begin{figure}[h!]
    \centering
    \includegraphics[width=0.5\textwidth]{Donor}
\end{figure}

При $T=0$ электрон связан с атомом примеси и не вносит вклад в проводимость. С ростом температуры всё больше ёлетронов покидают атом примеси и начинают движение по решётке. В этом случае проводимость только электронная.

\textit{Акцепторный полупроводник (p-тип)}

\begin{figure}[h!]
    \centering
    \includegraphics[width=0.5\textwidth]{Acceptor}
\end{figure}

При лигировании примесью с меньшим количеством валентных электронов, чем у основного атома одна связь остаётся незаполненной. На эту связь могут перемещаться электроны с другой (перемещая незаполненную связь по кристаллу) --- осуществляется дырочная проводимость.

\textbf{Примесные зоны. Классификация по степени лигирования.}

\begin{figure}[h!]
    \centering
    \includegraphics[width=0.95\textwidth]{prim}
\end{figure}

\begin{enumerate}
    \item Слабое лигирование. Среднее расстояние между примесями много больше боровского радиуса. Примеси не взаимодействуют.
    \item Среднее лигирование. Среднее расстояние порядка боровского радиуса, примеси взаимодействую. Перекрытие волновых функций электронов соседних примесей приводит к размытию уровней в зоны.
    \item Волновые функции электронов примесей сильно перерываются, примесная зона может достигать зоны проводимости и перекрываться с ней. Сильнолигированный полупроводник похлж на металл.
\end{enumerate}

Появление дефектов --- появление локальных уровней на фоне зонной структуры полупроводника. Нахождение уравнений дефектов --- решение уравнения Шрёдингера с потенциалом дефекта $V(\bar{\epsilon})$

$$
\left\{\frac{-\hbar^2}{2 m} \nabla^2+U(\bar{r})+V(\bar{r})\right\} \Psi(\bar{r})=E \psi(\bar{r})
$$

$\psi(\bar{r})$ --- волновая функция в окрестности дефекта, $\Psi(\bar{r})=\sum_{\bar{n}^{\prime} ; \bar{k^{\prime}}} C_{n^{\prime}}\left(\bar{k}^{\prime}\right) \psi_{n^{\prime} k^{\prime}}(\bar{r})$, $\bar{n}^{\prime}$ --- номер зоны, $\bar{k^{\prime}}$ --- точки в зоне Бриллюэна.

Задача не имеет общего решения. Выделяется самый простой возможный случай, в котором можно получить конечное решение: водородоподобная модель.

$$\Psi(\bar{v}) \sim \psi_{n, \bar{k}_0}(\bar{r})$$

\noindent $k_0$ --- точка экстремума зоны Бриллюэна.

$V(\bar{r})=\frac{e^2}{4 \pi \varepsilon \varepsilon_0 r}$

$$
\left\{\begin{array}{l}
\Delta E_d=\frac{13,6}{\varepsilon^2}\left(\frac{m^*}{m}\right) \frac{1}{n^2} \\
a_B=0,53 \frac{\varepsilon}{z}\left(\frac{m}{m^*}\right) n^2
\end{array}\right.
$$


\begin{enumerate}
    \item Случай <<мелких>> уровней $V(\bar{r})$ --- дальнодействующий потенциал. $\Delta x \Delta p \geq \hbar \Rightarrow \Delta x \Delta k \geq 1$, $\Delta x = a_B \gg a \Rightarrow \Delta k \ll 1/a$ (подходит для описания примесных уровней $Si$, $Ge$, $A^3B^5$, $A^2B^6$)
    
    \begin{figure}[h!]
        \centering
    \includegraphics[width=0.3\textwidth]{phys_10_1}
    \end{figure}

    \item Альтернативный случай --- <<глубокие>> уровни. В этом случае учитываются многие энергетические зоны и все точки зоны Бриллюэна. Реализуется при: $V(\bar{r})$ --- короткодействующий, $\Delta E_t \approx E_g$. $|Delta x \leq a \Rightarrow \Delta k \sim \Delta k_\text{З.Бр.}$
    \item Резонансные уровни --- уровни расположение на фоне разрешённых зон: возможны переходы электронов с уровня в зону и обратно из-за этого уширение уровня дефектов ($\Delta E \Delta t \geq \hbar$). Резонансные уровни могут быть как мелкими так и глубокими.
\end{enumerate}

\begin{figure}[h!]
    \centering
    \includegraphics[width=0.7\textwidth]{phys_10_2}
\end{figure}

Глубокие уровни образованы всеми точками зон Бриллюэна, они не могут следовать за какой-то конкретной.

\begin{figure}[h!]
    \centering
    \includegraphics[width=0.7\textwidth]{phys_10_3}
\end{figure}

\section{Уравнение электронейтральности. Статистика носителей заряда в полупроводнике с одним типом примеси.}

Из расчёта известно $n=N_c F_{1/2}(\eta)$, $p=N_v F_{1/2}(\eta - \Delta\varepsilon)$.

\noindent где $F_{1/2}$ --- интеграл Ферми, $\frac{E-E_{c}}{kT}=\varepsilon$ --- энергия электронов относительно дна зоны проводимости, $\frac{E_g}{kT}=\Delta \varepsilon$, $\frac{F-E}{kT}=\eta$, $F$ --- уровень Ферми, $N_{с}$ $N_{v}$ --- функции плотности состояний в зоне проводимости и в валентной зоне. Но не известно значение уровня Ферми. Цель введения закона электронейтральности --- расчет $F(T)$.

Суть (физический смысл): кристалл в целом электронейтрален, хотя состоит из заряженных частиц. $\sum \limits_i N_{-} = \sum \limits_i N_{+}$.

\begin{figure}[h!]
    \centering
    \includegraphics[width=0.9\textwidth]{phys_11_1}
\end{figure}

$N_d$ --- концентрация атомов донорной примеси
$N_a$ --- концентрация атомов акцепторной примеси
$n_d$ --- концентрация электронов на донорном уровне ($E_d$)
$p_d$ --- концентрация дырок на донорном уровне ($E_d$)
$n_a$ --- концентрация электронов на акцепторном уровне ($E_a$)
$p_a$ --- концентрация дырок на акцепторном уровне ($E_a$)
$N_d^+$ --- концентрация ионов донорной примеси
$N_a^-$ --- концентрация ионов акцепторной примеси

$$
\left\{\begin{array} { l } 
{ N _ { d } ^ { + } = p _ { d } } \\
{ N _ { a } ^ { - } = n _ { a } }
\end{array} \quad \left\{\begin{array}{l}
n_d+p_d=N_d \\
n_a+p_a=N_a
\end{array}\right.\right.
$$

Каждый примесный уровень может быть занят либо электроном, либо дыркой ($N_a^-=N_a-p_a$, $N_d^+=N_d-n_d$)

Уравнение электронейтральности: $n+N_a-p_a=p+N_d-n_d$ (можно обобщить для нескольких типов примеси $n+\sum n_d-p-\sum p_a=\sum N_d-\sum N_a$).

\textbf{Статистика полупроводника с одним типом примеси}

Из уравнения электронейтральности следует, что $n=N_d-n_d+p=p_d+p$.

Предположения:

\begin{itemize}
    \item невырожденный полупроводник $n=N_ce^\eta$, $p=N_ve^{-\eta-\Delta\varepsilon}$
    \item Мелкие примеси $\Delta E_d \ll E_g$ (это позволит разбить шкалу температур на характерные уровни)
\end{itemize}

\begin{figure}[h!]
    \centering
    \includegraphics[width=0.5\textwidth]{phys_11_3}
\end{figure}

1 --- область низких температур, в которой происходит ионизация только атомов примеси.
2 --- область высоких температур, в которой происходит ионизация атомов основного вещества.

1. Ионизация только примеси $\Rightarrow p=0 \Rightarrow n=p_d$.

$$\displaystyle p_d=\frac{N_d}{ge^{\frac{F-E_d}{kt}}+1} = \frac{N_d}{g e^{g+\varepsilon_d}+1}$$

%\vspace{0.5cm}

$$\displaystyle \frac{F-E_c}{kT} = \eta, \; \frac{E_c-E_d}{kT}=\frac{\Delta E_d}{kT} = \varepsilon_d, \; n=N_c e^\eta, \; e^\eta = \frac{n}{N_c}$$

$$n=\frac{N_{d}}{g \frac{n}{N_c} e^{\varepsilon_d}+1}$$

$$n^2+\underbrace{\frac{N_c}{g} e^{-\varepsilon d}}_a n \underbrace{-\frac{N_c N_d}{g} e^{-\varepsilon_d}}_c=0$$

$$x^2+ax+c=0$$

$$x=\frac{a}{2}\left( \pm \sqrt{1-\frac{4 c}{a^2}}-1\right)$$

$$
n=\frac{N_c}{2 g} e^{-\varepsilon_d}\left[ \pm \sqrt{1+\frac{4 g N_d}{N_c} e^{\varepsilon_d}}-1\right]
$$

\vspace{2cm}

а) $\displaystyle \frac{4gN_d}{N_c} e^{\varepsilon _d} \gg 1, \; \varepsilon_d \sim \frac{1}{T}, \; N_c \sim T^{3/2}$

\vspace{0.5cm}

Случай низких температур в области 1

$$
n \approx \frac{N_c}{2 g} e^{-\varepsilon_d} \sqrt{\frac{4 g N_d}{N_c} e^{\varepsilon_d}}=\sqrt{\frac{N_c N_d}{g}}e^{-\frac{-\varepsilon_d}{2}} = \sqrt{\frac{N_c N_d}{g}}e^{-\frac{E_c - E_d}{2kT}}
$$

Рост $n$ с увеличением $T$ --- 1а область примесной ионизации.

$$F(T) = \frac{E_c+E_d}{2} + \frac{kT}{2} \ln \frac{N_d}{g N_c}$$
\vspace{2cm}

б) $\displaystyle \frac{4 g N_d}{N_c} e^{\varepsilon_d} \ll 1 \quad\left(\sqrt{1+x} \approx 1+\frac{x}{2} \quad x<1\right)$

\vspace{0.5cm}

$$
n \approx \frac{N_c}{2 g} e^{-\varepsilon_d}\left\{1+\frac{2g N_d}{N_c} e^{\varepsilon_d}-1\right\}
$$

б --- область истощения (насыщения примеси)

$$
n=N_c e^{\frac{F-E_c}{k T}}=N_d
$$

$$
F(T)=E_c-k T \ln \frac{N_d}{N_c}
$$

\vspace{2cm}

2. $\displaystyle n=\frac{n_i^2}{n}+N_d$

\vspace{0.5cm}

$$
n^2-N_d n-n_i^2=0
$$

$$
n=\frac{N_d}{2} \pm \sqrt{\frac{N_d^2}{4}+n_i^2}
$$

\vspace{2cm}
в) $\displaystyle \frac{N_d^2}{4} \gg n_i^2, \; n=\frac{N_d}{2}+\frac{N_d}{2}=N_d$ --- всё ещё область истощения примеси.

\vspace{2cm}
г) $\displaystyle \frac{N_d^2}{4}<<n_i^2, n \rightarrow n_i(T)$ --- интервал собственной ионизации. Поведение аналогично собственному полупроводнику.
\vspace{0.5cm}

\begin{figure}[h!]
    \centering
    \includegraphics[width=0.85\textwidth]{phys_11_4}
\end{figure}


\begin{figure}[h!]
    \centering
    \subfigure{
    \includegraphics[width=0.4\textwidth]{phys_11_5}}
    \subfigure{
    \includegraphics[width=0.5\textwidth]{phys_11_6}}
\end{figure}

$T_s$, $T_i$ --- нижняя и верхняя температура истощения.
\clearpage

\section{Неравновесные носители заряда (время жизни, механизмы рекомбинации). Уравнение непрерывности и примеры его использования.}

Неравновесные носители заряда возникают из-за:

\begin{enumerate}
    \item Внешних воздействий 
    \begin{enumerate}
        \item Облучение квантами электромагнитного излучения
        \item Облучение быстрыми частицами (электроны, нейтроны, $\gamma$-излучение)
        \item Приложение электрического поля 
        \item Ток в p-n-переходе
    \end{enumerate}
    \item Внутренних воздействий
    \begin{enumerate}
        \item Изменение температуры
    \end{enumerate}
\end{enumerate}

% \textbf{Неравновесная функция распределения}

% $$
% n(T)=\int f_n(E ; T) N_c(E) d E \approx \int \frac{N_c(E)}{e^{\frac{E-F_n}{kT}}+1} d E=N_c F_{1/2}\left(\eta_n\right)
% $$

% $$
% p(T)=\int f_p(E ; T) N_v(E) d E \approx \int \frac{N_v(E)}{e^{\frac{E-F_p}{kT}}+1} d E=N_v F_{1/2}\left(-\eta_p - \Delta \varepsilon \right)
% $$

% $f_n$, $f_p$ --- в общем виде не известны
% $F_n$, $F_p$ --- квазиуровни Ферми
% $\eta_n$, $\eta_p$ --- приведённые безразмерные квазиуровни Ферми


\textbf{Время жизни}


Простая механическая модель: 1 тип электронов l механизмов рекомбинации. $W_{nl}=S_{nl}P_l \langle V_{nl} \rangle$ ($\left[ \frac{1}{\text{см}} \right] = \left[ \text{см}^2 \text{см}^{-3} \frac{\text{см}}{\text{с}} \right]$) --- вероятность рекомбинации электронов по механизму l.

$S_{nl}$ --- эффективное сечение процесса рекомбинации электронов по механизму l.

$\frac{1}{W_{nl}}=\tau_{nl}$ --- среднее время жизни электрона при рекомбинации по механизму l.

$W_n=\sum \limits_l W_{nl} \quad \frac{1}{\tau_n} = \sum \limits_l \frac{1}{\tau_{nl}}$ 

\textbf{Основные механизмы рекомбинации}

1) По типу переходов 

а) Межзонная рекомбинация 

\begin{figure}[h!]
    \centering
    \includegraphics[width=0.3\textwidth]{phys_12_1}
\end{figure}

б) Рекомбинация через рекомбинационные уровни

\begin{figure}[h!]
    \centering
    \includegraphics[width=0.3\textwidth]{phys_12_2}
\end{figure}

в) Экситонная рекомбинация

г) Поверхностная рекомбинация

\pagebreak
2) По виду выделяемой энергии

а) Межзонная рекомбинация (излучательная)
\begin{figure}[h!]
    \centering
    \includegraphics[width=0.3\textwidth]{phys_12_3}
\end{figure}


б) Безизлучательная рекомбинация 

\begin{figure}[h!]
    \centering
    \includegraphics[width=0.3\textwidth]{phys_12_4}
\end{figure}

-- Фононная (вся энергия идёт на создание фонона)

\begin{figure}[h!]
    \centering
    \includegraphics[width=0.3\textwidth]{phys_12_5}
\end{figure}

-- Ударная (Оже) рекомбинация

\begin{figure}[h!]
    \centering
    \includegraphics[width=0.3\textwidth]{phys_12_6}
\end{figure}

\textbf{Уравнение непрерывности и примере его использования}

В выделенном объеме концентрация носителей заряда может меняться за счет генерации, рекомбинации, диффузии и дрейфа.

$$
\frac{d n}{d t}=\overbrace{g_{n E}-r_{n E}}^{G_n}+\overbrace{g_{n I}-r_{n I}}^{-R_n}+\frac{1}{e} \text { div } \bar{j}_n(r)
$$

$$
\frac{d p}{d t}=\overbrace{g_{p E}-r_{p E}}^{G_p}+\overbrace{g_{p I}-r_{p I}}^{-R_p}-\frac{1}{e} \text { div } \bar{j}_p(r)
$$

$g$ --- скорость генерации носителей заряда
$r$ --- скорость рекомбинации носителей заряда
$G_{n,p}$ --- эффективная скорость генерации
$R_{n,p}$ --- эффективная скорость рекомбинации
$E$ --- внешние
$I$ --- внитренние

\textit{Рекомбинация при монополярной генерации (линейная рекомбинация)}

Предположения:

\begin{enumerate}
    \item Выключаем генерацию в момент времени t=0
    \item Полупроводник однородный ($n(\bar{r})=const, \; \bar{E}=0$)
    \item $\Delta \ll n_0$ --- условие низкоуровневого возбуждения (инжекции)
    \item Генерация примесная
\end{enumerate}

$$
\frac{d n}{d t}= \overbrace{G_n}^{0}-R_n+\overbrace{\frac{1}{e} \mathrm{div}\bar{j}_n}^0 \Rightarrow \frac{d n}{d t}=-W_n \Delta n=-\frac{\Delta n}{\tau_\text{ж}}
$$

$$
\Delta n(t)=\Delta n(0) e^{-\frac{t}{\tau_\text{ж}}}, \quad n=n_0+\Delta n
$$

$\tau_\text{ж}$ --- время за которое значение $\Delta n$ уменьшится в $e$ раз.


\begin{figure}[h!]
    \centering
    \subfigure{
    \includegraphics[width=0.6\textwidth]{phys_12_7}}
    \subfigure{
        \includegraphics[width=0.3\textwidth]{phys_12_8}}
\end{figure}


\textit{Рекомбинация при биполярной генерации (квадратичная рекомбинация)}

Предположения:

\begin{enumerate}
    \item Выключаем генерацию в момент времени t=0
    \item Полупроводник однородный ($\bar{E}=0$)
    \item Межзонная генерация ($\Delta n = \Delta p$)
    \item Уровень возбуждения любой
\end{enumerate}

$\frac{d n}{d t}=\frac{d p}{d t}=-R_n=-\gamma\left(n p-n_0 p_0\right)$, $\gamma$ --- коэффициент межзонной рекомбинации.

$\left.\begin{array}{l}n=n_0+\Delta n \\ p=p_0+\Delta p\end{array}\right\} \rightarrow-\gamma\left(n p-n_0 p_0\right)$

$\displaystyle \frac{d n}{d t}=-\gamma\left(n_0 p_0+n_0 \Delta p+p_0 \Delta n+\Delta n \Delta p-n_0 p_0\right)$ 

$\displaystyle \frac{d n}{d t}=-\gamma\left(n_0 \Delta p+p_0 \Delta n+\Delta n \Delta p\right)$

\vspace{0.5cm}

а) Низкий уровень инжекции ($\Delta n, \Delta p \ll n_0, p_0$)

-- p-тип ($p_0\gg n_0$), $\frac{d(\Delta n)}{d t}=-r p_0 \Delta n \sim \Delta n$

$\Delta n(t)=\Delta n(0) e^{-\frac{t}{\tau_n}}, \tau = \frac{1}{\gamma p_0}$

-- n-тип ($n_0\gg p_0$), $\frac{d(\Delta n)}{d t}=-r n_0 \Delta p \sim \Delta p$

$\Delta p(t)=\Delta p(0) e^{-\frac{t}{\tau_p}}, \tau=\frac{1}{\gamma n_0}$

Времена жизни в процессах определяется концентрацией неосновных носителей заряда.

б) Высокий уровень инжекции ($\Delta n, \Delta p \gg n_0, p_0$)

$\frac{d(\Delta n)}{d t}=\frac{d(\Delta n)}{d t}=-\gamma(\Delta n)^2 \sim(\Delta n)^2$ --- квадратичная рекомбинация.

$\frac{d(\Delta n)}{(\Delta n)^2}=-\frac{d t}{\gamma} \rightarrow \frac{1}{\Delta n}=\gamma t+C$. Граничное условие $t=0, \Delta n=\Delta n(0) \Rightarrow C=\frac{1}{n(0)}$

$\Delta n(t)=\frac{\Delta n(0)}{1+\gamma \Delta n(0) t}$

Не можем вывести всего один параметр ($\tau_n$ --- время жизни), который характеризует весь процесс.

Мнгновенное время жизни (обобщение $\tau_n$).

$$
\frac{d n}{d t}=-\frac{\Delta n(t)}{\tau_n(t)} \rightarrow \tau_n=-\frac{\Delta n(t)}{\frac{d n}{d t}}
$$


\begin{figure}[h!]
    \centering
    \subfigure{
    \includegraphics[width=0.3\textwidth]{phys_12_9}}
    \subfigure{
        \includegraphics[width=0.3\textwidth]{phys_12_10}}
        \subfigure{
        \includegraphics[width=0.3\textwidth]{phys_12_11}}
\end{figure}

\textit{Релаксация избыточной концентрации при конечном внешнем возбуждении}

Предположения:

\begin{enumerate}
    \item Включаем генерацию в момент времени t=0
    \item Полупроводник однородный ($\bar{E}=0$)
    \item Генерация монополярная
    \item Низкий уровень возбуждения
\end{enumerate}

$\frac{d n}{d t}=G_n-R_n$,  $R_n=\frac{\Delta n}{\tau _n}$

$$\frac{d(\Delta n)}{\Delta n-G_n \tau_n}=-\frac{d t}{\tau_n}$$

$$\ln \left(\Delta n-G_n \tau_n\right)=-\frac{t}{\tau_n}+C$$

$$\Delta n-G_n \tau_n=B e^{-\frac{t}{\tau_n}}$$

Граничные условия $t=0, \; \Delta n(t)=0 \Rightarrow -G_n\tau_n$

$$\Delta n(t)=G_n \tau_n\left[1-e^{-\frac{t}{\tau_n}}\right]$$

$$t \rightarrow \infty \quad \Delta n(t)=G_n \tau_n=\Delta n_\text{стац}$$

\begin{figure}[h!]
    \centering
    \includegraphics[width=0.8\textwidth]{phys_12_12}
\end{figure}

Прямоугольный импульс внешнего напряжения $\rightarrow$ искажённый импульс с наростаниями и спадами (фотоприёмники).

Большое время жизни $\Rightarrow$ большая амплитуда (хорошо), но затягивание спада (плохо).

\section{Контакт металл-полупроводник (контактная разность потенциалов, энергетические диаграммы, вольтамперная характеристика).}

\textbf{Полупроводник n-типа. $\Phi_\text{м}<\Phi_\text{п}$}

\begin{figure}[h!]
    \centering
    \includegraphics[width=0.9\textwidth]{phys_13_1}
\end{figure}

$$
\frac{d E}{d x}=\frac{\rho}{\varepsilon \varepsilon_0} \rightarrow \frac{d E}{d x}<0 \quad E>0
$$


$$
E=-\frac{d \varphi}{d x} \rightarrow \frac{d \varphi}{d x}<0 \quad \varphi>0
$$

$$
U=-e \varphi \rightarrow U<0
$$

\textbf{Полупроводник n-типа. $\Phi_\text{м}>\Phi_\text{п}$}


\begin{figure}[h!]
    \centering
    \includegraphics[width=0.9\textwidth]{phys_13_2}
\end{figure}

$$
E=\frac{e n_0}{\varepsilon \varepsilon_0}\left(L_0-x\right) \quad \varphi(x)=-\frac{e n_0}{2 \varepsilon \varepsilon_0}\left(L_0-x\right)
$$

Граничное условие: $x=0, \varphi(x)=-\varphi_k$

$$
-\varphi_k=-\frac{e n_0}{2 \varepsilon \varepsilon_0}\left(L_0\right)^2
$$

$\displaystyle L_0=\sqrt{\frac{2 \varepsilon \varepsilon_0 \varphi_k}{e n_0}}$ --- в сильном поле $e\varphi \gg kT$


Можно сравнить с $l_D$ в слабом поле $e\varphi \ll kT$ 

$$
l_D=\sqrt{\frac{\varepsilon \varepsilon_0 k T}{e^2 n_0}}
$$

При $T=300$~K $\varphi_k=1$~B, $\frac{L_0}{l_D}=10$


Для полупроводника p-типа аналогично (на рисунках показан уровень ферми такого полупроводника штрихом ($F_p$)).

\textbf{Энергетические диаграммы контакта металл-полупроводник во внешнем поле}

Рассмотрим контакт металл-полупроводник n-типа (с обеднённым слоем)

\begin{figure}[h!]
    \centering
    \subfigure{
    \includegraphics[width=0.6\textwidth]{phys_13_3}}
    \subfigure{
        \includegraphics[width=0.3\textwidth]{phys_13_5}}
\end{figure}

Изменится изгиб зон и толщина слоя ОПЗ (уменьшится при $U>0$, увеличится при $U<0$).

Напряжение приложено в прямом направление если металл +, а полупроводник -.

Токи при контакте: $$
j_{\text{ТП}}=A T^2 e^{-\frac{\Phi_\text{м}}{k T}}, \;
j_{\text{ТМ}}=A T^2 e^{-\frac{\Phi_\text{м} + e(\varphi_k+U)}{k T}}
$$

$$
j=j_{\text{ТП}}-j_{\text{ТМ}}=\underbrace{A T^2 e^{-\frac{\Phi_\text{м}}{k T}}}_{j_s}\left(e^{\frac{e U}{k T}}-1\right)
$$


$j_s$ --- ток насыщения.

\textbf{ВАХ}

\begin{figure}[h!]
    \centering
    \includegraphics[width=0.4\textwidth]{phys_13_4}
\end{figure}

Электроннодефецитный контакт металл-полупроводник --- диод Шоттки. Электронноизбыточный --- омический контакт (невыпрямляющий) применяется для исследования свойств полупроводников.

\section{Электронно-дырочный переход (энергетические диаграммы, идеальная ВАХ, ток насыщения).}

Предположения:

\begin{enumerate}
    \item Резкий p-n переход (концентрация примеси меняется скачком).
    \item Невырожденный полупроводник.
    \item Температура соответствует области истощения примеси.
    \item Использован один и тот же полупроводник, но с разными примесями.
\end{enumerate}

\begin{figure}[h!]
    \centering
    \includegraphics[width=0.85\textwidth]{phys_14_1}
\end{figure}

$\Phi_p > \Phi_n \Rightarrow j_\text{Tp}<j_\text{Tn}$

В обоих случаях заряд создают ионы примеси.

$\displaystyle e \varphi_k=\varphi_p-\varphi_n=\left(x+E_c-F\right)-\left(x+E_c-F\right)=F_n-F_p= \left(E_c-k T \ln \frac{N_c}{N_d}\right)-\left(E_v+k T \ln \frac{N_N}{N_a}\right)=-k T \ln \frac{n_i^2}{N_c N_v} -k T \ln \frac{N_c}{N_d}-k T \ln \frac{N_v}{N_a}=k T \ln \frac{N_c N_v}{n_i^2} \cdot \frac{N_d}{N_c} \cdot \frac{N_a}{N_v}= k T \ln \frac{N_d N_a}{n_i^2}$


$\displaystyle \varphi_k=\frac{k T}{e} \ln \frac{N_d N_a}{n_i^2} \quad e \varphi_k \leq E_g \quad \varphi_k \sim 1 B$

$$
n_i^2=N_c N_v e^{-\frac{E_g}{k T}} \rightarrow E_g=-k T \ln \frac{n_i^2}{N_c N_v}
$$


\textbf{Энергетические диаграммы p-n-перехода во внешнем поле}

\begin{figure}[h!]
    \centering
    \subfigure{
    \includegraphics[width=0.6\textwidth]{phys_14_2}}
    \subfigure{
    \includegraphics[width=0.3\textwidth]{phys_14_3}}
\end{figure}

\textbf{ВАХ}

$j_D$, $j_E$ --- Диффузионный ток и дрейфовый (дрейфовый ток не преодолевает энергетический барьер) соответственно.

$$
j=\underbrace{\left(\bar{j}_{D_p}+\bar{j}_{E_p}\right)}_0 + \underbrace{\left(\bar{j}_{D_n}+\bar{j}_{E_n}\right)}_0=0
$$

$$
j_E=j_{E_n}+j_{E_p}, \quad j_D=j_{D_n}+j_{D_p}
$$

\begin{figure}[h!]
    \centering
    \includegraphics[width=0.4\textwidth]{phys_14_4}
\end{figure}

$$
j=j_s\left(e^{\frac{eU}{k T}}-1\right)
$$

$j_s=j_{E_s}+j_{E_p}$ --- токи неосновных носителей заряда.

Образование $j_s$:

\begin{enumerate}
    \item Неосновные носители заряда диффундируют к оптимизации
    \item Мгновенный переброс через ОПЗ
\end{enumerate}

\textbf{Токи через p-n-переход.}

Дрейфовые токи преодолевают потенциальный барьер. При этом дрейфовые и диффузионные токи попарно компенсируют друг друга.

Ток насыщения: $j_s=j_{E_n}+j_{E_p}=e n_i^2\left(\frac{L_n}{p_p \tau_n}+\frac{L_e}{n_n \tau_p}\right) \sim n_i^2 \sim e^{-E_{j / k T}}=j_s\left(E_g, T\right)$, $E_g \uparrow \downarrow i_s$, $T \uparrow \uparrow j_s$.


\textbf{Отличия реальной ВАХ от идеальной.}

\begin{enumerate}
    \item Появление линейного участка при больших напряжениях.
    \item Около 0 появляются искажения из-за генерации/рекомбинации в ОПЗ.
    \item Пробой p-n-перехода при больших обратных напряжениях.
\end{enumerate}


\begin{figure}[h!]
    \centering
    \includegraphics[width=0.4\textwidth]{real_vah}
\end{figure}


\section{Поверхностные состояния. Поверхностная проводимость. Эффект поля.}


Свойства полупроводника (электропроводность, рекомбинационные явления, оптические свойства и т.д.) сильно зависят от состояния поверхности (состав атмосферы, обработка). Причина зависимости --- поверхностные уровни (локализованные состояния, подобные уровням дефектов).

Причины возникновения поверхностных уровней:

1) На поверхности нарушается периодичность потенциала

\begin{figure}[h!]
    \centering
    \includegraphics[width=0.8\textwidth]{phys_15_1}
\end{figure}

(Для атомов на поверхности уравнение Шрёдингера имеет другие решения по отношению к атомам в объёме).

2) На поверхности не образуются парные ковалентные связи. Из-за этого появляются локализованные поверхностные состояния.

\begin{figure}[h!]
    \centering
    \includegraphics[width=0.4\textwidth]{phys_15_2}
\end{figure}

3) Дефекты на поверхности (результаты обработки, чужеродные атомы атмосферы, жидкости).

Все причины приводят к появлению заряда на поверхности, он приводит к появлению заряда под поверхностью, из-за этого в приповерхностном слое появляется поле. Поле приводит к изгибу зон на глубине порядка $L_D=\sqrt{\frac{\varepsilon \varepsilon_0 k T}{e^2 n_0}}$.

\begin{figure}[h!]
    \centering
    \subfigure{
    \includegraphics[width=0.45\textwidth]{phys_15_3}}
    \subfigure{
    \includegraphics[width=0.45\textwidth]{phys_15_4}}
\end{figure}

При толщине образца сопоставимой с глубиной проникновения поля свойства образца координально меняются (полупроводник n-типа начинает вести себя как полупроводник p-типа).


\textit{//для справки//}

\textbf{Решение уравнения Пуассона в ОПЗ}

Вычислим $\varphi(x)$, $E(x)$, $U(x)$, $Q_{ss}$, $Q_{sp}$.

Предположения:

\begin{enumerate}
    \item Невырожденный полупроводник
    \item Область истощения примеси
\end{enumerate}

$$
\frac{d^2 \varphi}{d x^2}=-\frac{\rho(x)}{\varepsilon \varepsilon_0}
$$

$$
\rho(x)=-e\left(n-p+N_a^{-}-N_d^{+}\right)=-e\left[n_0\left(e^{\frac{e \varphi}{k T}}-1\right)-p_0\left(e^{-\frac{e \varphi}{k T}}-1\right)\right]
$$


$$
n=N_c e^{\frac{F-\left(E_c-e \varphi\right)}{k T}}=n_0 e^{\frac{e \varphi}{k T}}
$$

$$
p=N_v e^{\frac{\left(E_v-e \varphi \right)-F}{k T}}=p_0 e^{-\frac{e \varphi}{k T}}
$$

$$
\frac{d^2 \varphi}{d x^2}=\frac{e}{\varepsilon \varepsilon_0}\left[n_0\left(e^{\frac{e \varphi}{k T}}-1\right)-p_0\left(e^{-\frac{e \varphi}{k T}}-1\right)\right]
$$

$$
\frac{e}{k T} \frac{d^2 \varphi}{d x^2}=\frac{e n_i}{k T}  \frac{e}{\varepsilon \varepsilon_0}\left[\frac{n_0}{n_i}\left(e^{\frac{e \varphi}{k T}}-1\right)-\frac{p_0}{n_i}\left(e^{-\frac{e \varphi}{k T}}-1\right)\right]
$$

Замена $Y=\frac{e\varphi}{kT}$, $\frac{n_0}{n_i}=\lambda$, $\frac{p_0}{n_i}=\frac{n_i}{n_0}=\lambda^{-1}$

$$
\frac{d^2 Y}{d x^2}=L_{D}^{-2}\left[\lambda\left(e^Y-1\right)-\lambda^{-1}\left(e^{-Y}-1\right)\right] \qquad \mid \cdot 2 \frac{d y}{d x}
$$

$$
2 \frac{d^2 Y}{d x^2} \frac{d Y}{d x}=\frac{d}{d x}\left(\frac{d Y}{d x}\right)^2=2 L_{D}^{-2}\left[\lambda\left(e^Y-1\right)-\lambda^{-1}\left(e^{-Y}-1\right)\right] \frac{d Y}{d x}
$$

Интегрируем и извлекаем квадратный корень.

$$
\frac{d Y}{d x}=\sqrt{2} L_{D}^{-1} F(\lambda ; Y)
$$

$$
F(\lambda ; Y)=\left[\lambda\left(e^Y-1\right)+\lambda^{-1}\left(e^{-Y}-1\right)+\left(\lambda^{-1}-\lambda\right) Y\right]^{\frac{1}{2}}
$$

$$
\frac{d y}{\sqrt{2} L_D^{-1} F(\lambda ; y)}=d x
$$

$$
x(Y)=\frac{L_D}{\sqrt{2}} \int \frac{d Y}{F(\lambda; Y)}
$$


$x(Y)=\frac{L_D}{\sqrt{2}} \int \frac{d Y}{F(\lambda; Y)}$ --- общее решение, можно получить только числено.

$\displaystyle Q_{s p}=\int_0^{\infty} \rho(x) d x=-\varepsilon \varepsilon_0 \int \frac{d^2 \varphi}{d x^2} d x=\varepsilon \varepsilon_0 \frac{k T}{e} \int \frac{d^2 Y}{d x^2} d x=-\left.\varepsilon \varepsilon_0 \frac{k T}{e} \frac{d Y}{d x}\right|_{0 ; Y=Y_s} ^{\infty}=0-\left(-\varepsilon \varepsilon_0 \frac{k T}{e} \sqrt{2} L_{D}^{-1} F\left(\lambda ; Y_s\right)\right)=\sqrt{2} e n_i L_D F\left(\lambda ; Y_s\right)
$


$$
\left|Q_{s p}\right|=\left|Q_{s s}\right| \rightarrow \sqrt{2} e n_i L_{D} F\left(\lambda ; Y_s\right)=\frac{e N_s}{g_s e^{\frac{E_s-F}{kT}}+1}=en_a
$$

$n_a$ --- количество электронов на акцепторном уровне.


\textbf{Поверхностная проводимость}

\begin{enumerate}
    \item Эффект поверхностной проводимости: изменение проводимости тонких пластин при изменении состояния поверхности ($Y_s$ --- поверхностный потенциал)
    \item Количественная характеристика эффекта --- удельная проводимость $\Delta G=G_s$
\end{enumerate}

$\Delta G = G_s = e \mu _n \Delta N + e \mu _p \Delta P$, $\Delta N$, $\Delta P$ --- поверхностные избытки электронов и дырок. $[\Delta N]=[\Delta P]=\text{см}^{-2}$

$$
\Delta N=\int_0^{\infty}\left[n(x)-n_0\right] d x, \quad \Delta P=\int_0^{\infty}\left[p(x)-p_0\right] d x
$$

$$
\Delta N=\int_0^{\infty}\left[n(x)-n_0\right] d x=n_0 \int_0^{\infty}\left(e^{\frac{e \varphi}{k T}}-1\right) \frac{d x}{d Y} d Y= n_0 \frac{L _D}{\sqrt{2}} \int \frac{e^Y-1}{F(\lambda ; Y)} d Y
$$

$$
\Delta P=\int_0^{\infty}\left[p(x)-p_0\right] d x=p_0 \int_0^{\infty}\left(e^{-\frac{e \varphi}{k T}}-1\right) \frac{d x}{d Y} d Y =p_0 \frac{L _D}{\sqrt{2}} \int_0^{\infty} \frac{e^{-Y}-1}{F(\lambda ; Y)} d Y
$$

Поверхностные интегралы можно вычислить численно или по таблица.


\textbf{$G_s (Y_s)$}

\begin{figure}[h!]
    \centering
    \includegraphics[width=0.6\textwidth]{G(Y)}
\end{figure}

\begin{enumerate}
    \item $Y_s=0$ (Плоские зоны), $G_s=0$.
    \item $Y_s>0$, $G_s \approx e \nu_n \Delta N \sim e^{Y_s}$. Рост $G_s>0$, режим обогащения.
    \item $Y_s \leq 0$, $G_s \approx e \nu_n \Delta N \sim e^{-Y_s}$. Обеднение, $G_s<0$ и уменьшение до $Y_{min}$.
    \item $Y_s < 0$, $G_s \approx e \nu_p \Delta P \sim e^{Y_s}$. Скорость наростания неосновных носителей заряда больше, чем скорость убывания основных, образуется инверсный слой $G_s \uparrow$.
\end{enumerate}



\textbf{Эффект поля (влияние поля на $G_s$)}

Эффект поля --- влияние внешнего электрического поля на величину поверхностной
проводимости.

\begin{figure}[h!]
    \centering
    \includegraphics[width=0.9\textwidth]{phys_15_5}
\end{figure}


\textit{Моделдь полевого транзистора}

\begin{enumerate}
    \item Плавная регулеровка $Y_s$: замер ОПЗ, $G_s$, изгиба зон
    \item С таким эффектом можно определить поверхностный потенциал $Y_s$
\end{enumerate}

\textit{Алгоритм определения $Y_s$}

\begin{enumerate}
    \item Расчёт $G_s(Y_s)$ ($\mu_n$, $\mu_p$, $n_0$, $p_0$)
    \item Измерить (экспериментально) эффект поля
\end{enumerate}

\begin{figure}[h!]
    \centering
    \includegraphics[width=0.6\textwidth]{phys_15_6}
\end{figure}

$Y_\text{эксперимент}-Y_\text{теор}=Y_s$ --- расчёт поверхностных состояний.

\section{Классификация кинетических явлений. Электропроводность. Эффективная масса проводимости. Температурные зависимости подвижности носителей заряда в полупроводниках.}

\textbf{Классификация}

По физическому смыслу:

\begin{enumerate}
    \item $\bar{E}$ --- электропроводность
    \item $\nabla T$ --- теплопроводность
    \item $\bar{E} ; \bar{B}$ --- гальвано-магнитные эййекты (эффект Холла, магнитосопротивление)
    \item $\nabla T \leftrightarrow \bar{E}$ --- термоэлектрические эффекты (эффект Зеебека, эффект Пельтье)
\end{enumerate}

По характеру протекания:

\begin{enumerate}
    \item Условия: изотермические, адиабатические
    \item Однородность (степень): характер распределения примеси, структура.
    \item Распределение векторов сил ($\perp$, $\parallel$, $\varphi$)
    \item Первичные и вторичные эффекты
    \item Дополнительные внешние воздействия (свет, облучение)
\end{enumerate}

\pagebreak
\textbf{Электропроводность}

$$E \neq 0 \quad \nabla F=\nabla T=\bar{B}=0 \quad \Rightarrow j=e^2 \mathcal{K}_{11} \bar{E}=\sigma \bar{E}$$

$$\sigma=e^2 \mathcal{K}_{11}=e^2 \frac{n}{m^*} \langle E \tau \rangle = e^2 \frac{n}{m^*} \langle \tau \rangle = en\left(\frac{e \langle \tau \rangle}{m ^*}\right)=en\mu$$

$r=\frac{e\langle\tau\rangle}{m^*}$ --- подвижность.


1) Неворожденный полупроводник $E=\frac{p^2}{2 m^*}$, $\sigma=\frac{e^2 n}{m^*}\langle\tau\rangle = \frac{e^2 n}{m^*}(k T)^p \tau_0 \frac{\Gamma\left(p+\frac{5}{2}\right)}{\Gamma\left(\frac{5}{2}\right)}$

2) Вырожденный полупроводник $\sigma=\frac{e^2 n}{m^*}\langle\tau\rangle = \frac{e^2 n}{m^*} \tau(F)$

\begin{figure}[h!]
    \centering
    \includegraphics[width=0.3\textwidth]{phys_16_1}
\end{figure}




I) Анизотропный квадратичный зако дисперсии $E=\frac{P_x^2}{2 m_1}+\frac{P_y^2}{2 m_2}+\frac{P_z^2}{2 m_3}$, $\sigma=\frac{e^2 n}{\tilde{m}^*}\langle\tau\rangle$
\begin{figure}[h!]
    \centering
    \includegraphics[width=0.3\textwidth]{phys_16_2}
\end{figure}


II) Многодолинный полупроводник ($Si$, $M=6$ --- число эквивалентных подзон).
\begin{figure}[h!]
    \centering
    \includegraphics[width=0.3\textwidth]{phys_16_3}
\end{figure}

$\displaystyle \sum_{i=1}^6 \frac{1}{m_i^*}=2\underbrace{\left(\begin{array}{ccc}\frac{1}{m_t} & 0 & 0 \\ 0 & \frac{1}{m_{t}} & 0 \\ 0 & 0 & \frac{1}{m_l}\end{array}\right)}_{1; 4}+2\underbrace{\left(\begin{array}{ccc}\frac{1}{m_l} & 0 & 0 \\ 0 & \frac{1}{m_t} & 0 \\ 0 & 0 & \frac{1}{m_t}\end{array}\right)}_{2; 5}+2\underbrace{\left(\begin{array}{ccc}\frac{1}{m_t} & 0 & 0 \\ 0 & \frac{1}{m_l} & 0 \\ 0 & 0 & \frac{1}{m_t}\end{array}\right)}_{3; 6}= 2\left(\frac{2}{m_t}+\frac{1}{m_2}\right)$

$\displaystyle \widetilde{\sigma}=\sum_{i=1}^6 \widetilde{\sigma_i}=e^2\left\langle\tau \right\rangle n_j \sum_{i=1}^6 \frac{1}{m_i^*}=e^2\langle\tau\rangle n_j 2 \cdot 3\left(\frac{2}{m_t}+\frac{1}{m_l}\right) \cdot \frac{1}{3} =e^2\langle \tau \rangle n \frac{1}{m_c^*} \text{--- скалярная электропроводность.}$ Результат справедлив для большинства кубических многодолинных полупроводников ($Ge$, $A^3B^5$).

$m_c^*$ --- эффективная масса проводимости ($\frac{1}{m_c^*}= \frac{1}{3} \left( \frac{2}{m_t}+\frac{1}{m_l} \right) $).
$n_j$ --- концентрация носителей заряда в каждой долине.
$n=6n_j$ --- общая концентрация носителей заряда.

$Si$ --- самый простой пример многодолинного полупроводника (поверхности постоянных энергий вытянуты вдоль осей).

\textbf{Температурная зависимость подвижности}

$$\mu = \frac{e}{m_c^*} \langle \tau \rangle = \left\{ \begin{array}{cc}
    \frac{e}{m_c^*} \tau_0(T)(k T)^p \frac{\Gamma\left(p+\frac{5}{2}\right)}{\Gamma\left(\frac{5}{2}\right)} & \text{невырожденный}\\
    \frac{e}{m_c^*} \tau(F)=\frac{e}{m_c^*} \tau_0(T) F^p & \text{вырожденный}
\end{array} \right.$$

$\tau (E) = \tau_0 (T)E^p$

$$
p=\left\{\begin{array}{cl}
-\frac{1}{2} & \text{точечные дефекты, дислокации, аккустические фононы}^* \\
0 & \text{нейтральные атомы примеси} \\
\frac{1}{2} & \text{оптические фононы}^* \\
\frac{3}{2} & \text{ионы примеси} 
\end{array}\right.
$$

$^*$ для фононов $\tau_0 \sim T^{-1}$

\begin{table}[h!]
    \centering
    \begin{tabular}{cc}
     Вырожденный  & Невырожденный \\ \hline \hline 
    \multicolumn{2}{c}{Рассеяние на ионах примеси} \\ 
    $\mu \sim T^{\frac{3}{2}}$ & $\mu \sim T^{0}$ \\ 
    \multicolumn{2}{c}{Рассеяние на точечных дифектах и дислокациях} \\ 
    $\mu \sim T^{-\frac{1}{2}}$ & $\mu \sim T^{0}$ \\ 
    \multicolumn{2}{c}{Рассеяние на атомах примеси} \\  
    $\mu \sim T^{0}$ & $\mu \sim T^{0}$ \\ 
    \multicolumn{2}{c}{Рассеяние на аккустических фононах} \\ 
    $\mu \sim T^{-\frac{3}{2}}$ & $\mu \sim T^{-1}$ \\  
    \multicolumn{2}{c}{Рассеяние на оптических фононах} \\ 
    $\mu \sim T^{-\frac{1}{2}}$ & $\mu \sim T^{-1}$ \\ \hline 
    \end{tabular}
\end{table}

\begin{figure}[h!]
    \centering
    \includegraphics[width=0.9\textwidth]{phys_16_4}
\end{figure}

\clearpage
\begin{figure}[h!]
    \textit{//для справки//}

    Температурная зависимость электропроводности.
    \centering
    \includegraphics[width=0.8\textwidth]{phys_16_5}
\end{figure}

\pagebreak

\section{Оптические характеристики полупроводников. Определение параметров полупроводников по спектрам оптического поглощения.}

$I_0$ --- интенсивность светового излучения (количество квантов на единицу площади в единицу времени).

\textit{Отражение и пропускание}

\begin{figure}[h!]
    \centering
    \includegraphics[width=0.4\textwidth]{phys_17_1}
\end{figure}

$R=\frac{I_R}{I_0}$, $T=\frac{I_T}{I_0}$

\textit{Поглощение}

\begin{figure}[h!]
    \centering
    \includegraphics[width=0.4\textwidth]{phys_17_2}
\end{figure}

$dI = -\alpha I dx$, $I=I_0 e^{-\alpha x}$

$R(\lambda; \nu; \hbar \nu)$ --- спектр отражения
$T(\lambda; \nu; \hbar \nu)$ --- спектр пропускания
$\alpha(\lambda; \nu; \hbar \nu)$ --- спектр поглощения

$\alpha$ --- главный коэффициент (все остальные можно выразить через него).

\textbf{Определение параметров полупроводников по спектрам оптического поглощения.}

\textit{Собственное поглощение}

Реализуется в полупроводниках с прямозонной структурой, у которых экстремумы валентной зоны расположены в одной точке зоны Бриллюэна.

\begin{figure}[h!]
    \centering
    \includegraphics[width=0.4\textwidth]{phys_17_3}
\end{figure}

$$E^\prime = E + h\nu$$

$$\hbar k^\prime = \hbar k + \hbar k_p$$ 

\noindent $k_p$ --- волновой вектор фотона. $k_p \ll \Delta k_\text{З.Бр.}$, можно принебречь им в законе сохранения энергии.

$\alpha = A (h \nu -E_g)^m$, нужно построить зависимость в линейном виде и найти энергию запрещённой зоны (так определяют зависимость $E_g(T)$).

\begin{figure}[h!]
    \centering
    \subfigure{
    \includegraphics[width=0.6\textwidth]{phys_17_4}}
    \subfigure{
    \includegraphics[width=0.3\textwidth]{phys_17_5}}
\end{figure}


\textit{Собственное поглощение. Непрямые переходы.}

Реализуется в непрямой зонной структуре. Требуется участие частицы с волновым вектором, стравнимым с зоной Бриллюэна.

\begin{figure}[h!]
    \centering
    \includegraphics[width=0.9\textwidth]{phys_17_6}
\end{figure}

$$E^\prime = E + h\nu \pm E_p$$

$$\hbar \bar{k}^\prime = \hbar \bar{k} + \underbrace{\hbar \bar{k}_\text{фотона}}_{=0}+\hbar \bar{k}_p$$

$$N_p ^0 = \frac{1}{e^{\frac{E_g}{kT}}-1}$$


$$\alpha (h\nu) = \underbrace{\Delta_1 f_1(N_p)(h\nu - E_g - E_p)^2}_\text{с испусканием фононов} + \underbrace{A_2f_2(N_p) (h\nu - E_g + E_p)^2}_\text{с поглощением фононов}$$

$E_p$ --- энергия фонона на краю зоны Бриллюэна.

\begin{figure}[h!]
    \centering
    \includegraphics[width=0.4\textwidth]{phys_17_7}
\end{figure}

При понижении температуры исчезает ветвь, отвечающая за испускание с поглощением.

\textit{Осцилляции межзонного магнитосопротивления.}

Осцилляции межзонного магнитосопротивления позволяют с высокой точностью определить энергию запрещённой зоны.

$B \gg 0$, $B \parallel z$

\begin{figure}[h!]
    \centering
    \includegraphics[width=0.4\textwidth]{phys_17_8}
\end{figure}


\textit{Циклотронный резонанс.}

Позволяет точно определить циклотронную массу.

$m^*_c=\frac{eB}{\omega_\text{рез}}$

\begin{figure}[h!]
    \centering
    \includegraphics[width=0.4\textwidth]{phys_17_9}
\end{figure}

\textit{Примесное поглощение.}

При высокой чувствительности измерительного оборудования на спектре поглощения можно наблюдать пики примесного поглощения, по которым можно определить энергию донорного уровня и положение возбуждённых уровней (из тонкой структуры).

\begin{figure}[h!]
    \centering
    \includegraphics[width=0.8\textwidth]{phys_17_10}
\end{figure}

\textit{Поглощение свободных носителей заряда.}

\begin{enumerate}
    \item Только непрямые переходы.
    \item Участие фононови дефектов решётки.
\end{enumerate}

Даёт информацию об основном механизме рассеяния.

$$\alpha \sim \frac{1}{(h\nu)^{2+p}}$$

\begin{figure}[h!]
    \centering
    \includegraphics[width=0.8\textwidth]{phys_17_11}
\end{figure}

$$
p=\left\{\begin{array}{cl}
-\frac{1}{2} & \text{точечные дефекты, дислокации, аккустические фононы} \\
0 & \text{нейтральные атомы примеси} \\
\frac{1}{2} & \text{оптические фононы} \\
\frac{3}{2} & \text{ионы примеси} 
\end{array}\right.
$$